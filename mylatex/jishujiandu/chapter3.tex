\chapter{热控技术监督事件报告}
\begin{table}[htbp]
	\centering
	\caption{高压至中压双减减压阀波动事件报告}\label{one}
\begin{tabular}{|c|c|c|c|}
\hline
	企业名称 & \multicolumn{3}{|c|}{中节能潞安电力节能服务有限公司}\tabularnewline
\hline
	设备名称&1号净烟气粉尘仪&发生时间&2024年03月16日\tabularnewline  
\hline
	事件概况&\multicolumn{3}{|c|}{
		\begin{minipage}[c][70ex][c]{35em}
		2024年3月05日定期对1号粉尘仪检查标定投运后测量数据在同样工况下相对比之前波动幅度较大,粉尘数据变得非常敏感,不好控制。2024年3月07日23:23粉尘数据偏高,进行零点标定后观察一小时数据稳定,早上发现滤芯及取样管内有水珠,对高效过滤器和凝聚过滤器进行更换。2024年3月09日04:55粉尘数据变为0mg/m3,仪表参数显示旁路流量低,进一步检查发现旁路流量计入口流量偏低,检查入口处取样管线和接头有较为严重的结晶已影响正常采样,疏通结晶物后旁路流量有所恢复但是还未达到需求值(21SLPM),进一步检查采样泵出力不足,更换采样泵后流量恢复正常,粉尘仪可以正常投入运行。2024年3月11日粉尘数据波动至50mg/m3,仪表参数显示相对湿度30以上,旁路流量波动较大,取样管线内积水较为严重,对取样管内积水进行清理吹扫,清理检查散射表、采样喷嘴、更换旁路流量计、样气流量计、采样泵和各滤芯(金属滤芯、高效过滤器、凝聚滤芯)后恢复测量过程中取样管路内很快又有积水,且粉尘仪投运后湿度变大(50以上),粉尘仪无法正常投运,进一步判断可能粉尘仪探头内稀释气配比有问题,整体更换取样探杆后取样正常(取样管内无积水,湿度下降至3左右)各流量恢复正常,粉尘仪投运后粉尘数据恢复正常,且在相同工况下波动幅度变小。对替换下来的采样探杆进一步检查确认为稀释模块与探杆连接接头腐蚀漏气造成喷嘴流量偏大且不稳定造成粉尘数据偏高且波动大。
		\end{minipage}
	}\tabularnewline  
\hline
	原因分析&\multicolumn{3}{|c|}{
		\begin{minipage}[c][16ex][c]{35em}
			1、稀释模块与探杆连接接头腐蚀漏气,导致稀释气泄漏,测量过程中喷嘴流量偏大,烟气相对湿度偏大,粉尘数据偏高波动较大;\\
			2、定期检查项目只是对稀释模块,喷嘴进行清理检查,未进一步对连接接头上结晶进行清理检查。
		\end{minipage}
	}\tabularnewline  
\hline
	采取的措施&\multicolumn{3}{|c|}{
		\begin{minipage}[c][22ex][c]{35em}
1、清理其他3台粉尘仪探杆尾部结晶并检查金属连接件是否有腐蚀情况,根据检查情况进行更换;
2、每季度对粉尘仪探杆尾部进行清理检查,查看金属连接件腐蚀情况并视腐蚀情况进行更换;
4、将烟道内部探杆列入设备停运检修项目,进行彻底清理检查,重点进行系统气密性检查;
5、将CEMS定期工作检查项目表格化,每次定检项目形成清单记录,针对烟道内探杆检查清理需专人进行验收确认。
		\end{minipage}
	}\tabularnewline  
\hline
	\multirow{2}{*}{监督专责任人} & \multirow{2}{*}{赵华鹏}&联系电话&\tabularnewline  
	\cline{3-4}&&邮箱&\tabularnewline
\hline
	\parbox[c][8ex][c]{6em}{生产副厂长或\\总\,\,\,\,工\,\,\,\,程\,\,\,\,师}&&日期&年月日\tabularnewline  
\hline
\end{tabular}
\end{table}
\clearpage
\begin{table}[htbp]
	\centering
	\caption{5号汽轮机轴向位移2测点显示偏高且波动}\label{sec}
\begin{tabular}{|c|c|c|c|}
\hline
	企业名称 & \multicolumn{3}{|c|}{中节能潞安电力节能服务有限公司}\tabularnewline
\hline
	设备名称&5号汽轮机轴向位移1&发生时间&2023年07月25日11时20分\tabularnewline  
\hline
	事件概况&\multicolumn{3}{|c|}{
		\begin{minipage}[c][2ex][c]{35em}
		5号汽轮机轴向位移2数据波动较大
		\end{minipage}
	}\tabularnewline  
\hline
	原因分析&\multicolumn{3}{|c|}{金积品牌前置器质量问题,导致测量异常}\tabularnewline  
\hline
	采取的措施&\multicolumn{3}{|c|}{
\begin{minipage}[c][50ex][c]{35em}
			\begin{enumerate}
				\item 更换新的前置器(已申报备件,近两周到货后更换)在未更换前置器前每日巡检观察轴向位移2测点曲线,如果参数显示偏高0.1mm时,热控专业退出保护,重新插拔探头与前置器接头让参数恢复正常后再投入保护;
				\item 更换其他品牌设备,根据之前轴振动的8mm电涡流探头试验结果,更换为本特利品牌前置器后设备参数运行平稳,由于前置器测量与定位特殊性,需要将电涡流探头和前置器整套进行更换后方可正常运行;
			\end{enumerate}
		\end{minipage}
}\tabularnewline  
\hline
	\multirow{2}{*}{监督专责任人} & \multirow{2}{*}{赵华鹏}&联系电话&\tabularnewline  
	\cline{3-4}&&邮箱&\tabularnewline
\hline
	\parbox[c][8ex][c]{6em}{生产副厂长或\\总\,\,\,\,工\,\,\,\,程\,\,\,\,师}&&日期&年月日\tabularnewline  
\hline
\end{tabular}
\end{table}
\clearpage
\begin{table}[htbp]
	\centering
	\caption{1号汽轮机DEH系统DI卡件通道故障报警}\label{sec}
\begin{tabular}{|c|c|c|c|}
\hline
	企业名称 & \multicolumn{3}{|c|}{中节能潞安电力节能服务有限公司}\tabularnewline
\hline
	设备名称&1号汽轮机DEH控制系统卡件&发生时间&2023年10月03日11时20分\tabularnewline  
\hline
	事件概况&\multicolumn{3}{|c|}{
		\begin{minipage}[c][25ex][c]{35em}
		Ticon控制系统硬件报警,现场检查诊断为其中DI卡件32通道报警(检测无效),经过进一步检查确认为DI卡件内部三冗余通道防粘连检测异常报警,初步判断为该通道内部异常,目前已拆除该通道接线,报警已消除。
		\end{minipage}
	}\tabularnewline  
\hline
	原因分析&\multicolumn{3}{|c|}{
\begin{minipage}[c][30ex][c]{35em}
		Ticon控制系统DI卡件31通道内部检测回路故障
		\end{minipage}
}\tabularnewline  
\hline
	采取的措施&\multicolumn{3}{|c|}{
\begin{minipage}[c][50ex][c]{35em}
			\begin{enumerate}
				\item 更换新的前置器(已申报备件,近两周到货后更换)在未更换前置器前每日巡检观察轴向位移2测点曲线,如果参数显示偏高0.1mm时,热控专业退出保护,重新插拔探头与前置器接头让参数恢复正常后再投入保护;
				\item 更换其他品牌设备,根据之前轴振动的8mm电涡流探头试验结果,更换为本特利品牌前置器后设备参数运行平稳,由于前置器测量与定位特殊性,需要将电涡流探头和前置器整套进行更换后方可正常运行;
			\end{enumerate}
		\end{minipage}
}\tabularnewline  
\hline
	\multirow{2}{*}{监督专责任人} & \multirow{2}{*}{赵华鹏}&联系电话&\tabularnewline  
	\cline{3-4}&&邮箱&\tabularnewline
\hline
	\parbox[c][8ex][c]{6em}{生产副厂长或\\总\,\,\,\,工\,\,\,\,程\,\,\,\,师}&&日期&年月日\tabularnewline  
\hline
\end{tabular}
\end{table}