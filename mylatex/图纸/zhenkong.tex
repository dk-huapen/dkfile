\ifx \allfiles \undefined		%编译PPT时注释该行
\documentclass{ctexart}				%编译PPT时注释该行
\usepackage{ifthen}
%\usepackage[a3paper,landscape,showframe,margin=1.25in]{geometry}
%\usepackage[a3paper,landscape,margin=1.1in]{geometry}
\usepackage[a3paper,landscape,top=1.25in,bottom=0.8in,left=1in,right=1in]{geometry}
\usepackage{tikz,units}
\usepackage{circuitikz}
\usepackage{subfig}
\usetikzlibrary{backgrounds,circuits.ee.IEC.relay}
\usetikzlibrary{positioning}
\usepackage{tikzsymbols}
\usepackage{pgffor}
\usetikzlibrary {math}
\usepackage{lastpage}
\usepackage{fancyhdr}
\pagestyle{fancy}
\fancyhf{}
\fancyhead[ER,OL]{\heiti \zihao{-5} 热工专业图纸}
\fancyhead[OR,EL]{\heiti \zihao{-5} \leftmark}
\fancyfoot[CE,CO]{热工专业图纸~第~\thepage~页(共 \pageref{LastPage} 页)}
\renewcommand{\headrulewidth}{0.4pt}
\renewcommand{\footrulewidth}{0.4pt}
\tikzset{
box/.style={rectangle,minimum height=17pt,minimum width=20pt,text=red}
}
\tikzset{
boxA/.style={rectangle,minimum height=17pt,minimum width=20pt,draw=black}
}
\tikzset{
boxB/.style={rectangle,minimum height=17pt,minimum width=30pt,draw=black}
}
\tikzset{
boxC/.style={rectangle,minimum height=17pt,minimum width=120pt,draw=black}
}
\tikzset{
boxD/.style={minimum width=140pt,above left}
}

\lhead{热工专业图纸}
\rhead{真空泵就地控制柜电气原理图}
\cfoot{热工专业图纸~第~\thepage~页 (共 \pageref{LastPage} 页)}
\begin{document}
		\else						%编译PPT时注释该行
		\chapter{真空泵就地控制柜电气原理图}	%编译PPT时注释该行
		\newpage
		\fi						%编译PPT时注释该行
\begin{center}
{\huge 真空泵就地控制柜电气原理图}
\end{center}
\begin{center}

	\begin{figure}[h]
\subfloat[接线端子图]{
\label{fig:improved_subfig_a}
		\begin{minipage}{410pt}
\begin{tikzpicture}[circuit ee IEC relay,thick,scale=1]

\node[boxA] (A1) at (0,0) {1};
\node[box,above=0 of A1] (A) {X1};
\foreach \i in {1,2,...,3}
{
\tikzmath{
int \j;
\j = \i+1;
}
\node[boxB,left=0 of A\i] (B\i) {};


\node[boxA,below=0 of A\i] (A\j) {\j};

\node[boxB,left=0 of A\j] (B\j) {};

}
\node[boxC,right=0 of A1] (C1) {220VAC电源进线$+$};
\node[boxC,right=0 of A2] (C2) {220VAC电源进线$-$};
\draw[dashed] (B1) -- ++(-2,0);
\draw[dashed] (B2) -- node[boxD]{来至电子间220VAC电源柜} ++(-2,0);
\node[boxC,right=0 of A3] (C3) {L};
\node[boxC,right=0 of A4] (C4) {N};
	\end{tikzpicture}
\begin{tikzpicture}[circuit ee IEC relay,thick,scale=1]
\node[boxA] (A1) at (0,0) {1};
\node[box,above=0 of A1] (A) {X2};
\foreach \i in {1,2,...,7}
{
\tikzmath{
int \j;
\j = \i+1;
}
\node[boxB,left=0 of A\i] (B\i) {};


\node[boxA,below=0 of A\i] (A\j) {\j};

\node[boxB,left=0 of A\j] (B\j) {};

}
\node[boxC,right=0 of A1] (C1) {打开入口蝶阀电磁阀$+$};
\node[boxC,right=0 of A2] (C2) {打开入口蝶阀电磁阀$-$};
\node[boxC,right=0 of A3] (C3) {关闭入口蝶阀电磁阀$+$};
\node[boxC,right=0 of A4] (C4) {关闭入口蝶阀电磁阀$-$};
\node[boxC,right=0 of A5] (C5) {压力开关$NO$};
\node[boxC,right=0 of A6] (C6) {压力开关$COM$};
\node[boxC,right=0 of A7] (C7) {电气开关已合闸$+$};
\node[boxC,right=0 of A8] (C8) {电气开关已合闸$-$};
\draw[dashed] (B7) -- ++(-2,0);
\draw[dashed] (B8) -- node[boxD]{来至电气MCC控制柜} ++(-2,0);
	\end{tikzpicture}
\begin{tikzpicture}[circuit ee IEC relay,thick,scale=1]
\node[boxA] (A1) at (0,0) {1};
\node[box,above=0 of A1] (A) {X3};
\foreach \i in {1,2,...,5}
{
\tikzmath{
int \j;
\j = \i+1;
}
\node[boxB,left=0 of A\i] (B\i) {};


\node[boxA,below=0 of A\i] (A\j) {\j};

\node[boxB,left=0 of A\j] (B\j) {};

}
\node[boxC,right=0 of A1] (C1) {入口蝶阀开反馈$+$};
\node[boxC,right=0 of A2] (C2) {入口蝶阀开反馈$-$};
\node[boxC,right=0 of A3] (C3) {入口蝶阀关反馈$+$};
\node[boxC,right=0 of A4] (C4) {入口蝶阀关反馈$-$};
\node[boxC,right=0 of A5] (C5) {液位变送器$+$};
\node[boxC,right=0 of A6] (C6) {液位变送器$-$};
\draw[dashed] (B5) -- ++(-2,0);
\draw[dashed] (B6) -- node[boxD]{来至DCS机柜} ++(-2,0);
	\end{tikzpicture}
\begin{tikzpicture}[circuit ee IEC relay,thick,scale=1]
\node[boxA] (A1) at (0,0) {1};
\node[box,above=0 of A1] (A) {X4};
\foreach \i in {1,2,...,5}
{
\tikzmath{
int \j;
\j = \i+1;
}
\node[boxB,left=0 of A\i] (B\i) {};


\node[boxA,below=0 of A\i] (A\j) {\j};

\node[boxB,left=0 of A\j] (B\j) {};

}
\node[boxC,right=0 of A1] (C1) {补水电磁阀$+$};
\node[boxC,right=0 of A2] (C2) {补水电磁阀$-$};
\draw[dashed] (B1) -- node[boxD]{} ++(-2,0);
\node[boxC,right=0 of A3] (C3) {液位高开关$NC$};
\node[boxC,right=0 of A4] (C4) {液位高开关$COM$};
\node[boxC,right=0 of A5] (C5) {液位低开关$NO$};
\node[boxC,right=0 of A6] (C6) {液位低开关$COM$};

	\end{tikzpicture}
		\end{minipage}
}
\subfloat[控制回路]{
\label{fig:improved_subfig_b}
		\begin{minipage}{600pt}
\begin{tikzpicture}[circuit ee IEC relay,thick,
	x=12\tikzcircuitssizeunit,
	y=10\tikzcircuitssizeunit,
	every term/.style={gray,font=\scriptsize},
	every term'/.style=every term,
	every term''/.style=every term]
		
			\draw (0,0) node [shape=coordinate](A){} -- ++(1,0) node [contact,name=A1]{}  -- ++(1,0) node [contact,name=B1]{}
				-- ++(1,0)node [contact,name=C1]{}  -- ++(1,0)node [contact,name=D1]{} -- ++(1,0) node [contact,name=E1]{} -- ++(1,0) node [shape=coordinate](F1){} ;
		\draw (A1)
		to [bulb={info=HL1,term=$X1$,term'=$X2$}] ++(0,1) -- ++(0,3)
		node [contact,name=A2]{};
		\draw (B1)
		to [relay coil={info=$K1$,term=$13$,term'=$14$}] ++(0,1)
		-- ++(0,2)
		node [contact,name=B2]{}
		to [make contact={info=$IE$,term=$X2-7$,term'=$X2-8$,term''=电气开关合闸反馈}] ++(0,1)
		node [contact,name=B3]{};
		\draw (C1)
		to [relay coil={info=$YA1$,term=$1$,term'=$2$,term''=开门电磁阀}] ++(0,1)
		node [contact,name=C2]{}
		to [make contact={info=$47b$,term=$X2-5$,term'=$X2-6$,term''=压力开关}] ++(0,1)
		node [contact,name=C3]{}
		-- ++(0,1)
		to [make contact={info=$K1$,term=$5$,term'=$9$}] ++(0,1)
		node [contact,name=C4]{};
		\draw (D1)
		to [relay coil={info=$YA2$,term=$1$,term'=$2$,term''=关门电磁阀}] ++(0,1)
		-- ++(0,2)
		to [break contact={info=$K1$,term=$4$,term'=$12$}] ++(0,1)
		node [contact,name=D2]{};
		\draw (E1)
		to [relay coil={info=$K2$,term=$13$,term'=$14$}] ++(0,1)
		node [contact,name=C2]{}
		to [make contact={info=$67c$,term=$X4-5$,term'=$X4-6$,term''=液位低开关}] ++(0,1)
		node [contact,name=C3]{}
		-- ++(0,1)
		to [break contact={info=$68c$,term=$X4-3$,term'=$X4-4$,term''=液位高开关}] ++(0,1)
		node [contact,name=C4]{};
		\draw (C2) -- ++(0.5,0)
		to [make contact={info=$K2$,term=$5$,term'=$9$}] ++(0,1)
		-- (C3);
		\draw (F1)
		to [relay coil={info=$YA3$,term=$1$,term'=$2$,term''=补水电磁阀}] ++(0,1)
		-- ++(0,2)
		to [make contact={info=$K2$,term=$8$,term'=$12$}] ++(0,1)
		node [shape=coordinate](F2){};
		\draw (F2) -- (A2) -- ++(-1,0)
		to [make contact={name=QF,term=$L$,term'=$1$}] ++(-1,0)
		node[above,red]{L};
		\draw (A) -- ++(0,3.5)
		to [make contact={term=$N$,term'=$2$}] ++(-1,0)
		node[above,blue]{N};
\draw[dashed](QF.mid) -- ++(0,-0.8) node[below,draw,solid,minimum size=3mm,label={[left]:$QF$}]{};

	\end{tikzpicture}
	\begin{description}
		\item[气动蝶阀联锁]电机运行且压力开关动作(-72KPa)联锁打开气动蝶阀,电机停运联锁关闭气动蝶阀
		\item[补水电磁阀联锁]液位低开关动作联锁打开补水电磁阀,液位高开关动作联锁关闭补水电磁阀
		\item[注意事项]气动蝶阀电磁阀为双电控电磁阀,为保证气动蝶阀正常动作,必须将电磁阀强制开关螺钉旋至O位
		\item[优化方向]目前电磁阀指令为电平指令,优化为脉冲指令即可实现同样功能还能延长电磁阀线圈使用寿命
	\end{description}

		\end{minipage}
}
\caption{真空泵就地控制柜电气原理图}
\label{fig:improved_subfig}
	\end{figure}
\end{center}
\footnote{1、2、4、5号汽轮机8台真空泵就地控制柜使用该图纸}
		\ifx \allfiles \undefined
\end{document}

\fi
