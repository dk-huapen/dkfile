
\documentclass[12pt,hyperref={CJKbookmarks=true}]{beamer} %14pt为字体尺寸,默认值11pt.有8-12;14;17;20;bigger;smaller
\usepackage{ifthen}
%\usepackage[a3paper,landscape,showframe,margin=1.25in]{geometry}
%\usepackage[a3paper,landscape,margin=1.1in]{geometry}
\usepackage[a3paper,landscape,top=1.25in,bottom=0.8in,left=1in,right=1in]{geometry}
\usepackage{tikz,units}
\usepackage{circuitikz}
\usepackage{subfig}
\usetikzlibrary{backgrounds,circuits.ee.IEC.relay}
\usetikzlibrary{positioning}
\usepackage{tikzsymbols}
\usepackage{pgffor}
\usetikzlibrary {math}
\usepackage{lastpage}
\usepackage{fancyhdr}
\pagestyle{fancy}
\fancyhf{}
\fancyhead[ER,OL]{\heiti \zihao{-5} 热工专业图纸}
\fancyhead[OR,EL]{\heiti \zihao{-5} \leftmark}
\fancyfoot[CE,CO]{热工专业图纸~第~\thepage~页(共 \pageref{LastPage} 页)}
\renewcommand{\headrulewidth}{0.4pt}
\renewcommand{\footrulewidth}{0.4pt}
\tikzset{
box/.style={rectangle,minimum height=17pt,minimum width=20pt,text=red}
}
\tikzset{
boxA/.style={rectangle,minimum height=17pt,minimum width=20pt,draw=black}
}
\tikzset{
boxB/.style={rectangle,minimum height=17pt,minimum width=30pt,draw=black}
}
\tikzset{
boxC/.style={rectangle,minimum height=17pt,minimum width=120pt,draw=black}
}
\tikzset{
boxD/.style={minimum width=140pt,above left}
}
 % 載入封包與文檔配置
% 参考文献样式
% 使用biblatex-gb7714-2015宏包提供的\footfullcite命令实现脚注引用。
\usepackage[backend=biber,autolang=hyphen,style=gb7714-2015,
gbtype=true,gbalign=gb7714-2015,
doi=false,url=false,isbn=false]{biblatex}
\tikzset{
dpu/.style={rectangle,rounded corners=2pt,
minimum width=35pt,minimum height=40pt,inner sep=5pt,
draw=black,fill=black!20,scale=0.8},
switch/.style={rectangle,rounded corners=5pt,
minimum width=15pt,minimum height=50pt,inner sep=5pt,
draw=black,fill=black!20}
}




\makeatletter
\def\progressbar@progressbar{} % the progress bar
\newcount\progressbar@tmpcounta% auxiliary counter
\newcount\progressbar@tmpcountb% auxiliary counter
\newdimen\progressbar@pbht %progressbar height
\newdimen\progressbar@pbwd %progressbar width
\newdimen\progressbar@rcircle % radius for the circle
\newdimen\progressbar@tmpdim % auxiliary dimension

\progressbar@pbwd=\linewidth
\progressbar@pbht=1pt
\progressbar@rcircle=2.5pt

% the progress bar
\def\progressbar@progressbar{%

    \progressbar@tmpcounta=\insertframenumber
    \progressbar@tmpcountb=\inserttotalframenumber
    \progressbar@tmpdim=\progressbar@pbwd
    \multiply\progressbar@tmpdim by \progressbar@tmpcounta
    \divide\progressbar@tmpdim by \progressbar@tmpcountb

  \begin{tikzpicture}
    \draw[pbblue!30,line width=\progressbar@pbht]
      (0pt, 0pt) -- ++ (\progressbar@pbwd,0pt);

    \filldraw[pbblue!30] %
      (\the\dimexpr\progressbar@tmpdim-\progressbar@rcircle\relax, .5\progressbar@pbht) circle (\progressbar@rcircle);

    \node[draw=pbblue!30,text width=4em,align=center,inner sep=1pt,
      text=pbblue!70,anchor=east] at (0,0) {\textnormal{%
             \pgfmathparse{\insertframenumber*100/\inserttotalframenumber}%
             \pgfmathprintnumber[fixed,precision=2]{\pgfmathresult}\,\%%
        }%
};
  \end{tikzpicture}%
}

\addtobeamertemplate{headline}{}
{%
  \begin{beamercolorbox}[wd=\paperwidth,ht=4ex,center,dp=1ex]{white}%
    \progressbar@progressbar%
  \end{beamercolorbox}%
}
\makeatother








\begin{document}
	
	\kaishu
	
	\title{DEH控制系统培训}
\subtitle{Hollysys/Tricon系统}
	\author{热控专业}
	\institute[检修部热控班组]{热控班组}
	\date{\today}
	\begin{frame}
		\titlepage
	\end{frame}
\begin{frame}{\textbf{目录}}
\tableofcontents
\end{frame}
	\section{DEH系统概述}
	\subsection{DEH系统介绍}
	\begin{frame}{DCS 800XA系统概念}{DCS系统概念阐述:分散控制集中管理}
		DCS(DistributedControlSystem)是分散控制系统的简称,一般习惯称为集散控制系统。它是一个由过程控制级和过程监控级组成的以通信网络为纽带的多级计算机系统,综合了计算机(Computer)、通讯(Communication)、显示(CRT)和控制的技术(Control)等4C技术,其基本思想是分散控制、集中操作、分级管理、配置灵活、组态方便。
	\end{frame}

\begin{frame}{800XA系统概念与架构}{DCS系统概念讨论:概念和思想}
		\begin{block}{DCS系统概念}
			\begin{enumerate}[]
				\item  过程控制级:操作设定、计算输出
				
				\item  过程监控级:采集数据、计算显示
				
				\item  多级计算机系统:客户端、服务器、控制器

			\end{enumerate}
		\end{block}
\pause
\begin{block}{DCS系统思想}
			\begin{enumerate}
				\item  分散控制:现场控制分散化、主次分明合理分布、故障风险影响最小化
				
				\item  集中操作:监视管理集中化、现场工况集中控制
				
				\item  分级管理:系统架构层次化、模块化、分工明确
			\end{enumerate}
		\end{block}
	\end{frame}
\begin{frame}{800XA系统概念与架构}{800XA系统构架介绍}
\begin{block}{上位机:规划控制、决策层}
服务器运行软件,提供了系统功能,客户端运行软件,为用户提供了各种形式的互动。
		\end{block}
		\begin{exampleblock}{下位机:执行任务、执行层}
			控制器运行逻辑,输出计算结果,I/O模件采集/输出数据,为现场设备提供接口。
		\end{exampleblock}
		\begin{alertblock}{\heiti 连接枢纽!}
			互连服务器(CS)经过交换机与控制器交换数据
		\end{alertblock}
		
	\end{frame}
\begin{frame}{800XA系统概念与架构}{800XA系统构架讨论:节点和网络}
		\begin{block}{节点:系统中每一个控制器为一个节点}
			\begin{enumerate}
				\item  属性服务器(AS):提供了属性目录和服务器和对象的管理、名字、安全等相关。
				
				\item  互接服务器(CS):提供对控制器和其他数据源的访问
				
				\item  工作站(ENG/OPR):为用户提供监视、操作、组态功能
				
				\item  控制器(DPU):根据CS命令,采集设备信息,通过预定义逻辑计算输出结果
			\end{enumerate}
		\end{block}
\end{frame}
\subsection{我厂DEH系统配置}
\begin{frame}{800xA系统架构实例}{单网段系统:脱硫}
		\begin{figure}[htbp]
 \centering
\includegraphics[width=290pt,keepaspectratio]{dan.png}
\caption{单网段系统分布图例}
\label{fig:myphoto}
\end{figure}
	\end{frame}
\begin{frame}{800xA系统架构实例}{多网段系统:主机}
		\begin{figure}[htbp]
 \centering
\includegraphics[width=290pt,keepaspectratio]{da.png}
\caption{多网段系统分布图例}
\label{fig:myphoto}
\end{figure}
	\end{frame}
\begin{frame}{800xA系统架构实例}{我厂主机DCS系统架构}
		\begin{block}{主机DCS系统共分为5个网段}
			\begin{enumerate}
				\item  1号网段:1、2号炉及其附属设备的控制
				
				\item  2号网段:3、4号炉炉及其附属设备的控制
				
				\item   3号网段:1、2、3号汽轮机及其附属设备的控制
				
				\item  4号网段:4、5号汽轮机及其附属设备的控制
\item   5号网段:公用管网、高加、除氧器、给水泵、化学加药及煤灰水处理等公用设备
			\end{enumerate}
		\end{block}
\begin{alertblock}{\heiti 两台锅炉在同一网段!}
			1、2号锅炉在1网段、3、4号锅炉在2网段\\  一个网段故障会影响两台锅炉!
		\end{alertblock}
	\end{frame}
\subsection{800XA系统节点故障影响及处理方法}
\begin{frame}{800XA系统节点故障影响及处理方法}{各节点故障影响}
\begin{block}{\heiti 单台服务器故障}
			每组服务器冗余配置,并且无扰切换,单台服务器故障不会影响机组正常运行,系统会有报警。
		\end{block}
\pause
\begin{exampleblock}{\heiti 两台服务器均故障!}
			\begin{enumerate}
				\item  AS服务器故障:失去对所有设备监视、操作功能
				
				\item   CS服务器故障:失去对该网段设备的监视、操作功能。
				
			\end{enumerate}
		\end{exampleblock}
\pause
\begin{alertblock}{\heiti 服务器故障时控制器不受任何影响!!!}
			设备运行状态、相关联锁和自动功能正常!
		\end{alertblock}
\end{frame}
\begin{frame}{800XA系统节点故障影响及处理方法}{节点故障该做什么}
\begin{block}{\heiti 单台服务器故障,操作员站画面短时间打叉}
			操作员站自连接至备用服务器,恢复后正常操作。
		\end{block}
\pause
\begin{exampleblock}{\heiti 两台服务器均故障,失去监视操作功能!}
			
				服务器恢复正常前安排专人到现场监护重要设备和参数,确认短时间无法恢复则打闸停机。
			
		\end{exampleblock}
\pause
\begin{alertblock}{\heiti 操作员站无法监视调整工况!}
			故障时工况稳定?故障时正在调整工况?
		\end{alertblock}
\end{frame}
\section{DEH系统设备讲解}
\subsection{系统转速}
\begin{frame}{汽轮机转速}{1号汽轮机转速采集}
\begin{block}{超速保护器转速采集}
			\begin{enumerate}
				\item 转速卡
				\item 继电器卡
				\item 模拟量输出卡
			\end{enumerate}
		\end{block}
\begin{block}{DEH系统转速采集}
			\begin{enumerate}
				\item PI开采集生成转速
				\item 系统转速
			\end{enumerate}
		\end{block}

\end{frame}
\begin{frame}{汽轮机转速}{4号汽轮机转速采集}
\begin{block}{TSI系统转速采集}
			\begin{enumerate}
				\item 6500转速卡
			\end{enumerate}
		\end{block}
\begin{block}{DEH系统转速采集}
			\begin{enumerate}
				\item 转速卡采集生成转速
				\item 系统转速生成
			\end{enumerate}
		\end{block}

\end{frame}
\subsection{高调门控制}
\begin{frame}{高调门控制}{1号汽轮机伺服阀开环控制}
\begin{block}{伺服阀信号控制}
			\begin{enumerate}
				\item 阀位指令:20-160mA指令
				\item  无阀位反馈,靠油缸动态平衡
			\end{enumerate}
		\end{block}
\pause
\begin{alertblock}{\heiti 开环控制!!!}
			无LVDT参与阀位控制 
		\end{alertblock}
\end{frame}
\begin{frame}{高调门控制}{3号汽轮机伺服阀}
\begin{block}{伺服阀信号控制}
			\begin{enumerate}
				\item 阀位指令:20-160mA指令
				\item  伺服卡
			\end{enumerate}
		\end{block}
\pause
\begin{alertblock}{\heiti 开环控制!!!}
			无LVDT参与阀位控制 
		\end{alertblock}
\end{frame}
\begin{frame}{高调门控制}{4号汽轮机伺服阀}
\begin{block}{伺服阀信号控制}
			\begin{enumerate}
				\item 阀位指令:20-160mA指令
				\item  伺服卡
			\end{enumerate}
		\end{block}
\pause
\begin{alertblock}{\heiti 开环控制!!!}
			无LVDT参与阀位控制 
		\end{alertblock}
\end{frame}
\section{DEH系统功能}
\subsection{超速保护}
\begin{frame}{超速保护}{1号汽轮机超速}
\begin{block}{\heiti 1号汽轮机DEH请求停机中电超速保护动作条件}
				主控超速110打闸:系统转速大于5975
		\end{block}
\begin{block}{\heiti 1号汽轮机ETS超速保护动作条件}
				超速保护器开关量信号三取二
		\end{block}
\begin{alertblock}{\heiti 超速保护器动作信号为故障安全型!!!}
			\begin{enumerate}
				\item 超速保护器开关量正常转速下是1,转速高于5975变为0
				\item 故障安全型信号优缺点
				\item 超速保护器转速监测
			\end{enumerate}
	\end{alertblock}
\end{frame}
\begin{frame}{超速保护}{3号汽轮机超速}
\begin{block}{\heiti 3号汽轮机DEH请求停机中超速保护动作条件}
				主控超速110打闸:系统转速大于5975
		\end{block}
\begin{block}{\heiti 3号汽轮机ETS超速保护动作条件}
				超速保护器开关量信号三取三且DEH转速任意一个转速大于4381
		\end{block}
\begin{alertblock}{\heiti ETS超速停机保护!!!}
			\begin{enumerate}
				\item 超速保护器开关量正常转速下是1,转速高于4381变为0
				\item 导致ETS超速停机条件如此的原因
			\end{enumerate}
	\end{alertblock}
\end{frame}
\begin{frame}{超速保护}{4号汽轮机超速}
\begin{block}{\heiti 4号汽轮机DEH请求停机中电超速保护动作条件}
			\begin{enumerate}
				\item  主控超速110打闸:系统转速大于3300
				\item  测速板超速110打闸:转速卡超速动作信号三取二
		\end{enumerate}
		\end{block}
\begin{block}{\heiti 4号汽轮机ETS超速保护动作条件}
				TSI系统转速卡动作信号三取二
		\end{block}
\end{frame}
\begin{frame}{超速保护}{1号汽轮机OPC超速保护}
\begin{block}{\heiti 1号汽轮机OPC保护动作条件}
			\begin{enumerate}
				\item 未并网前,系统转速大于
				\item 油开关跳闸:并网信号消失瞬间,功率大于7.5Kw,触发OPC动作
		\end{enumerate}
		\end{block}
\begin{block}{\heiti 1号汽轮机OPC保护复位条件}
			\begin{enumerate}
				\item 系统转速小于
				\item OPC触发两秒后复位
		\end{enumerate}
		\end{block}
\begin{block}{\heiti 1号汽轮机OPC保护动作设备}
			\begin{enumerate}
				\item 高调门全关
				\item 中调门全关
		\end{enumerate}
		\end{block}
\end{frame}
\begin{frame}{超速保护}{3号汽轮机OPC超速保护}
\begin{block}{\heiti 3号汽轮机OPC保护动作条件}
			\begin{enumerate}
				\item 未并网前,系统转速大于
				\item 油开关跳闸:并网信号消失瞬间,功率大于7.5Kw,触发OPC动作
		\end{enumerate}
		\end{block}
\begin{block}{\heiti 1号汽轮机OPC保护复位条件}
			\begin{enumerate}
				\item 系统转速小于
				\item OPC触发两秒后复位
		\end{enumerate}
		\end{block}
\begin{block}{\heiti 4号汽轮机OPC保护动作设备}
			\begin{enumerate}
				\item 高调门全关
		\end{enumerate}
		\end{block}
\end{frame}
\begin{frame}{超速保护}{4号汽轮机OPC超速保护}
\begin{block}{\heiti 4号汽轮机OPC保护动作条件}
			\begin{enumerate}
				\item 主控超速103
				\item 油开关跳闸
		\end{enumerate}
		\end{block}
\begin{block}{\heiti 4号汽轮机OPC保护复位条件}
			\begin{enumerate}
				\item 系统转速大于
				\item 并网信号消失瞬间,功率大于7.5Kw,触发OPC动作
		\end{enumerate}
		\end{block}
\begin{block}{\heiti 1号汽轮机OPC保护动作设备}
			\begin{enumerate}
				\item OPC电磁阀动作
				\item 中调门全关
				\item 高调门全关
		\end{enumerate}
		\end{block}
\end{frame}
\subsection{转速控制}
\begin{frame}{转速控制}{1号汽轮机转速控制}
\begin{block}{\heiti 1号汽轮机启动允许条件}
			\begin{enumerate}
				\item OOTTV和OOTV全开
				\item 控制油压力和润滑油压力正常
				\item ETS停机复位机组未并网静态试验未投入
		\end{enumerate}
		\end{block}
\begin{block}{\heiti 目标转速和升速率设置}
			\begin{enumerate}
				\item 临界转速升速率
		\end{enumerate}
		\end{block}
\begin{block}{\heiti 定速后并网设置}
			\begin{enumerate}
				\item 并网瞬间高调门开度增加3.5
		\end{enumerate}
		\end{block}
\end{frame}
\begin{frame}{转速控制}{3号汽轮机阀门切换}
\begin{block}{\heiti 3号汽轮机启动允许条件}
			\begin{enumerate}
				\item OOTTV和OOTV全开
				\item 控制油压力和润滑油压力正常
				\item ETS停机复位机组未并网静态试验未投入
		\end{enumerate}
		\end{block}
\begin{block}{\heiti 目标转速和升速率设置}
			\begin{enumerate}
				\item 临界转速升速率
		\end{enumerate}
		\end{block}
\begin{block}{\heiti 定速后并网设置}
			\begin{enumerate}
				\item 并网瞬间高调门开度增加3.5
		\end{enumerate}
		\end{block}
\end{frame}
\begin{frame}{转速控制}{3号汽轮机转速控制}
\begin{block}{\heiti 3号汽轮机启动允许条件}
			\begin{enumerate}
				\item OOTTV和OOTV全开
				\item 控制油压力和润滑油压力正常
				\item ETS停机复位机组未并网静态试验未投入
		\end{enumerate}
		\end{block}
\begin{block}{\heiti 目标转速和升速率设置}
			\begin{enumerate}
				\item 临界转速升速率
		\end{enumerate}
		\end{block}
\begin{block}{\heiti 定速后并网设置}
			\begin{enumerate}
				\item 并网瞬间高调门开度增加3.5
		\end{enumerate}
		\end{block}
\end{frame}
\subsection{在线试验}
\begin{frame}{我厂DCS系统架构实例}{重要辅机系统在控制器内分布情况}
\begin{block}{\heiti 分散布置的系统}
			\begin{enumerate}

				\item  六大风机、空预器、制粉系统分为两组分别分布在两组控制器
				
				\item   4台电动给水泵系统分为两组分别分布在两组控制器
		\end{enumerate}
		\end{block}
\pause
\begin{alertblock}{\heiti 存在问题}
			\begin{enumerate}
				\item  设备分散不彻底,锅炉侧调门、变频器在同一台控制器
				
				\item  主设备停运后触发MFT信号集中在同一台控制器
		\end{enumerate}
		\end{alertblock}
\end{frame}

\begin{frame}{我厂DCS系统架构实例}{重要辅机系统在控制器内分布情况}
\begin{exampleblock}{\heiti 分布在一台控制器}
			\begin{enumerate}
				\item  六大风机入口调门和变频器、4台制粉系统冷热风调门和给煤机变频器、主/辅给水调门
				
				\item  3组高加系统、4组除氧器系统
				
				\item   1、2、3号汽轮机凝结水泵、空冷风机

				\item   4、5号汽轮机控制油泵、润滑油泵、凝结水泵、空冷风机
		\end{enumerate}
		\end{exampleblock}
\pause
\begin{alertblock}{\heiti 风险}
			\begin{enumerate}
				\item  一组控制器故障会影响所有设备!
\item  如果必须执行初始化下装,所有设备会恢复初始状态!
		\end{enumerate}
		\end{alertblock}
\end{frame}
\subsection{重要测点在I/O模件内分布情况}

\begin{frame}{我厂DCS系统架构实例}{重要测点在I/O模件内分布情况}
\begin{alertblock}{\heiti 保护测点分布在同一块卡件}
			\begin{enumerate}
				\item  炉膛压力高高、炉膛压力低低测点
			
		\end{enumerate}
		\end{alertblock}
\pause
\begin{alertblock}{\heiti 调门或变频器指令分布在同一块卡件}
			\begin{enumerate}
				\item  管网双减调节阀指令
				\item  除氧器压力、液位、高压加热器液位调节阀指令
				\item 六大风机入口调门、磨煤机冷热风调门、给煤机指令
			
		\end{enumerate}
		\end{alertblock}
\begin{alertblock}{\heiti DO通道配置有长电平指令设备}
			\begin{enumerate}
				\item  六大风机、空预器、制粉系统停运触发MFT信号
			
		\end{enumerate}
		\end{alertblock}
\end{frame}
\begin{frame}

\begin{table}
	\caption{800XA故障试验结果}
	\begin{tabular}{|c|c|c|c|c|}
		\hline
		       操作    & AI         & DI     & AO&DO  \\   \hline
		卡件故障    & 保持        & 保持      & 保持&保持  \\  \hline 
		更换卡件    & 保持    & 保持   & 复位&复位\\  \hline  
		通讯电缆故障    & 保持        & 保持      & 保持& 保持 \\   \hline  
		站头故障    &保持          & 保持       & 保持& 保持 \\     \hline
CI854故障    &保持          & 保持       & 保持& 保持 \\     \hline
控制器故障    &保持          & 保持       & 保持& 保持 \\     \hline
初始化下装    &复位          & 复位       & 复位& 复位 \\     \hline
	\end{tabular}
	\end{table}
注:站头CI840同时拔出后下属卡件值都会复位


\end{frame}
%\section{DCS典型案例}
\begin{frame}{DCS典型案例}{C3控制器故障处理导致1号锅炉停运}
\begin{block}{\heiti 控制器初始化下装导致锅炉灭火}
			CEX电缆松动导致两台控制器故障且控制器内程序丢失,需要初始化下装逻辑后才能正常运行,将相关设备切至就地后进行初始化下装,下装过程中3台风机变频器跳闸触发炉膛压力低低保护。
		\end{block}
\pause
\begin{alertblock}{\heiti DO电平指令初始化下装过程中复位为0}
			
				检查为风机变频器有软急停电平指令接至DCS DO通道,指令变为0时变频器急停。
			
		
		\end{alertblock}
\end{frame}
\begin{frame}{DCS典型案例}{净化装置控制器故障处理导致全厂停运}
\begin{block}{\heiti 控制器初始化下装导致停车}
			两台控制器故障且控制器内程序丢失,需要初始化下装逻辑后才能正常运行,下装过程调门关闭导致停车
		\end{block}
\pause
\begin{alertblock}{\heiti AO指令初始化下装过程中复位为0}
			
				调节阀指令恢复为0导致调节阀关闭,系统停车
			
		
		\end{alertblock}
\end{frame}
\begin{frame}{DCS典型案例}{所有上位机失电导致工况无法监控调整30分钟}
\begin{block}{\heiti 所有上位机失电导致工况无法监控调整}
			因电气误停UPS电源导致DCS系统公用总电源失电后,所有上位机失电,操作员站、工程师站、各服务器失电导致工况无法监控调整,陆续恢复电源并恢复服务器和操作员站后工况稳定无影响。
		\end{block}
\pause
\begin{alertblock}{\heiti 怎么做???风险!!!
}
			
				紧急停运,停运措施是否完整,风险?\\继续运行,工况是否平稳,风险?
			
		
		\end{alertblock}
\end{frame}

\section{存在的问题与做什么}
\begin{frame}{存在的问题与做什么}{存在的问题}
\begin{alertblock}{\heiti 目前存在问题!!!}
			\begin{enumerate}
				\item  CEX电缆松动会导致一组控制器内程序清空
\item  重要调门集中在同一组控制器上
\item  重要调门AO指令在同一块卡件上未分散
				\item  重要辅机对应控制器故障后具体的处理细节
\item  服务器、工作站备份恢复无法实现
				
				\item  800XA系统各状态有效的监视手段

				\item  网络变量、全局变量的合理使用

				
\item  AO卡件故障后现场对应启动调门应该采取什么措施
		\end{enumerate}
		\end{alertblock}
\end{frame}
\begin{frame}{存在的问题与做什么}{做什么}
\begin{alertblock}{\heiti 下一步做什么???}
			\begin{enumerate}
				\item  合理布置AO调门指令分布
				\item   对现场设备AO、DO指令情况进行具体排查分类确认故障及处理过程中应对措施
\item   DCS系统指令复位时各专业如何保证现场设备保持原状态
				\item   实现备份、恢复服务器
				\item   利用现有备件搭建一套单节点系统


		\end{enumerate}
		\end{alertblock}
\end{frame}


	
	

	
\begin{frame}{排查内容}{排查清单}  
 \begin{thebibliography}{99}

\bibitem{AO} AO指令: \emph{ABB系统AO点排查},
热控专业, 2023-11-16
\bibitem{DO} DO指令: \emph{ABB系统DO点排查},
热控专业, 2023-11-16
\bibitem{AO} 气动调门: \emph{AO供电调节阀},
热控专业, 2023-11-16
\bibitem{AO} 变频器: \emph{变频器远方就地切换状态},
电气专业, 2023-11-16
\end{thebibliography}
\end{frame}
\end{document}
