\ifx \allfiles \undefined
\documentclass{ctexart}
\usepackage[landscape]{geometry}
\usepackage{tikz,units}
\usepackage{circuitikz}
\usetikzlibrary{backgrounds,circuits.ee.IEC.relay}
\begin{document}
\else
\chapter{脱硫包装机夹带装置控制回路}
\fi
	\begin{center}
{\huge 脱硫包装机夹带装置控制回路}\\
\begin{tikzpicture}[scale=1.8,circuit ee IEC relay,thick]
		\draw (-2,6) node[left]{$\unit[24]{VDC}$} -- (4,6);
		\draw (-2,0) node[left]{$\unit[0]{VDC}$} -- (4,0);
		\draw (0,0) node[contact]{}
			to [make contact={info=$KA$,,term=8,term'=12}] ++(0,2)
			to[relay coil={info=$KA$,term=13,term'=14}] ++(0,2)
			to [break contact={push button={info=$SB$},term''=松开}] ++(0,2)
			node[contact]{};

		\draw (4,0) to[relay coil={info=Y}] ++(0,2)
			to [make contact={info=$KA$}] ++(0,2) -- ++(0,2);

		\draw (2,2)node[npn](Q){Q};

%\draw (Q.B) to [short,short] (OUT);
\draw (Q.E) to [short,short] (2,0) node[below]{蓝};
	\draw (2,6) node[above]{棕} to [short,short] (Q.C);
\draw (0,2) node[left]{黑} to (Q.B);
	\end{tikzpicture}
\\注:Q为NPN型三线制接近开关(检测到时输出端为0V,未检测到时输出端为24VDC),Y为单电控电磁阀线圈\\
拨动拨片,接近开关Q动作夹带装置夹紧,按下按钮SB夹带装置松开\\
\end{center}
\ifx \allfiles \undefined
\end{document}
\fi