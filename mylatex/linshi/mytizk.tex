\documentclass{ctexart}
\usepackage[active,tightpage,xetex]{preview}
\usepackage{tikz}
\usepackage{hyperref}
 \usetikzlibrary{fadings}
\newcommand{\Fin}{node[xshift=-1.5ex,rotate=10]{F}
node[rotate=170]{i}
node[xshift=1.5ex,rotate=45]{n}}
\def\arete{3} \def\epaisseur{5} \def\rayon{2}
\newcommand{\ruban}{(0,0)
++(0:0.57735*\arete-0.57735*\epaisseur+2*\rayon)
++(-30:\epaisseur-1.73205*\rayon)
arc (60:0:\rayon) -- ++(90:\epaisseur)
arc (0:60:\rayon) -- ++(150:\arete)
arc (60:120:\rayon) -- ++(210:\epaisseur)
arc (120:60:\rayon) -- cycle}
\begin{document}
dasfasdf
\label{test}

你好大额
\hyperref{test}[连接]
\begin{preview}
\begin{tikzpicture}[scale=10,transform shape]
	\hyperref[test]{\node[rectangle,rounded corners,draw=gray] (B) at(2,0) {B node}};
\end{tikzpicture}
\end{preview}
\begin{preview}
\begin{tikzpicture}[very thick,top color=white,bottom color=gray]
\shadedraw \ruban;
\shadedraw [rotate=120] \ruban;
\shadedraw [rotate=-120] \ruban;
\draw (-60:4) node[scale=5,rotate=30]{Ti{\color{orange}\textit{k}}Z};
\draw (180:4) node[scale=3,rotate=-90]{l’impatient};
\clip (0,-6) rectangle (6,6); % pour croiser
\shadedraw \ruban;
\draw (60:4) node [gray,xscale=-3,yscale=3,rotate=30]{pour};
\end{tikzpicture}
\end{preview}
\end{document}
