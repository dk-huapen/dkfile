%!TEX program = xelatex
\documentclass[a4paper,zihao=-4,linespread=1]{ctexrep}
\renewcommand{\CTEXthechapter}{\thechapter}%目录显示章节编号样式命令
% 最小行间间距设定
\setlength{\lineskiplimit}{3pt}
\setlength{\lineskip}{3pt}
% 下划线宏包
\usepackage[normalem]{ulem}
%LaTeX符号宏包
\usepackage{hologo}
\newcommand{\xelatex}{\Hologo{XeLaTeX}}%输出XeLaTeX标志
%颜色
\usepackage[table]{xcolor}
%奇怪的小定义
\newcommand{\RED}[1]{{\color{cyan}{#1}}}
\newcommand{\co}[1]{{\bfseries{#1}}}
% 编号列表宏包,并自定义了三个列表
\usepackage[inline]{enumitem}
        \setlist[enumerate]{font=\bfseries,itemsep=0pt}
        \setlist[itemize]{font=\bfseries,leftmargin=\parindent}
        \setlist[description]{font=\bfseries\uline,labelindent=\parindent,itemsep=0pt,parsep=0pt,topsep=0pt,partopsep=0pt}
\newenvironment{fead}{
    \begin{description}[font=\bfseries\uline,labelindent=\parindent,itemsep=0pt,parsep=0pt,topsep=0pt,partopsep=0pt]}
        {\end{description}}
%带宽度的
\newenvironment{para}{
        \begin{description}[font=\bfseries\ttfamily,itemsep=0pt,parsep=0pt,topsep=0pt,partopsep=0pt]}
        {\end{description}}
	%下边的定义猜测是缩小行间距
\newenvironment{feai}{/
        \begin{itemize}[font=\bfseries,itemsep=0pt,parsep=0pt,topsep=0pt,partopsep=0pt]}
        {\end{itemize}}
	\newenvironment{feae}{
        \begin{enumerate}[font=\bfseries,labelindent=0pt,itemsep=0pt,parsep=0pt,topsep=0pt,partopsep=0pt]}
	{\end{enumerate}}
%目录和章节样式
\usepackage{titlesec}
\usepackage{titletoc}
\titlecontents{chapter}[1.5em]{}{\contentslabel{1.5em}}{\hspace*{-2em}}{\hfill\contentspage}
\titlecontents{section}[3.3em]{}{\contentslabel{1.8em}}
        {\hspace*{-2.3em}}{\titlerule*[8pt]{$\cdot$}\contentspage}
\titlecontents*{subsection}[2.5em]{\small}{\thecontentslabel{} }{}{, \thecontentspage}[;\qquad][.]
% 章节样式
\setcounter{secnumdepth}{3} % 自动编号一直到subsubsection
\newcommand{\chaformat}[1]{%
        \parbox[b]{.5\textwidth}{\hfill\bfseries #1}%
        \quad\rule[-12pt]{2pt}{70pt}\quad
        {\fontsize{60}{60}\selectfont\thechapter}}
\titleformat{\chapter}[block]{\hfill\LARGE\sffamily}{}{0pt}{\chaformat}[\vspace{2.5pc}\normalsize
        \startcontents\setlength{\lineskiplimit}{0pt}\printcontents{}{1}{\setcounter{tocdepth}{2}\songti}]
\titleformat*{\section}{\centering\Large\bfseries}
%\titleformat{\subsubsection}[hang]{\bfseries\large}{\rule{1.5ex}{1.5ex}}{0.5em}{}%有它的时候subsubsection显示方块,所以删除了它
% 图表
\usepackage{array,multirow,makecell}
  \setlength\extrarowheight{2pt} % 行高增加
\usepackage{diagbox}
\usepackage{longtable}
\usepackage{graphicx,wrapfig}
  \graphicspath{{./tikz/}}
\usepackage{animate}
\usepackage{caption,subcaption}
  \captionsetup[sub]{labelformat=simple}
  \renewcommand{\thesubtable}{(\alph{subtable})}
% 三线表
\usepackage{booktabs}
%\renewcoemand{\contentsname}{目录}
% 代码环境
\usepackage{listings}%代码高亮显示
% 复制代码时不复制行号
\usepackage{accsupp}
  \newcommand{\emptyaccsupp}[1]{\BeginAccSupp{ActualText={}}#1\EndAccSupp{}}
\usepackage{tcolorbox}
  \tcbuselibrary{listings,skins,breakable,xparse}

% global style
\lstset{
  basicstyle=\small\ttfamily,
  % Word styles
  keywordstyle=\color{blue},
  commentstyle=\color{green!50!black},
  columns=fullflexible,  % Avoid too sparse word spaces
  keepspaces=true,
  % Numbering
  numbers=left,
  numberstyle=\tiny\color{red!75!black}\emptyaccsupp,
  % Lines and Skips
  aboveskip=0pt plus 6pt,
  belowskip=0pt plus 6pt,
  breaklines=true,
  breakatwhitespace=true,
  emptylines=1,  % Avoid >1 consecutive empty lines
  escapeinside=``
}

% 对于 tcolorbox 中 listings 库的 ''tcblatex'' style 的重现,
% 添加了新的关键词
\lstdefinestyle{latexcn}{
  language=[LaTeX]TeX,
  % More Keywords
  classoffset=0,
  texcsstyle=*\color{blue},
  moretexcs={
    % LaTeX extension
    chapter,section,subsection,setlength,
    thechapter,thesection,thesubsection,theequation,
    chaptermark,chaptername,appendix,
    bibname,refname,bibpreamble,bibfont,citenumfont,bibnumfmt,bibsep,
  },
  classoffset=1,
  texcsstyle=*\color{orange!75!black},
  moretexcs={
    % XeCJK & CTeX
    xeCJKsetup,setCJKmainfont,newCJKfontfamily,CJKfontspec,
    CTEXthechapter,songti,heiti,fangsong,kaishu,yahei,lishu,youyuan,
    % AMSmath / AMSsymb / AMSthm
    middle,text,tag,boldsymbol,mathbb,dddot,ddddot,iint,varoiint,
    dfrac,tfrac,cfrac,leftroot,uproot,underbracket,xleftarrow,xrightarrow,
    overset,underset,sideset,mathring,leqslant,geqslant,because,therefore,
    shortintertext,binom,dbinom,implies,thesubequation,
    impliedby,genfrac,theoremstyle,qedhere,
    % Other math packages
    wideparen,intertext,
    xlongequal,xLeftrightarrow,xleftrightarrow,xLongleftarrow,xLongrightarrow,
    % xcolor
    definecolor,color,textcolor,colorbox,fcolorbox,
    % hyperref
    hyperref,autoref,href,url,nolinkurl,
    % Graph & Table
    includegraphics,graphicspath,scalebox,rotatebox,animategraphics,
    newcolumntype,arraybackslash,multirow,captionsetup,
    thead,multirowcell,makecell,Xhline,Xcline,diagbox,
    toprule,midrule,bottomrule,DeclareFloatingEnvironment,
    % ulem
    uline,uuline,dashuline,dotuline,uwave,sout,xout,
    % fancyhdr
    lhead,chead,rhead,lfoot,cfoot,rfoot,
    fancyhf,fancyhead,fancyfoot,fancypagestyle,
    % fontspec
    newfontfamily,
    % titlesec & titletoc
    titlelabel,titleformat,titlespacing,titleline,titlerule,dottecontents,titlecontents,
    % enumitem
    setlist,
    % Listings & tcolorbox
    lstdefinelanguage,lstdefinestyle,lstset,lstnewenvironment,
    tcbuselibrary,newtcblisting,newtcbox,DeclareTCBListing
    % citation & index: natbib, imakeidx
    setcitestyle,printindex,
    % Other packages
    hologo,lettrine,endfirsthead,endhead,endlastfoot,columncolor,rowcolors,modulolinenumbers,MakeShortVerb,tikz,Hologo
  }
}
%定义shell命令格式
\lstdefinestyle{shell}{
  language=bash,
  % More Keywords
  classoffset=0,
  texcsstyle=*\color{blue},%shell命令颜色
  morekeywords={ssh,screen,scp}%增加shell命令关键词
}

% cmd & envi

\newtcbox{\latexline}[1][green]{on line,before upper=\ttfamily\char`\\,
  arc=0pt,outer arc=0pt,colback=#1!10!white,colframe=#1!50!black,
  boxsep=0pt,left=1pt,right=1pt,top=1pt,bottom=1pt,
  boxrule=0pt,bottomrule=1pt,toprule=1pt}
%pkg
\newtcbox{\pkg}[1][orange!70!red]{on line,before upper={\rule[-0.2ex]{0pt}{1ex}\ttfamily},
  arc=0.8ex,colback=#1!30!white,colframe=#1!50!black,
  boxsep=0pt,left=1.5pt,right=1.5pt,top=1pt,bottom=1pt,
  boxrule=1pt}

% tcblisting definitions
\newtcblisting{latex}{breakable,skin=bicolor,colback=gray!30!white,
  colbacklower=white,colframe=cyan!75!black,listing only, 
  left=6mm,top=2pt,bottom=2pt,fontupper=\small,
  listing options={style=latexcn}
}
%声明shell环境
\newtcblisting{shell}{breakable,skin=bicolor,colback=gray!30!white,
  colbacklower=white,colframe=cyan!75!black,listing only, 
  left=6mm,top=2pt,bottom=2pt,fontupper=\small,
  listing options={style=shell}
}
\NewTCBListing{codeshow}{ !O{listing side text} }{
  skin=bicolor,colback=gray!30!white,
  colbacklower=pink!50!yellow,colframe=cyan!75!black,
  valign lower=center,
  left=6mm,righthand width=0.4\linewidth,fontupper=\small,
  % listing style
  listing options={style=latexcn},#1,
}
%索引与参考文献
\usepackage{makeidx}%调用索引宏包
\bibliographystyle{plain}
\renewcommand{\bibname}{参考文献}
\usepackage[numindex,numbib]{tocbibind}
\usepackage[square,super,sort&compress]{natbib}
%引用
\usepackage{hyperref}
%目录,标签变成红色
\hypersetup{colorlinks, bookmarksopen = true, bookmarksnumbered = true, pdftitle=LaTeX-cn, pdfauthor=K.L Wu, pdfstartview=FitH}

\title{简单粗暴\LaTeX\ }
\author{dk-huapen\\
{本手册是自己学习laTeX的排版材料}
}
\date{最后更新于:\today}
\makeindex%开启索引的收集

\begin{document}
\maketitle
\setlength{\lineskiplimit}{0pt}
\tableofcontents
\setlength{\lineskiplimit}{3pt}
\chapter{序}
%\noindent{第一稿序}
%\begin{enumerate}[font=\bfseries,labelindent=0pt,itemsep=0pt,parsep=0pt,topsep=0pt,partopsep=0pt]
为啥会有这个玩意儿呢(暂时还不能称它为一本书),就是因为爱折腾,这么能折腾不记录下是不是会有点遗憾呢?最开始这些记录是在我自己的那个网站里边记录的,但是使用的时候感觉不太方便也不太美观,直到遇见了它,它很对我的脾气,有点爱不释手,最后决定边学习边记录,说不定最后还真能写出点东西来,期待吧...

现在的书大部分是系统的专业的讲解一些东西,我个人感觉在现在这个信息化发达的环境中,学习资料是不缺乏的,反而因为学习资料太多对我这种泥腿子造成的困惑更大,我更需要的是一个方向和一个一个有效的小目标,可事实确实要么在原地打转重复的学着相同的内容,要么步子跨的太大学者不感兴趣莫名其妙的东西直到放弃。id其实在之前我是有一稿手册的,开始撰写的日期大概在2015年4月,但是自己觉得写的太烂,因此索性推倒重写了这一版.这一版的主要特征是:
\begin{feae}
	\item 我希望能够吸引初学者快速上手,解决手头的问题.因此去掉了枯燥的讲解和无穷无尽的宏包用法介绍,直接使用实例;
	\item 力求突出实用性.当然,也会提点一些可以深入学习的内容,读者可以自行查阅,或者阅读本手册中的扩展阅读章节(即带星号*的章节).
\	\item 本手册使用的编辑器为\TeX\ Studio,而非之前的商业软件WinEdt. 这使得学习\LaTeX\ 的门槛更低.当然了,你有权使用任何编辑器.
\end{feae}
		\index{手册}主体分为六大部分\cite{LaTeX-cn,LHY2013latex}:
\begin{fead}
	\item[写给读者*]
	\item[基础]
	\item[数学排版]
	\item[进阶]
	\item[绘图*]
	\item[附录]
\end{fead}
由于工作全部由我一人完成,限于视野,难免存在错漏之处。恳请读者指正。如遇到的手册中无法解决的问题,欢迎向我提出。推荐书目可参考本手册附录。
最后,还要感谢在\LaTeX\ 学习中为我解答疑惑的同学,特别是来自\LaTeX\ 度吧的吧友;本手册中许多的解决方案都是由他们提供的。我谨在此记录。

\vfill

Mail:wklchris@botmail.com

Chris Wu

September 17,2016于Davis,CA
\clearpage%开始新的一页
更新日志:
版本号以
更早的版本的更新细节,请到。。浏览。


%Main Contents

		%!TEX root = ../dk_log.tex
		\chapter{\mbox{写给自己和有缘人}*}
		
		\section{写给自己的话}
感觉好多时候都在原地打转,重复做着一些事情,如果从这里开始可以记录下来作为查阅资料的话应该会好很多吧,那就从它开始吧。	
		\section{\TeX\ 与\LaTeX\ 的优缺点}
		\TeX\ 的优点:
		\section{为什么需要\LaTeX\ ?}
		你可能基于以下原因学习
		\section{MS Word难道不优秀吗?}
		我想说的是
		\section{\LaTeX\ 生成的文件格式?}
		一般广为使用的是pdf

		%!TEX root = ../dk_log.tex
\chapter{操作系统}
本章主要记录我在使用各款操作系统过程中值得记录的一些东西,包括安装各款操作系统时遇到的问题以及使用各款操作系统的体验,最重要的是记录一些常用的、零碎的命令或知识点方便以后查阅。
\section{操作系统选择}
操作系统还用选择?买上电脑用就完了,有啥好选择的。在我初次接触linux系统前真的没有想过这个问题,记得那是大专一年级2018年的时候,学习C语言课程后开始发现了自己有点喜欢编成,然后开始学习MFC写一些Windows下的程序,那会儿用的还是Vis\\
操作系统就是计算机上的工作平台,目前主流的就是Windows、macOS、Linux,Linux系统呢又有很多版本,比如Unbont、CentOS、Debian。
\subsection{计算机磁盘规划}
这里主要记录下计算机硬盘的选择和配置要求,个人计算机呢用的时间长了会卡经常会重装系统,针对这个怎样配置硬盘最方便呢?服务器要求是稳定运行,所以硬盘会采用磁盘阵列做冗余配置,那应该选择哪一种呢?
\subsubsection{个人计算机磁盘规划}
个人电脑的话我感觉使用两块硬盘,一块SSD固态硬盘用作安装操作系统和常用软件方便整套系统备份恢复,一块使用HDD机械硬盘主要用来存放数据文件方便数据备份和迁移。下面记录下关于MBR和GPT分区以及Logic和UEFI引导的记录。

MBR和GPT最明显的区被就两点:
\begin{enumerate}
	\item GPT是在计算机发展过程中发现MBR不够用才出来的一种分区方式
	\item MBR最大只支持2TB硬盘,如果硬盘大于2TB只能用GPT
	\item MBR支持分区数量有限,GPT理论上无限个可以
	\item MBR支持Logic引导系统,GPT支持UEFI引导系统
	\item MBR启动分区需要标记,GPT启动分区默认C盘
	\item win7及以前系统用MBR,win10及以后系统用GPT
\end{enumerate}
这个是在一次安装win10电脑上安装win7过程中遇到情况,安装完win7后启动不了找不到系统盘最后猜测是原来年win10是GPT分区的UEFI引导启动,后来我把win7装到GPT分区后仍热无法引导启动查资料说win7本来不支持UEFI引导,需要做一些工作才能实现UEFI引导,这个后续有机会的话试一下。
\subsubsection{服务器磁盘规划}
工业用电脑或服务器的话还是配置RAID 1比较实用,毕竟备份恢复起来太方便了。这里我简单记录下我接触过的两种型号服务器\index{RAID}配置方法,以及RAID故障恢复的方法。

首先是DCS用的HP Gen8服务器,磁盘阵列控制器型号是HP Smart Array P420i,硬盘是刀片式插槽,现场配置的是RAID 1,在做硬盘备份恢复过程中用到的。

磁盘阵列配置
\begin{enumerate}
	\item 将硬盘插入插槽,开机,看服务器硬件情况大概得5分钟以上
	\item 当自检界面出现Press <F5> to run the Option ROM Configuration for Arrays Utility时按下<F8>进入磁盘阵列配置界面
	\item 选择View Logical Drive选项查看当前阵列配置情况,如果是RAID 1,就会显示状态是OK/RECOVERY,还会显示实际硬盘安装情况OK/MISSING
	\item 选择Delete Logical Drive可以删除当前阵列配置
	\item 选择Create Logical Drive创建阵列,创建RAID 0的话就直接选择单硬盘就可以,如果是RAID 1的话就可以选择两块甚至更多硬盘组成镜像盘
	\item Select Boot Volume是设置系统启动硬盘,服务器里启动选项可以选择磁盘阵列卡
\end{enumerate}
RAID 1磁盘阵列备份硬盘
\begin{enumerate}
	\item 服务器配置了RAID 1模式并指定了冗余硬盘盘位为前提,下边提到的插拔硬盘都是在RAID 1配置盘位上进行的
	\item 操作系统正常运行过程一组硬盘中硬盘的禁止弹出指示灯不亮说明冗余正常拔出任意一块硬盘都不会影响当前系统正常运行,拔出其中一块硬盘后剩下的一块硬盘禁止弹出指示灯会点亮说明此时该硬盘拔出会影响正常运行
	\item 插入一块空硬盘,重点是空硬盘,清除完RAID标记后的没有建分区表的空硬盘,此时新插入硬盘灯亮开始同步数据,这时原硬盘禁止弹出指示灯仍然点亮,说明备份没有完成RAID 1没有建立冗余
	\item 视硬盘数据大小同步时间会有差别,大概1小时左右,同步完成后硬盘的禁止弹出指示灯都会熄灭,此时备份完成可以任意拔出其中一块硬盘作为备份盘或RAID 1冗余运行
\end{enumerate}
还使用过一种惠普塔式工作站,主板上磁盘阵列卡是定制的默认配置RAID 1,直接就可以硬盘同步备份。

另一种服务器是搭建SIS系统时用的戴尔服务器,这个使用过程中给我映像比较深的是更换备份硬盘后需要选择配置启动硬盘,以后再用到的时候记录下。



\subsection{操作系统安装}
最开始安装XP系统时常用的还是CD,后来到了Windows7以后用的U盘安装,硬盘安装比较多一点,安装Linux系统的时候用的U盘安装比较多一点,后来也用过U盘引导硬盘安装的方法,下面就简单记录下感受吧。
现在ISO安装光盘都很大了尤其是Linux的超过4.7GB了,而且现在电脑光驱都不用了尤其是个人电脑,所以CD和DVD安装就不多说了。
U盘安装是把ISO文件用U盘启动制作工具刻录在U盘里,但是废U盘啊。还有一种就是把U盘制作成PE文件引导安装Windows系统,这种的广告插件是真的多啊,这种应该只适用于Windowns安装。使用U盘引导硬盘安装Linux系统还可以就是制作U盘引导有点费劲,但是上边的方法需要根据要安装系统的情况和需求进行选择,做成各种安装介质备起来或用的时候临时做,我之前就是这么作的。直到遇见它中级杀人武器积各种技能与一身的Ventoy,用它制作好U盘后只需要把你的所有ISO镜像文件拷贝放到U盘里,U盘启动后选择你要用的ISO文件就可以了。
\subsection{Ventoy}

用官网的话来说\href{https://www.ventoy.net/cn/index.html}{Ventoy}是一个制作可启动U盘的开源工具,你只需要把ISO等类型文件直接拷贝到U盘里面就可以启动了,无需其他操作,可以一次性拷贝很多不同的镜像,它会形成一个菜单供用户选择,一个U盘可以同时支持x86Legacy BIOS、UEFI模式。通俗点讲就是系统镜像拷贝进取就可以使用,只要U盘够大可以把想安装的操作系统镜像都放到U盘里,随时可以安装任何一款操作系统。这里记录下U盘安装Ventoy和硬盘安装Ventoy方法。
\subsubsection{U盘安装Ventoy}
安装的过程没有什么复杂的现在新版本是图形化操作,选中U盘安装就可以了,值得一提的是U盘里边应该放哪些系统ISO,全方?太大!不全方?用的时候又不方便!我使用过程中感觉有以下这么几种选择:
\begin{enumerate}
	\item 用小点U盘专门安装Ventoy工具,再用移动硬盘存储系统镜像用哪个选择哪个
	\item 用U盘专门安装Ventoy工具,再放入Linux Live系统和winPE系统用来处理问题,需要安装什么系统再找对应ISO文件
\end{enumerate}

\subsubsection{移动硬盘安装Ventoy}

\subsection{WinPE}
网上有很多选择比如老毛桃、大白菜、雨沐林枫等,我\href{https://lmt.psydrj.com/index.html}{老毛桃}用的比较多就以它为例吧,在Windwos系统下安装、恢复用。用官网的话来说Laomaotao-winPE是一款系统预安装环境(PE)支持BIOS(Legacy)与UEFI两种启动模式。我感觉它比较实用的是PE系统里边的一些工具比如磁盘分区、系统密码破解,最好用的应该是Windows系统无法启动时拷贝数据和GHOST系统备份和恢复。这里主要记录下使用WinPE盘备份和恢复系统。
\subsubsection{WinPE备份和恢复系统}
ABB DCS系统操作员站OP53恢复过程记录
\begin{enumerate}
	\item 首先准备WinPE系统盘、惠普Z230主机箱要求C盘符80GB、op16920.GHO备份镜像
	\item 进入WinPE系统后将op16920.GHO文件拷贝至D盘,启动Ghost工具
	\item 在Ghost软件中选择恢复系统选项,选择op16920.GHO文件,选择目标分区C盘符,开始恢复...等待10分钟左右,恢复完成
	\item 关机重启进入新恢复的系统,默认登录用户是OP16,切换为管理员修改IP地址为172.16.48.153和172.17.48.153
	\item 计算机名称为OP53
	\item 将计算机接入DCS系统网络后,登录管理员确认网络正常,添加或修改域名为LuanPP.Local,重启计算机后LUANPP.operator用户登录,查看站点状态是否正常。
	\item 修改域名需要计算机正常接入DCS系统户对应语

\end{enumerate}

打开\TeX\ Studio后,选择选项$\rightarrow$设置\TeX\ Studio $\rightarrow$ 构建 $\rightarrow$ 默认编译器,选择\xelatex{}。这主要是基于中文文档编译的考虑,同时\xelatex 也能很好的编译英文文档。我建议始终使用它作为默认编译器。

\subsection{Debian Live}
这个是一个可以装到U盘里的Debian系统,很方便的,我目前只使用它清空过\index{RAID标记}硬盘数据。这里简单记录下
我们厂使用的是ABB DCS系统,它属于C/S架构,光服务器就有16台,后来准备搭建一套ABB DCS系统服务器来做培训和测试用,但是ABB这个授权很贵,而且厂家费用很高,就计划自己试着搭建以下,后来一看安装软件就20多GB,而且特别多,估计安装难度很大,现场服务器硬盘配置是RAID 1冗余,所以就选择了硬盘备份这种方式,从外边采购相型号配置的旧服务器,旧的服务器硬盘都有数据,如果用RAID 1同步的话就需要清空旧硬盘中的数据,关键是RAID标识,我在网上查了下清空RAID标识方法总结起来有三种:1.linux下用dd命令擦除硬盘最后RAID数据;2.linux下用命令删除RAID标识;3.linux下用fdisk删除硬盘分区和RAID标识。这三中方法都是在linux系统下,所以就用U盘作了个Debian Live系统,用第三中方法试了下,可行。下边就分别记录下三种使用方法的具体过程。
\subsubsection{Debian Live清除RIAD标识}
之后你可以在窗口输入一篇小文档,并保存为tex扩展名的文件进行测试:
\begin{latex}
\documentclass{ctexart}
\begin{document}
Hello,world!
你好,世界!
\end{document}
\end{latex}
点击编译按钮生成,F7查看。生成pdf在你的tex文件保存目录中。具体各行的含义我们后在后文介绍。

\section{Windows操作系统}
Windows操作系统安装有两种:一种是用原版的操作系统安装介质,一种是用别人GHOST的备份来还原。前者是安装步骤较多,最后需要自己激活甚至安装驱动,但是干净,后者是安装方便,傻瓜式一键安装,但是系统有别人给预装的插件广告等(自己脑补吧),我喜欢后者,原因很简单干净、还算有点挑战性(大折腾吗)。
以下是针对使用前者安装方式的
Windows操作系统安装最重要的几点就是安装介质、系统激活和安装驱动(尤其是安装完没有网卡驱动的)。
磁盘规划:最好是一块固态硬盘直接安装操作系统和软件,一块机械硬盘按需求分区后存放数据,这样备份恢复系统,迁移数据最方便。
\subsection{XP操作系统}
曾经神一样的存在,只是时代在进步,不论是从硬件还是软件上它都不足以支撑,不过工业上还有很多在用,因为工业软件需求单一,稳定为主没有更新,操作系统还使用经典的XP系统。
我记得2008年时开始接触操作系统,后来就帮人装系统,那个时候就是XP,后来不知道从哪里找到一个XP系统ISO安装文件(Microsoft Windows XP Professional版本2002 Service Pack3),然后又试出一个\index{XP系统注册码}:CM3HY-26VYW-6JRYC-X66GX-JVY2D,可以一直重复使用的哦,值得提一句的是XP系统是安装过程中就要求输入授权码的,否则就不能继续往下走了。然后就一直使用它们,直到2020年上班后因为连接西门子S7200PLC时还在虚拟机上安装了XP系统专门用来上载、修改、下载PLC程序。
\subsection{Win7/Win10操作系统}
Win7和Win10安装方式基本一样,其实Win7及后来的操作系统安装方式基本没有什么变化,都是安装后进行注册激活。相比XP多了一种安装方式,硬盘安装,这种方式前提是安装操作系统的电脑有旧的有windows操作系统存在,在旧操作系统下将ISO镜像文件解压到D盘根目录下然后点击SETUP安装程序即可,简单吧。

下边就以记录下我最近安装WIN10系统的过程吧,这个WIN10安装介质是我们厂SIS系统安装时厂家在客户端安装的WIN10(Windows 10专业版)系统时用的正版光盘,我做成了ISO文件。
\begin{enumerate}
	\item 拷贝WIN10.ISO文件到安装Ve的U盘
	\item 电脑U盘启动选择WIN10,nomal install
	\item 系统安装过程。。。
	\item 安装完成后检查设备驱动正常,简直不要太顺
\end{enumerate}
\subsubsection{Win7操作系统激活}
Win7系统激活是有一个激活程序(PCSKYS\_Windows7Loader\_v3.27.exe),安装万系统后,打开激活软件按照提示激活,如果激活失败需要到我的电脑->管理->磁盘管理里检查是不是有盘符是空白,给空白盘符添加驱动号后,再重新激活
\subsubsection{激活WIN10系统}
目前在Windows 10专业版测试是可以激活的
\begin{enumerate}
	\item 进入win10系统桌面中,鼠标右键点击桌面左下角的window按钮,在弹出的菜单中选择“命令提示符(管理员)”选
	\item 在打开的命令符号符界面中输入slmgr.vbs /upk命令,将win10系统中原来的激活密钥卸载,进入下一步,弹出成功卸载了产品密匙
	\item 输入slmgr /ipk NPPR9-FWDCX-D2C8J-H872K-2YT43,弹出:"成功的安装了产品密钥"
	\item 输入slmgr /skms zh.us.to,弹出"密钥管理服务计算机名成功的设置为zh.us.to"
	\item 输入slmgr /ato,弹出"成功的激活了产品"
	\item 在系统属性界面中可以看到win10企业版的激活状态
\end{enumerate}
\LaTeX\ 中的\co{命令}通常是由一个反斜杠加上命令名称,再加上花括号内的参数构成的(有的命令不带参数,例如:\latexline{TeX})。
\begin{latex}
\documentclass{ctexart}
\end{latex}
如果有一些选项是备选的,那么通常会在花括号前用方括号标出。比如:
\begin{latex}
\documentclass[a4paper]{ctexart}
\end{latex}
还有一种重要指令叫做\co{环境}。它被定义与控制命令\latexline{begin\{environment\}}\\和\latexline{end\{environment\}}间的内容。比如:%不加入强制换行会溢出?
\begin{latex}
\begin{document}
...内容...
\end{document}
\end{latex}
环境如果有备选参数,只需要写在\latexline{begin[...]\{name\}}这里就行。

注意:不带花括号的命令后面如果想打印空格,请加上{\color{cyan}{一对内部为空的花括号}}再键入空格。否则空格会被忽略。例如:\verb+\LaTex{} Studio+。
\subsection{保留字符}
\LaTeX\ 中有许多字符有着特殊的含义,在你生成文档时不会直接打印。例如每个命令的第一个字符:反斜杠。单独输入一个反斜杠在你的行文中不会有任何帮助,甚至可能产生错误。\LaTeX\ 中的保留字符有:
\begin{center}
	\texttt{\# \$ \% \^ \& \_ \{ \} \char92}
\end{center}

它们的作用分别是:
%\begin{description}[font=\bfseries\ttfamily,itemsep=0pt,parsep=0pt,topsep=0pt,partopsep=0pt]
\begin{para}
\item[\#{}:]自定义命令时,用于标明参数序号。
\item[\S{}:]数学环境命令符。
%\end{description}
\end{para}

以上除了反斜杠外,均能用前加反斜杠的形式输出。即你只需要键入:
\begin{center}
\verb|\# \$ \% \^ \& \_ \{ \}|
\end{center}

唯独反斜杠的输出比较头痛,你可以尝试:
\begin{codeshow}
$\backslash$ \textbackslash
\texttt{\char92}
\end{codeshow}

其中命令\latexline{char[num]}是一个特殊的命令
\begin{verbatim}
\texttt{\char`~}%输出一个波浪线
\end{verbatim}

\subsection{导言区}
任何一份\LaTeX{}文档都应该包含以下结构:
\begin{latex}
\documentclass['\itshape options']{doc-class}%没有斜体option
\begin{document}
...
\end{document}
\end{latex}
其中,在语句\latexline{begin\{document\}}之前的内容成为\co{导言区}。导言区可以留空,也可以进行一些、文档的准备操作。你可以粗浅地理解为:\RED{导演区即模板定义}。\\

文档类的参数doc-class和可选选项{\textit{options}}有取值:%引用图表失败
%\begin{table}[!htb]原文中有htb,现在无法正确编译
	\begin{table}[!htb]
	\centering
	\caption{文档类和选项}
	\label{tab:documentclass}
	\begin{tabular}{p{5em} @{\ -\ } p{24em}}
		\hline
		\multicolumn{2}{l}{\bfseries doc-class文档类\footnotemark}\\
		\hline
		article & 科学期刊,演示文稿,短报告,邀请函。\\
		proc	& 基于article的会议论文集。\\
		report	& 多章节的长报告、博士论文、短篇书。\\
		book	& 书籍。\\
		slides	& 幻灯片,使用了大号Scans Serif字体。\\
		\hline
		\multicolumn{2}{l}{\bfseries\itshape options}\\
		\hline
		字体	&默认10pt,可选11pt和12pt。\\
		\hline
	\end{tabular}
\end{table}

在本文中,多数的文档类提及的均为report/book类。如果有article类将会特别指明。其余的文档类不与说明。本手册排版即使用了report类。

在导言区最常见的是\co{宏包}的加载工作,命令形式如:\latexline{usepackage\{package\}}。通俗地讲,宏包是指一系列已经制作好的功能``模块'',在你需要使用一些原生\LaTeX\ 不带有的功能时,只需要调用这些宏包就可以了。比如文本的代码就是利用\pkg{listings}宏包实现的。

宏包的具体使用将参在个部分内容说明中进行讲解。如果你想学习一个宏包的使用,按Win+R组合键呼出运行对话框,输入texdoc加上宏包名称即可打开宏包帮助pdf文档。例如:\verb|texdoc xeCJK|。
	\footnotetext{此外还有\pkg{beamer}宏包定义的beamer文档类,常用于创建幻灯片。}

\subsection{错误的排查}
	\label{subsec:debug}
	在编辑器界面上,下方的日志是显示编译过程的地方。在你编译通过后,会出现这样的字样:
\begin{feai}
	\item {\bfseries{Errors错误}}:严重的错误。一般地,编译若通过了,该项是零。
	\item {\bfseries{Warnings警告}}:一些不影响生成文档的瑕疵。
	\item {\bfseries{Bad Boxes坏箱}\footnote{Box是\LaTeX{}中的一个特殊概念,具体将在进行讲解。}}:指排版中出现的长度问题,比如长度超出(Overfull)等。后面的Badness表示错误的严重程度,程度越高数值越大。这类问题需要检查,排除Badness高的选项。\marginpar{\textcolor{red!70}{此处注解在后续章节,目前未链接}}
\end{feai}

	你可以向上翻越日志记录(即.log文件),来找到Warning开头的记录,或者Overfull/ Underfull开头的记录。这些记录会指出你的问题出在哪一行(比如line 1-2)或者在pdf的哪一页(比如active[12]。注意,这个12表示计数器技术页码,而不是文件打印出来的真实页数)。此外你还需要了解:
\begin{feai}
\item 值得指出的是,由于\LaTeX{}的编译原理(第一次生成aux文件,第二次再引用它),目录想要合理显示{\bfseries{需要连续编译两次}}。在连续编译两此后,你会发现一些Warnings会在第二次编译后消失。在\TeX\ Studio中,你可以只单击一次“构建并查看”,他会检测到文章的变化并自动决定是否需要编译两次。
\item 对于大型文档,寻找行号十分痛苦。你需要学会合理地拆分tex文件,参阅内容。%未引用3.13节内容
\end{feai}

	这里也推荐宏包\pkg{syntonly},在导言区加入它支持的\latexline{syntaxonly}命令,会只排查语法错误而不生成任何文档,这可以使你更快地编译。不过他似乎不太稳定,例如本文档可以正常编译,但是使用该命令时则会出错。
\subsection{文件输出}
	\LaTeX{}的输出一般推荐pdf格式,有\LaTeX\ 直接生成dvi的方法并不推荐。

你在tex文档的文件夹下可能看到的其他文件类型:
\begin{tabbing}
	.sty{\hspace{2em}}\=宏包文件\\%第一行必须此格式否则会编译出错
	.cls	\> 文档类文件。\\
	.aux	\> 用于存储交叉引用信息的文件。\\
	.out	\> 宏包生成的pdf书签记录。
\end{tabbing}
\section{UNIX操作系统}
	UNIX系统种类很多啊,尤其是近几年,越来越火,记得我是2009年第一次接触Ubuntu,那时候只是好奇,为了好玩,去电脑商城买电脑的时候非要选预装Ubuntu系统的笔记本,别人看我的眼神是那样的,最后转了一圈也没找下,不过最后找下了一款Nseries标志的笔记本电脑,就是现在用的DELL vostro 1088,到现在已经15年了,记得当时买回去第一件事情就是重装Ubuntu系统,当时纯属装C,不过也多亏了那股劲才坚持到了现在,中间学习操作系统的时候还安装了minix操作系统,后来上班后真正搭建服务器时用上了CentOS,CentOS7后就停止维护了寻找替代系统过程中试着用Debian,这会儿已经到了Debian12版本了,不得不说Debian真是稳定啊,占用资源少,果断把15年的笔记本换成Debian,健步如飞啊。就剩Fedora了,有机会了试试。
	\subsection{minix操作系统}
	英文单引号并不使用两个\verb|'|符号组合。左单引号是重音符\verb|`|(键盘上数字1左侧),而右单引号是常用的引号符。英文中,{\color{cyan}{左双引号就是连续两个重音符号}}。
	英文下的引号嵌套需要借助\latexline{thinspace}命令分隔,比如:
\begin{codeshow}[listing side text, listing options={escapeinside=++}]%方括号里的是啥意思?
``\thinspace`Max' is here.''
\end{codeshow}
中文下的单引号和双引号你可以用中文输入法直接输入。

\subsection{Linux操作系统}
	接触Linux操作系统很早大约是在10年,那个时候我记得用的还是CentOS5,那会主要是接触了一本名叫一个Orange操作系统的书籍就迷上了操作系统的实现,这本书是在linux操作系统下使用Boch虚拟机来测试调试实现的迷你操作系统,中间会用到许多修改查看二进制数据的小工具,这些工具linux系统都自带不需要安装很方便,所以就开始使用Linux操作系统了,不过那会儿的Linux操作系统体验很差,只有在写操作系统的时候才使用,后来慢慢有所改善,一直到CentOS8,后来CentOS系统停止发布,又改为了CentOS Stream,感觉不好就开始寻找其它的Linux操作系统,最后选择了Debian,使用后感觉Debian是真稳定啊。
	linux系统下边需要特别说的两点分别是硬盘规划和软件源配置。
英文的短横分为三种:
\begin{feai}
\item 连字符:输入一个短横:\verb|-|,效果如daughter-in-law
\item 数字起止符:输入两个短横:\verb|--|,效果如:page 1--2
\item 破折号:输入三个短横:\verb|---|,效果如:Listen---I'm serious.
\end{feai}

中文的破折号你也许可以直接使用日常的输入方式。中文的省略号同样。但是注意,英文的省略号使用\latexline{ldots}这个命令而不是三个句点。
\subsection{Debian操作系统}
使用Debian系统比较晚大约是在2023年,当时最新版本是Debian11代号Bullseye,这部分内容更新于2025年2月份,最新版本Debian系统为12代号Bookworm,Debian11和Debian12的区别是Debian11拥有大量成熟的软件包适合在服务器和生产环境中使用,Debian12功能更新颖,软件版本更新对部分软件支持不是很好,更适合尝鲜和体验。所以当时我就在现场使用的笔记本上安装了Debian11,在个人笔记本上安装了Debian12,这样两者就都能兼顾了。
下面以Debian11和12系统安装为例子简单记录下,系统安装完成后需要优化的一些操作。需要特别注意的是安装Debian11过程中先不要配置网络参数,否则安装过程特被漫长,感觉它在网络上下载镜像源而不是使用本地ISO镜像源,安装过程中有个选项是不使用网络镜像源,但是还是特别慢,如果不配置网络参数的话也就40分钟就安装完成了,Debian12安装不会有这种现象,只用选着不使用网络镜像源就可以了。
\begin{enumerate}
	\item 配置镜像源,系统安装完成apt源默认配置的是本地光盘镜像,网络镜像源速度比较慢,更新为国内镜像源,\href{https://blog.csdn.net/qq_48118072/article/details/132339096?spm=1001.2014.3001.5506}{Debian12源地址},\href{https://blog.csdn.net/zqr4818/article/details/129657792?spm=1001.2014.3001.5506}{Debian11源地址}。
\begin{shell}
su#登陆root权限
cd /etc/apt/#换到源地址配置目录
move sources.list sources.list.back#备份原来的源文件
#用指定源替代sources.list文件中的源
apt-get update#更新源
\end{shell}
	\item 将用户添加到sudo组中,将当前用户添加到sudo组中,这样在执行需要root权限时临时使用sudo命令即可,直接登陆root用户比较危险哦!将用户添加到sudo组后需要重启电脑后方可完全生效,这一点不是很明白,理论上应该不需要重启才对。
\begin{shell}
su#登陆root权限,这个时候还是需要的哦
apt-get install sudo#安装sudo软件
/usr/sbin/usermod -a -G sudo $(echo $USER)
#将当前用户添加至sudo组
#以后使用sudo输入用户密码就可以临时切至root权限了
\end{shell}
	\item 将/usr/sbin添加至环境变量,不然/usr/sbin下命令使用时还需要使用路径,比较麻烦
\begin{shell}
sudo vim /etc/profile#需要root权限才能修改该文件
#在最后一行加入
export PATH=$PATH:/usr/sbin#$将/usr/sbin加入PATH
#这样该目录下程序就可以直接执行了,比如上例中的usermod命令
#如果新安装软件需要的话也可以加进去,比如我们的\LaTeX
#保存退出后
source /etc/profile#重新加载配置文件使修改生效
\end{shell}
	\item 设置Debian启动后默认运行级别,如果是个人PC电脑的话默认的graphical.target就可以,如果是服务器的话就需要设置成multi-user.target级别(服务器一般不设置桌面环境,浪费资源而且桌面不稳定因素较多影响服务器稳定性)。
\begin{shell}
systemctl get-default#查看当前运行级别
graphical.target#图形话桌面
systemctl list-units --type=target
#查看可供替换的运行级别
systemctl set-default multi-user.target(修改为多用户文本)
startx#多用户文本级别下启动图形桌面

\end{shell}
	\item Debian系统桌面环境安装与切换,一般安装是就已经默认安装号了桌面环境和启动配置,如果使用过程中想要切换桌面的画可以参考\href{https://blog.csdn.net/seaship/article/details/86234453?spm=1001.2014.3001.5506}{这里}。这里主要记录下服务器设置为多用户文本启动后,startx默认启动的桌面环境。
\begin{shell}
update-alternatives --config x-session-manager#设置默认桌面
\end{shell}
\end{enumerate}

\LaTeX\ 中专门有个叫做\latexline{emph\{text\}}的命令,可以强调文本。对于通常的西文文本,上述命令的作用就是斜体。如果你对一段已经这样转换为斜体的文本再使用这个命令,它就会取消斜体,而成为正体。

西文中一般采用上述的斜体强调方式而不是粗体,例如在说明书的时候可能就会使用以上命令。关于字体更多内容参考字体这一节。

\section{常用命令}
\LaTeX\ 原生提供的\latexline{underline}命令简直烂的可以,建议你使用\pkg{ulem}宏包下的\texttt{uline}命令代替,它还支持换行文本。\pkg{ulem}宏包还提供了一些实用命令:
\subsection{systemctl命令}
	systemctl 检查和控制系统与服务管理的状态
	\begin{description}
		\item[systemctl] COMMAND server
	\end{description}
\begin{shell}
systemctl start apache2#启动apache2服务
systemctl status apache2#检查apache2服务状态
systemctl stop apache2#停止apache2服务
systemctl enable apache2#设置apache2服务开机自启动
systemctl disable apache2#禁止apache2服务开机自启动
\end{shell}
\subsection{ssh命令}
	OpenSSH SSH 客户端 (远程登录程序)
	\begin{description}
		\item[ssh] [-p port] user@hostname
		\item[-p] port 指定远程主机端口
	\end{description}
\begin{shell}
ssh debian_ibm@dklovelich.iok.la -p 18210 #通过外网使用域名登录远程debian系统
ssh debian_ibm@192.168.1.16 -p 22 #通过局域网使用IP地址登录debian系统
\end{shell}
\subsection{scp命令}
	安全复制(远程文件复制程序,用法和cp命令相似)
	\begin{description}
		\item[scp] [user@host1:]file1 [user@host2:]file2
		\item[-r] 递归复制整个目录
		\item[-P] port是指定数据传输用到的端口号,默认22端口
	\end{description}
\begin{shell}
scp -P 18210 file1 debian_ibm@dklovelich.iok.la:/home/debian_ibm/dk/ #从本地将文件file1传输到服务器/home/debian_ibm/dk/目录下
scp -Pr 18210 dir1 debian_ibm@dklovelich.iok.la:/home/debian_ibm/dk/ #从本地将文件夹dir1传输到服务器/home/debian_ibm/dk/目录下
scp -P 18210 debian_ibm@dklovelich.iok.la:/home/debian_ibm/dk/file1 ./ #将服务器上的file1文件传输到本地./目录下
scp -Pr 18210 debian_ibm@dklovelich.iok.la:/home/debian_ibm/dk ./ #将服务器上的dk文件夹传输到本地./目录下
\end{shell}
\subsection{screen命令}
	screen是一个多任务窗口管理器
	\begin{description}
		\item[screen] [-ls]|[-S sessionname]|[-r sessionname]
		\item[-ls] 列出所有会话
		\item[-S] 指定会话名称
		\item[-r] 恢复会话
		\item[-d] 断开指定的会话,但不会杀死会话中的任务
	\end{description}
\begin{shell}
screen -ls #查看当前正在运行的说有会话
screen -S log #启动一个名为log的会话
screen -r log #恢复一个已经脱离的名为log的会话
screen -d log #断开log的会话,但不会杀死会话中的任务
\end{shell}
\subsection{rar命令}
	rar是解压缩rar压缩文件的命令
	\begin{description}
		\item[rar] [-ls]|[-S sessionname]|[-r sessionname]
		\item[-ls] 列出所有会话
		\item[-S] 指定会话名称
		\item[-r] 恢复会话
	\end{description}
\begin{shell}
rar x file.rar#解压缩file.rar文件到当前目录
\end{shell}
\subsection{du命令}
	du是linux系统里的文件大小查看的命令
	\begin{description}
		\item[du] [-ls]|[-S sessionname]|[-r sessionname]
		\item[-s] 查看文件夹总大小
		\item[-h] 智能显示文件大小
		\item[-r] 恢复会话
	\end{description}
\begin{shell}
du -sh dk#查看dk文件夹总大小
\end{shell}
\subsection{grep命令}
	grep是linux系统里的搜索命令
	\begin{description}
		\item[grep] [-rn] char dir
		\item[-r] 当前路径下循环搜索
		\item[-n] 结果输出显示行号
		\item[-l] 只显示文件名,不显示匹配的文本
		\item[-I] 忽略匹配二进制文件
	\end{description}
\begin{shell}
grep -rn  topnav2.php ./
#递归查找当前目录下所有包含topnav2.php字符串的文件
\end{shell}
\subsection{sed命令}
	du是linux系统里流编辑器,用于批量修改文件内容
	\begin{description}
		\item[du] [-nefri]|[动作]
		\item[-i] 直接修改读取的文件内容
		\item[s] 替换动作
	\end{description}
\begin{shell}
sed -i "s/topnav2.php/topnav.php/g" `grep "topnav2.php" -rIl ./`
#替换当前目录下除了二进制文件的所有文件中topnav2.php字符串为topnav.php
#grep -I参数很有必要,在git目录下会避免修改git仓库内二进制文件
#否则损坏git仓库索引导致git不可使用
\end{shell}




		%!TEX root = ../dk_log.tex
\chapter{软件安装配置}
这一章主要记录使用过的一些值得记录的软件的安装、配置和使用。
\section{常用的软件}
	第一款软件无疑就是VIM了,那第二款肯定是Latex了,这玩意就是用他两鼓捣出来的。	编辑器的配置大概是需要讲解一下的,毕竟对于初学者来说是很头疼的事情。本手册就以\TeX\ studio为例进行配置。首先你应该安装一个\TeX{} Live,他是完全免费的,网址:\url{http://tug.org/texlive/}。

虽然它体积较大,但是却是最一劳永逸、最不需要花时间去配置的方法,同时它大概也是功能支持最强的\LaTeX\ 发行版。

打开\TeX\ Studio后,选择选项$\rightarrow$设置\TeX\ Studio $\rightarrow$ 构建 $\rightarrow$ 默认编译器,选择\xelatex{}。这主要是基于中文文档编译的考虑,同时\xelatex 也能很好的编译英文文档。我建议始终使用它作为默认编译器。


之后你可以在窗口输入一篇小文档,并保存为tex扩展名的文件进行测试:
\begin{latex}
\documentclass{ctexart}
\begin{document}
Hello,world!
你好,世界!
\end{document}
\end{latex}
点击编译按钮生成,F7查看。生成pdf在你的tex文件保存目录中。具体各行的含义我们后在后文介绍。

\section{VIM}
vim堪称上古神器,第一小节我们给这款神奇再叠加buff。
\subsection{vim-plug}
刚安装的VIM功能都比较朴实,但是VIM的扩展性很强可以安装各种各样的插件,vim-plug是一款VIM的插件管理工具,有了它可以很方便的安装、调用、卸载各种插件,项目网址:\url{https://github.com/junegunn/vim-plug}。

下载plug.vim文件至~/.vim/autoload/下,同时创建~/.vim/plugged目录用来放置将来要安装的插件,配置~/.vimrc文件并添加调用插件模块,Debian12用户根目录下没有vimrc文件,复制/usr/share/vim/vim90/目录下vimrc\_example.vim文件即可。

还有一个关于Vim插件的网站是\href{https://vimawesome.com}{VimAwesome},上边有几乎所有的Vim插件安装使用方法,其中有一篇叫\href{https://vimawesome.com/plugin/vim-plug-own-character}{vim-plug的文章}是一位大神Vim的配置文件,可以很容易配置一款强大的Vim。

\begin{shell}
ls ~/ #~目录
.vim
#用户目录下没有vimrc文件
cp /usr/share/vim/vim90/vimrc_example.vim ~/.vimrc
ls ~/ #~目录
.vim .vimrc
ls ~/.vim
autoload plugged
ls ~/.vim/autoload
plug.vim#plug.vim下载原文件

vim ~/.vimrc #在.vimrc文件尾部添加下面内容
..........
call plug#begin('~/.vim/plugged')
#Plug 要安装的插件
call plug#end()

:PlugStatus#重新打开vim后在命令模式下使用该命令查看插件安装情况

:PlugInstall#在vim命令模式下安装插件

:PlugClean#卸载插件,需要先在vimrc配置文件中删除对应插件配置信息
:PlugUpgrade#更新vim-plug插件自身
\end{shell}


\subsubsection{NERDTree}
NERDTree是Vim编辑器的文件系统资源管理器。
\begin{shell}
vim ~/.vimrc #在.vimrc文件尾部添加下面内容
..........
call plug#begin('~/.vim/plugged')
Plug 'preservim/nerdtree'#加载nerdtree插件
call plug#end()
:source %#重新加载.vimrc配置文件
:PlugInstall#在vim命令模式下安装插件
:NERDTree#启动NERDTree插件
\end{shell}
\subsubsection{markdown-preview}
markdown-preview插件可以让vim在编辑markdown文件时在浏览器下同步滚动展示效果,是同步哦!
\begin{shell}
vim ~/.vimrc #在.vimrc文件尾部添加下面内容
..........
call plug#begin('~/.vim/plugged')
Plug 'iamcco/markdown-preview.nvim'#加载插件
call plug#end()
:source %#重新加载.vimrc配置文件
:PlugInstall#安装插件
:call mkdp#util#install()#安装支持软件
:MarkdownPreview#开始预览
:MarkdownPreviewStop#停止预览
\end{shell}
\subsection{VIM常用操作}
\subsubsection{使用寄存器与剪切板}
	\index{剪切板寄存器}用于与系统剪切板进行交互
\begin{shell}
vim --version|grep clipboard #查看clipboard与xtem_clipboard功能是否开启
#-clipboard -xtem_clipboard说明功能没有开启
sudo apt-get install vim-gtk3#安装对应功能模块
#再次查询为+clipboard +xtem_clipboard说明功能已开启,Debian12下
"+yy#复制当前行到系统剪切板
"+p#粘贴至其它文件中
\end{shell}
\LaTeX\ 中有许多字符有着特殊的含义,在你生成文档时不会直接打印。例如每个命令的第一个字符:反斜杠。单独输入一个反斜杠在你的行文中不会有任何帮助,甚至可能产生错误。\LaTeX\ 中的保留字符有:
\begin{center}
	\texttt{\# \$ \% \^ \& \_ \{ \} \char92}
\end{center}

它们的作用分别是:
%\begin{description}[font=\bfseries\ttfamily,itemsep=0pt,parsep=0pt,topsep=0pt,partopsep=0pt]
\begin{para}
\item[\#{}:]自定义命令时,用于标明参数序号。
\item[\S{}:]数学环境命令符。
%\end{description}
\end{para}

以上除了反斜杠外,均能用前加反斜杠的形式输出。即你只需要键入:
\begin{center}
\verb|\# \$ \% \^ \& \_ \{ \}|
\end{center}

唯独反斜杠的输出比较头痛,你可以尝试:
\begin{codeshow}
$\backslash$ \textbackslash
\texttt{\char92}
\end{codeshow}

其中命令\latexline{char[num]}是一个特殊的命令
\begin{verbatim}
\texttt{\char`~}%输出一个波浪线
\end{verbatim}

任何一份\LaTeX{}文档都应该包含以下结构:
\begin{latex}
\documentclass['\itshape options']{doc-class}%没有斜体option
\begin{document}
...
\end{document}
\end{latex}
其中,在语句\latexline{begin\{document\}}之前的内容成为\co{导言区}。导言区可以留空,也可以进行一些、文档的准备操作。你可以粗浅地理解为:\RED{导演区即模板定义}。\\

文档类的参数doc-class和可选选项{\textit{options}}有取值:%引用图表失败
%\begin{table}[!htb]原文中有htb,现在无法正确编译
	\begin{table}[!htb]
	\centering
	\caption{文档类和选项}
	\label{tab:documentclass}
	\begin{tabular}{p{5em} @{\ -\ } p{24em}}
		\hline
		\multicolumn{2}{l}{\bfseries doc-class文档类\footnotemark}\\
		\hline
		article & 科学期刊,演示文稿,短报告,邀请函。\\
		proc	& 基于article的会议论文集。\\
		report	& 多章节的长报告、博士论文、短篇书。\\
		book	& 书籍。\\
		slides	& 幻灯片,使用了大号Scans Serif字体。\\
		\hline
		\multicolumn{2}{l}{\bfseries\itshape options}\\
		\hline
		字体	&默认10pt,可选11pt和12pt。\\
		\hline
	\end{tabular}
\end{table}

在本文中,多数的文档类提及的均为report/book类。如果有article类将会特别指明。其余的文档类不与说明。本手册排版即使用了report类。

在导言区最常见的是\co{宏包}的加载工作,命令形式如:\latexline{usepackage\{package\}}。通俗地讲,宏包是指一系列已经制作好的功能``模块'',在你需要使用一些原生\LaTeX\ 不带有的功能时,只需要调用这些宏包就可以了。比如文本的代码就是利用\pkg{listings}宏包实现的。

宏包的具体使用将参在个部分内容说明中进行讲解。如果你想学习一个宏包的使用,按Win+R组合键呼出运行对话框,输入texdoc加上宏包名称即可打开宏包帮助pdf文档。例如:\verb|texdoc xeCJK|。
	\footnotetext{此外还有\pkg{beamer}宏包定义的beamer文档类,常用于创建幻灯片。}

\subsection{错误的排查}
	\label{subsec:debug}
	在编辑器界面上,下方的日志是显示编译过程的地方。在你编译通过后,会出现这样的字样:
\begin{feai}
	\item {\bfseries{Errors错误}}:严重的错误。一般地,编译若通过了,该项是零。
	\item {\bfseries{Warnings警告}}:一些不影响生成文档的瑕疵。
	\item {\bfseries{Bad Boxes坏箱}\footnote{Box是\LaTeX{}中的一个特殊概念,具体将在进行讲解。}}:指排版中出现的长度问题,比如长度超出(Overfull)等。后面的Badness表示错误的严重程度,程度越高数值越大。这类问题需要检查,排除Badness高的选项。\marginpar{\textcolor{red!70}{此处注解在后续章节,目前未链接}}
\end{feai}

	你可以向上翻越日志记录(即.log文件),来找到Warning开头的记录,或者Overfull/ Underfull开头的记录。这些记录会指出你的问题出在哪一行(比如line 1-2)或者在pdf的哪一页(比如active[12]。注意,这个12表示计数器技术页码,而不是文件打印出来的真实页数)。此外你还需要了解:
\begin{feai}
\item 值得指出的是,由于\LaTeX{}的编译原理(第一次生成aux文件,第二次再引用它),目录想要合理显示{\bfseries{需要连续编译两次}}。在连续编译两此后,你会发现一些Warnings会在第二次编译后消失。在\TeX\ Studio中,你可以只单击一次“构建并查看”,他会检测到文章的变化并自动决定是否需要编译两次。
\item 对于大型文档,寻找行号十分痛苦。你需要学会合理地拆分tex文件,参阅内容。%未引用3.13节内容
\end{feai}

	这里也推荐宏包\pkg{syntonly},在导言区加入它支持的\latexline{syntaxonly}命令,会只排查语法错误而不生成任何文档,这可以使你更快地编译。不过他似乎不太稳定,例如本文档可以正常编译,但是使用该命令时则会出错。
\subsection{文件输出}
	\LaTeX{}的输出一般推荐pdf格式,有\LaTeX\ 直接生成dvi的方法并不推荐。

你在tex文档的文件夹下可能看到的其他文件类型:
\begin{tabbing}
	.sty{\hspace{2em}}\=宏包文件\\%第一行必须此格式否则会编译出错
	.cls	\> 文档类文件。\\
	.aux	\> 用于存储交叉引用信息的文件。\\
	.out	\> 宏包生成的pdf书签记录。
\end{tabbing}
\section{Git}
	Git是一款分布式版本控制系统,Git最初是由Linux开发者Linus用两周时间编写的,用来管理Linux源代码的。谈到Git就必须提到大名鼎鼎的\href{https://github.com/}{github},它用的就是git系统来管网站的,github和git是两个东西,github是一个社区,git是一个服务系统,github只支持git分布式系统,所以故名成为github。
\subsection{Git安装与配置}
\subsubsection{linux系统下安装配置git}
\href{https://blog.csdn.net/bjbz_cxy/article/details/116703787?spm=1001.2014.3001.5506}{linux下安装和配置相当方便},甚至git已经内置与linux系统。

\begin{shell}
sudo apt install git#安装git
git config --list#查看git配置,用户名、邮箱
git config --global user.name "dk_huapen"#配置用户名称
git config --global user.email "zhaohuapeng7947@163.com"#配置用户邮箱
mkdir mygit
cd mygit
git init#初始化本地当前目录mygit为仓库
git status#查看本地仓库文件状态
git add file#将文件file提交到本地仓库的缓存区
git add --all#将所有改动过的文件提交到缓存区
git add .#将所有改动过的文件提交到缓存区
git commit -m "描述信息"#将本地仓库的提交缓存提交到本地仓库
git log#查看历史提交日至
#指定哪些文件或目录应该被Git忽略,不纳入版本控制
vim .gitignore
*~#忽略最后一个文件名字符是~的文件,用于忽略vim编辑生成的中间文件
\end{shell}
\subsubsection{windows系统下安装配置git}
\href{https://blog.csdn.net/weixin_30824277/article/details/98382857?spm=1001.2014.3001.5506}{windows下需要额外安装git},安装git后会有一个类是linux的命令行工具Git Bash。
\subsection{Git常用命令}
\subsection{github}
github是一款使用git命令作为基础框架的开源网站,可以把自己的项目分享到github上,gibhub上是无限制代码的。首先需要在github上注册一个账号,我使用的是自己的163邮箱,密码是死骑love巫妖!其实国内也有许多可以平替github的平台,而且github是国外平台,经常会网卡甚至掉线,不过我还是选择使用github,因为它足够原生吧,就像我选择Debian一样。
\subsubsection{本地git与github关联}
\begin{shell}
#在本地创建一个ssh的可以,使用自己的邮箱
ssh-keygen -t rsa -C "zhaohuapeng7947@163.com"
ls
id_rsa#ssh生成的私匙,不能告诉任何人
id_rsa.pub#ssh生成的公匙,添加到github中的
ssh -T git@github.com
#测试ssh keys是否设置成功,也就是本地git与github是否关联成功
#下面的操作需要本地与github关联成功后才可以正常使用
#将本地仓库关联至github仓库并推送本地仓库至github
git remote add origin github仓库ssh地址#本地仓库与github仓库关联
git push -u origin master#将本地仓库推送至github
#clone的仓库自动关联github仓库
git clone <url>#用github远程默认仓库main初始化本地仓库
git clone <url> myhubname
#用github远程默认仓库main初始化本地仓库时自定义本地仓库名字
git clone -b master <url>#用github远程仓库master分支初始化本地仓库
git push#将本地仓库更新推送至github
git pull#从github获取更新

\end{shell}
\subsubsection{github加速}


\section{make}
make是一个自动化工具。
\section{\LaTeX}
\LaTeX\ 是免费的、开源的排版系统,它的设计目标是分离内容与格式,以便作者能够专注于内容创作而非版式设计,并能得到高质量的排版作品。
	\subsection{latex安装与配置}
	latex的光盘镜像发布于https://www.tug.org/texlive。在光盘镜像中texlive-doc目录下有一份texlive-zh-cn.pdf文件,里边详细描述了各操作系统安装latex的方法。

	这里主要说两点注意事项:
	\begin{enumerate}
		\item 默认安装目录在/usr/local/texlive/下,安装完成后该目录有8.7GB,安装系统时需要预留够磁盘分区空间
		\item 安装菜单中create symlinks in standard directories选项默认不选,安装手册也不推荐,所以我安装也没选,安装完成后就得手动添加texlive安装路径到环境变量中。
	\end{enumerate}
\begin{shell}
#在/etc/profile文件尾部加入所有用户都生效
#在~/.profile文件尾部加入当前用户生效$
PATH=$PATH:/usr/local/texlive/2023/bin/x86_64-linux
export PATH
\end{shell}
\subsection{latex常用命令}
\subsubsection{texdoc命令}
	\index{texdoc}可以查询宏包帮助文档和其他相关的有用文档
	\begin{description}
		\item[texdoc] [想要查询的内容]
	\end{description}
\begin{shell}
texdoc fancyhdr #查看fancyhdr宏包帮助文档
texdoc usrguide #自带的用户手册
texdoc clsguide #自带文档类和宏包编写手册
texdoc fntguide #自带的字体使用手册
texdoc symbols-a4 #命令速查表
texdoc latexcheat #有趣的命令表
\end{shell}
\subsubsection{latexmk命令}
	\index{texdoc}\href{https://blog.csdn.net/bleedingfight/article/details/84946793?ops_request_misc=%257B%2522request%255Fid%2522%253A%2522a782193c5e5af7bfb740173a44657da1%2522%252C%2522scm%2522%253A%252220140713.130102334..%2522%257D&request_id=a782193c5e5af7bfb740173a44657da1&biz_id=0&utm_medium=distribute.pc_search_result.none-task-blog-2~all~sobaiduend~default-1-84946793-null-null.142^v101^pc_search_result_base7&utm_term=latexmk&spm=1018.2226.3001.4187}{latexmk命令}可以实现\LaTeX\ 文档自动编译,指定输出目录等功能。我目前在Makefine文件中使用,看说明和论坛好像该命令本身就类是Make工具,不需要借助make就可以实现自动编译,目前还没弄明白这一点。
	\begin{description}
		\item[texdoc] [参数][file]
	\end{description}
\begin{shell}
latexmk -xelatex file.tex #自动编译file.tex文件
latexmk -c #清空指定临时文件
latexmk ? #指定输出文件目录,这个暂时没弄明白
\end{shell}

中文的破折号你也许可以直接使用日常的输入方式。中文的省略号同样。但是注意,英文的省略号使用\latexline{ldots}这个命令而不是三个句点。
\section{MySQL}
MySQL是一款关系型数据库管理系统,这里主要记录下在Windows系统下的安装配置方法,在Linux下使用的是MariaDB,在搭建LAMP服务器中再具体介绍MariaDB的安装配置方法。MariaDB可以作为MySQL的替代品,MariaDB是开源的性能更加优秀。
\subsection{安装配置MySQL}
\href{https://downloads.mysql.com/archives/community/}{MySQL}官网网站可以下载各个版本,每种版本中有安装版和解压版两种,前者是直接安装不需要过多配置的傻瓜式安装,后者是解压后自己配置文件直接就可以使用的免安装版本,果断选择后者。这种方式在Windows下很方便,很直观自己知道每一步在做什么,其实也没有几部,不像安装包干了点啥都不知道,还有可能出现安装失败的情况。
我这里使用的是MySQL5.5版本,原因就是初次使用网上有篇\href{https://blog.csdn.net/qq_30061785/article/details/102504115?spm=1001.2014.3001.5506}{参考教程}讲的就是这个版本。

解压压缩包后,文件夹下有5个ini文件,这5个文件是针对用户五种情况下的配置文件,根据自己使用场景选择一个复制为my.ini文件,并修改部分设置主要是文件路径。
\begin{shell}
mysqld -install#安装mysql服务
mysqld -removel#卸载mysql服务
mysql -u root -p #登陆mysql,初次登录密码为空
#下边是在SQL环境下操作
update mysql.user set password=PASSWORD(‘123456’) where User=‘root’;
#设置root用户密码
flush privileges;
#立即启用修改
GRANT SELECT ON tanhecha.* TO 'tanhechauser'@'localhost' IDENTIFIED BY 'tanhechapassword';
#添加用户并授权select权限
\end{shell}
\subsection{MySQL常用操作}
\subsubsection{备份与恢复数据库}
我这里记录的备份是通过mysqldump工具将数据库备份到sql文本文件中,再登录数据库通过source命令将sql文件导入,整个恢复过程相当于按照备份的sql文件中sql语句操作数据库。

source命令还有一个很好的用法就是批量操作数据库时,先按要求编辑号sql文本文件,然后source导入sql文本文件,让数据库自动批量执行slq文本中的命令,我在碳核查过程中使用的就是这个法方法备份、裁减、拼接成为每月独立的数据库的,需要进一步写成系统自动执行该系列操作。
\begin{shell}
mysqldump -u root -p test > test.sql
#输入密码开始备份,备份test数据库数据至sql文本文件
mysql -u root -p
#输入密码登陆mysql
mysql>create database test;#创建test数据库
mysql>use test;#切换至test数据库
mysql>set names utf8;#设置编码格式为utf8
mysql>source test.sql;#将test.sql导入test数据库
\end{shell}
\subsubsection{批量操作数据库}
这里记录使用MySQL的sql文件批量操作数据库,主要是包括批量删除指定条件的数据、批量修改数据库内数据表的注释、清除主键自增值、合并两个数据表的值。使用的时命令行下登录数据库后通过source命令将写好的sql文件导入来批量操作数据库,在windows下使用MySQL Workbench执行时会提示错误,尤其时数据量大时,好像时数据库安全方面的而且速度慢,在命令行下没问题而且数据也可以。

source命令还有一个很好的用法就是批量操作数据库时,先按要求编辑号sql文本文件,然后source导入sql文本文件,让数据库自动批量执行slq文本中的命令,我在碳核查过程中使用的就是这个法方法备份、裁减、拼接成为每月独立的数据库的,需要进一步写成系统自动执行该系列操作。
\begin{shell}
mysqldump -u root -p test > test.sql
#输入密码开始备份,备份test数据库数据至sql文本文件
mysql -u root -p
#输入密码登陆mysql
mysql>create database test;#创建test数据库
mysql>use test;#切换至test数据库
mysql>set names utf8;#设置编码格式为utf8
mysql>source test.sql;#将test.sql导入test数据库
\end{shell}
\begin{shell}
delete from table;
#删除指定行数的数据
drop table if exists table;
#删除数据表所有数据和结构
truncate table table1,table2;
#删除数据表中所有数据但保留表结构,重置自增字段
ALTER TABLE course RENAME TO coursecopy;
#将course表重命名为coursecopy
CREATE TABLE course LIKE coursecopy;
#—将coursecopy表复制结构并创建名为course的新表
INSERT INTO course SELECT * FROM coursecop;
#将coursecopy表数据复制到course表
\end{shell}
\subsubsection{数据库同步}
单向同步
\section{MariaDB}
讲解MySQL数据库时提到了MariaDB数据库,在Linux系统下还是安装使用MariaDB方便。MariaDB使用基本与MySQL相同,在这里记录在MariaDB下的操作。
\subsection{安装配置MariaDB}
下面以在\href{https://blog.csdn.net/erliujian111/article/details/135732876?spm=1001.2014.3001.5506}{Debian下安装MariaDB}进行记录。
\begin{shell}
sudo apt install mariadb-server mariadb-client
#安装MariaDB数据库
sudo apt purge mariadb-server#卸载MariaDB
sudo rm -rf /var/lib/mysql/#彻底删除MariaDB
sudo systemctl start mariadb #启动MariaDB数据库
sudo systemctl enable mariadb #设置MariaDB自动启动
sudo systemctl status mariadb #查看MariaDB状态
sudo mariadb-secure-installation #初始化MariaDB数据库
#包括设置root密码(默认为空),删除匿名用户等
#有一项是禁止root用户远程登陆,需要根据个人需求选择
mysql -u root -p#登陆MariaDB数据库
\end{shell}
\subsection{触发器}
触发器是一种特殊的存储过程,它在插入,删除或修改特定表中的数据时触发执行,我是在更新数据表的同时用更新的数据计算更新另一张表内数据。
触发器必须是在指定数据库内的,就是得先use database然后在在database上创建触发器。
\begin{shell}
use buff;#进入buff数据库
#创建触发器
create trigger update_trigger2 after update on ua for each row update boler set value = NEW.value where kks = left(OLD.kks,12);
show triggers\G;#查看所有触发器
show create trigger update_trigger2\G;#查看指定触发器
drop trigger update_trigger2;#删除触发器
\end{shell}
\subsection{事物}

\section{PHP}
PHP全称Hypertext Preprocessor,中文名:“超文本预处理器”是一种通用开源脚本语言。语法吸收了C语言、java和Perl的特点,利于学习,使用广泛,主要适用于web开发领域。用PHP做出的动态页面与其他的编程语言相比,PHP是将程序嵌入到HTML(标准通用标记语言下的一个应用)文档中去执行,执行效率比完全生成HTML标记的CGI要高许多;PHP还可以执行编译后代码,编译可以达到加密和优化代码运行,使代码运行更快。
\subsection{phpmyadmin的使用}
搭建LAMP平台后将解压后的phpmyadmin文件放到网站根目录下/var/www/html/即可访问数据库。
\subsection{PHP二维码生成}
PHP QRCode全称PHP Quick Response Code,是一个在PHP平台下生成二维码的开源库。\href{https://phpqrcode.sourceforge.net/}{官网}下有很多使用它生成二维码的例子,这里记录两种我使用过的情况。
\subsubsection{静态网页直接显示二维码}
这种情况是在生成网页的时候直接将二维码打印在网页上,用官网的话叫outputs image directly into browser, as PNG stream。
\begin{shell}
 <?php

    include('../lib/full/qrlib.php');
        
    $param = $_GET['id']; // remember to sanitize that - it is user input!
    
    // we need to be sure ours script does not output anything!!!
    // otherwise it will break up PNG binary!
    
    ob_start("callback");
    
    // here DB request or some processing
    
    // end of processing here
    $debugLog = ob_get_contents();
    ob_end_clean();
    
    // outputs image directly into browser, as PNG stream
    QRcode::png($codeText); 
?>
\end{shell}
\subsubsection{AjaX动态更新二维码}
这种情况是在点击不同元素动态生成二维码打印在网页上,动态更新网页就需要用到AjaX技术,但是AjaX返回数据要求是字符串不能是二进制流,所以需要先\href{https://blog.csdn.net/LJFPHP/article/details/79111844}{将PNG stream转换成字符串}返回到前端后再显示。
\begin{shell}
<?php
include_once("./phpqrcode/qrlib.php");
$param = $_GET['id'];
$codeText = $param;
ob_start();
QRcode::png($codeText);
$debugLog = base64_encode(ob_get_contents());
ob_end_clean();
echo $debugLog;
?>
#前端javascript中
document.getElementById('src').src=
'data:image/png;base64,'+this.responseText;
\end{shell}
\subsection{php生成PDF文件}
加载离线文件就可以生成PDF文件
\section{Python}
Python是一种解释型语言,它的优点就是有丰富的库,使得编成变的简单,我是在实现OPC DA的DCOM编成和OPC UA服务器时开始使用它,库确实很强大,很容易就完成了想要的功能,所以决定学习该语言。
\subsection{pip}
pip是用来安装和更新Python库的,由于pip默认的源是国外的,安装更新很慢导致安装库时经常超时失败,所以安装pip后需要将源换成阿里云 http://mirrors.aliyun.com/pypi/simple/

国内镜像源:
\begin{enumerate}
	\item 清华大学 https://pypi.tuna.tsinghua.edu.cn/simple/
	\item 中国科技大学 https://pypi.mirrors.ustc.edu.cn/simple/
	\item 豆瓣 http://pypi.douban.com/simple
	\item Python官方 https://pypi.python.org/simple/
	\item v2ex http://pypi.v2ex.com/simple/
	\item 中国科学院 http://pypi.mirrors.opencas.cn/simple/
	\item 中国科学技术大学 [http://pypi.mirrors.ustc.edu.cn/simple/
	\item 华中理工大学:http://pypi.hustunique.com/
	\item 山东理工大学:http://pypi.sdutlinux.org/
\end{enumerate}
\begin{shell}
#首先省级最新pip版本
python -m pip install --upgrade pip
#单次安装使用国内镜像源以tensorflow库为例
pip install tensorflow -i http://mirrors.aliyun.com/pypi/simple/ --trusted-host mirrors.aliyun.com#安装最新版本
pip install tensorflow==2.13.0 -i http://mirrors.aliyun.com/pypi/simple/ --trusted-host mirrors.aliyun.com#安装指定版本
#永久设置国内镜像源,两项都要设置否则报错
pip config set global.index-url http://mirrors.aliyun.com/pypi/simple/#设置index-url
pip config set global.trusted-host mirrors.aliyun.com#设置trusted-host
#以上两句命名相当于在~/.config/目录下创建pip配置文件
cd ~/.config/
mkdir pip
vim pip.conf

[global]
index-url = http://mirrors.aliyun.com/pypi/simple/
trusted-host = mirrors.aliyun.com
\end{shell}
\subsection{Python虚拟环境}
Python虚拟环境可以创建一个独立的环境,用于安装不同项目所需的特定Python包和依赖项,甚至是不同版本的Python环境,这个功能对于需要不同版本Python同时安装和Python环境迁移非常有用。下面以Linux和Windows下安装分别记录

\href{https://blog.csdn.net/qq_34444097/article/details/142733302?spm=1001.2014.3001.5506}{Debian系统安装Python虚拟环境}
\begin{shell}
sudo apt-get install python3-venv#创建虚拟环境的env模块
sudo apt-get install python3-pip#Python包管理工具
python3 -m venv myvenv#创建名称为myvenv的虚拟环境
source myvenv/bin/activate#激活myvenv虚拟环境
deactivate#推出myvenv虚拟环境
\end{shell}
\href{https://blog.csdn.net/cl_kleiber0802/article/details/142006096?spm=1001.2014.3001.5506}{Windows系统安装Python虚拟环境}
\begin{shell}
pip install virtualenv #创建虚拟环境的env模块
virtualenv -p D:\Python\Python312\Python.exe myvenv
#创建名称为myvenv的虚拟环境
myvenv/Scripts/activate#激活myvenv虚拟环境
myvenv/Scripts/deactivate#退出myvenv虚拟环境
\end{shell}
\subsection{爬虫}
爬虫不能一直研究只能随缘,属于奇淫技巧,不能耗费过多时间。
\subsubsection{百度翻译}
这个是第一个学习的第一个爬虫项目-\href{https://blog.csdn.net/m0_58378947/article/details/123905684?spm=1001.2014.3001.5506}{百度翻译的单词爬虫},结果是运行爬虫后直接输入单词或汉字可以直接翻译出结果来。
\begin{shell}
import requests

def spider(url,headers,data):

    response = requests.post(url=url, headers=headers, data=data).json()  # 对目标url发起post请求
    for key in response['data'][0]:
        print(key,response['data'][0][key])

def main():

    url = 'https://fanyi.baidu.com/sug'  #需要请求的url
    headers = {  #进行UA伪装
        'User-Agent':'Mozilla/5.0 (Windows NT 10.0; Win64; x64) AppleWebKit/537.36 (KHTML, like Gecko) Chrome/98.0.4758.102 Safari/537.36 Edg/98.0.1108.56'
    }
    while True:  #使程序进入死循环
        kw = input("输入需要查询的单词:")
        data = {     #post请求携带的参数
            'kw':kw
        }
        spider(url=url,headers=headers,data=data)  #调用自定义函数spider

main()
\end{shell}
\subsubsection{爬取在线视频}
\href{https://blog.csdn.net/u011223449/article/details/136110053}{这是一个爬取在线视频项目},爬取喜马拉雅音频的失败了,因为path都为空,使用了最新的反爬手段

在爬去视频时明白原来网络视频都是一小段一小段,一个40分钟电视剧分散成有700多个小视频,插入广告非常方便,去除广告也非常方便哦!
\subsection{自动化办公}
\href{https://blog.csdn.net/weixin_41261833/article/details/106028038?spm=1001.2014.3001.5506}{Python的自动化办公}可是比较有名的,所以我也试验了下,确实很强大。我的想法是借助Python自动生成Word和Excel文件功能实现服务器端存储数据源文件,需要的时候自动生成需要的格式文件。
\begin{shell}
#这里只简单记录下需要安装的模块,详细的参考网页资料
#值得注意的是只支持指定格式的文件并非所有excel和word文件
pip install openpyxl#安装xlsx文件的模块
pip install python-docx#安装docx文件的模块,导入时是import docx
\end{shell}

\subsection{百度网盘}
\href{https://blog.csdn.net/u010751000/article/details/130191192?ops_request_misc=%257B%2522request%255Fid%2522%253A%25221986353ffb0d0d65f2126a7f2b677eda%2522%252C%2522scm%2522%253A%252220140713.130102334..%2522%257D&request_id=1986353ffb0d0d65f2126a7f2b677eda&biz_id=0&utm_medium=distribute.pc_search_result.none-task-blog-2~all~sobaiduend~default-2-130191192-null-null.142^v101^pc_search_result_base7&utm_term=python%E7%99%BE%E5%BA%A6%E7%BD%91%E7%9B%98&spm=1018.2226.3001.4187}{Python使用bypy模块可以操作百度网盘},包括显示文件列表、同步目录、文件上传,不过只支持/apps/bypy目录。

借助该模块可以搭建自己的1T云备份和同步系统。
\begin{shell}
pip install bypy#安装模块
bypy list #显示云盘根目录下文件列表
#首次操作点击终端上方的蓝色链接,复制授权码并回车完成授权
\end{shell}
\begin{shell}
from bypy import ByPy
bp = ByPy()
print(bp.list())
bp.upload(r"localfile","disfile")#上传文件
bp.download(r"localfile","disfile")#下载文件
bp.syncup(r"localdir","disdir")#上传文件夹
bp.syncdown("remotedir",r"localdir")#下载文件夹
\end{shell}
\subsection{Django}
\section{ImageMagick}
ImageMagick是一套功能强大、稳定而且免费的工具集和开发包,它可以单独使用来处理图片而且还可以在PHP调用来自动处理图片。

webp 是一种新的图像格式,用于web项目,可以大大提高网站访问速度。同样的分辨率,大小比 jpg、png 小 25\% 以上。我的网站上上传的图片就是在PHP中自动将图片转换为webp格式图片,转换后的图片大小小了不止25\%。
\subsection{ImageMagick安装}
记录在\href{https://blog.csdn.net/Wufjsjjx/article/details/135401894?spm=1001.2014.3001.5506}{Debian12下安装ImageMagick}。
\begin{shell}
#安装依赖库
sudo apt-get install build-essential 
sudo apt-get install libjpeg-dev libpng-dev libtiff-dev libgif-dev libwebp-dev 
sudo apt-get install webp
#github下载ImageMagick源文件进行编译安装
tar xf ImageMagick-7.1.1-47.tar.gz
cd ImageMagick-1.1-47
./configure
sudo make install
magick --version#安装完成后,检查版本信息
#如果出现下面错误
magick: error while loading shared libraries: libMagickCore-7.Q16HDRI.so.10: cannot open shared object file: No such file or directory 
#执行以下命令
echo "/usr/local/lib" >>sudo  /etc/ld.so.conf
sudo ldconfig
magick --version#再次检查版本信息
\end{shell}
\subsection{ImageMagick命令}
\begin{shell}
#将png图片格式转换为webp格式
/usr/local/imagemagick/bin/convert file1.png file3.webp
#将图片裁减为指定大小
/usr/local/imagemagick/bin/convert -sample 768x1024 file3 file3
#在图片上添加水印
/usr/local/imagemagick/bin/convert -fill red -pointsize 60 -draw 'text 300,80 "dklovelich"' file3 file3
\end{shell}
\subsection{PHP下安装ImageMagick扩展}
\href{https://blog.csdn.net/JineD/article/details/108318106?spm=1001.2014.3001.5506}{安装PHP下的ImageMagick扩展模块}后就可以在PHP中使用ImageMagick处理图片了。但是安装未能成功。

\section{Nginx}
\section{Apache}
\subsection{Apache安装配置}
\subsection{apache2根目录修改}
Apache2的根目录默认为/var/www,如果需要修改到自定义地址,涉及两个关键配置文件的调整。 
\begin{enumerate}
	\item 修改/etc/apache2/apache2.conf,把文件里面的/var/www改成你的目标地址
	\item 修改/etc/apache2/sites-enabled/000-default.conf,把文件里面的/var/www改成你的目标地址
\end{enumerate}
\section{Docker}
docker是一种虚拟化技术,和之前接触比较类似的就是虚拟机,区别就是虚拟机需要安装操作系统然后安装要使用的应用程序,消耗资源比较多,docker就是在本机上构建一个独立的虚拟环境来运行应用程序比较节省资源。
\href{https://blog.csdn.net/m0_61503020/article/details/125456520?spm=1001.2014.3001.5506}{Docker是一个开源的应用容器引擎},让开发者可以打包他们的应用以及依赖包到一个可抑制的容器中,然后发布到任何流行的Linux机器上,也可以实现虚拟化。容器完全使用沙盒机制,相互之间不会存在任何接口。几乎没有性能开销,可以很容易的在机器和数据中心运行。最重要的是,他们不依赖于任何语言、框架或者包装系统。
\subsection{Docker安装部署}
在CSDN上找了很多篇帖子也没有在Debian11上安装成功,后来在\href{https://www.cnblogs.com/jason-zhao/p/18150268}{博客园上找到一篇安装成功}了。安装成功后无法正常拉取镜像,后来\href{https://blog.csdn.net/weixin_39764056/article/details/145042307?spm=1001.2014.3001.5506}{修改镜像}后拉取成功了。
\begin{shell}
sudo apt-get update#更新软件包索引
sudo apt-get install apt-transport-https ca-certificates curl gnupg2 software-properties-common#安装必要的软件包
curl -fsSL https://mirrors.aliyun.com/docker-ce/linux/debian/gpg | sudo apt-key add -#添加阿里云的Docker官方GPG密钥
sudo add-apt-repository "deb [arch=amd64] https://mirrors.aliyun.com/docker-ce/linux/debian $(lsb_release -cs) stable"#添加Docker仓库地址源
sudo apt-get update
sudo apt-get install docker-ce docker-ce-cli containerd.io#安装Docker CE
sudo systemctl status docker
sudo usermod -aG docker $USER#设置非root用户也能运行Docker
#更新Docker源,否则无法正常拉取镜像
cd /etc/docker/
vim daemon.json#没有就创建该文件
{
 "registry-mirrors": ["https://docker.registry.cyou",
"https://docker-cf.registry.cyou",
"https://dockercf.jsdelivr.fyi",
"https://docker.jsdelivr.fyi",
"https://dockertest.jsdelivr.fyi",
"https://mirror.aliyuncs.com",
"https://dockerproxy.com",
"https://mirror.baidubce.com",
"https://docker.m.daocloud.io",
"https://docker.nju.edu.cn",
"https://docker.mirrors.sjtug.sjtu.edu.cn",
"https://docker.mirrors.ustc.edu.cn",
"https://mirror.iscas.ac.cn",
"https://docker.rainbond.cc"]
}
systemctl daemon-reload
systemctl restart docker

\end{shell}
\subsection{Docker使用}
\begin{shell}
docker pull mysql:5.7#拉取mysql的镜像
docker images#查看本地镜像
\end{shell}
\section{Grafana}
Grafana是一款开源的数据可视化工具,主要用于大规模指标数据的可视化展示。这是我在搭建Pyscada平台时接触到的一款软件,它可以用于历史曲线显示和监控画面显示,很厉害。可以在\href{https://grafana.com/grafana/download?platform=windows}{官网下载}该软件的linux和windows版本。
\subsection{安装配置Grafana}
Grafana的安装比较简单,在windows下甚至不需要安装直接解压就可以使用了。
\begin{shell}
#Debian下安装
wget https://dl.grafana.com/oss/release/grafana_10.2.3_amd64.deb
sudo dpkg -i grafana_10.2.3_amd64.deb
sudo systemctl start grafana-server
sudo systemctl enable grafana-server
\end{shell}
\subsubsection{使用Grafana}
Grafana的使用比较复杂,安装完成后在浏览器使用http://localhost:3000就可以登录页面,首次登录用户名和密码都是admin,登录后需要修改密码。

配置数据源,需要首先增加用户并授权指定数据库的select权限才能添加成功,否则会添加失败,用root用户无法设置权限不足。





\section{GoldenDict}
\href{https://blog.csdn.net/networkhunter/article/details/117127021?spm=1001.2014.3001.5506}{GoldenDict}是一款免费的linux预装的字典,它可以导入下载好的字典离线查询,也可以设置好网址后在线查询。.
\section{OpenTTD}
OpenTTD是一款运输模拟游戏,安装很简单,直接从\href{https://www.openttd.org/}{官网}上下载解压即可,无需安装,但是它的\href{https://wiki.openttd.org/zh/Main Page}{玩法}门槛很高而且容易上头。

西文中一般采用上述的斜体强调方式而不是粗体,例如在说明书的时候可能就会使用以上命令。关于字体更多内容参考字体这一节。

\subsection{下划线与删除线}
\LaTeX\ 原生提供的\latexline{underline}命令简直烂的可以,建议你使用\pkg{ulem}宏包下的\texttt{uline}命令代替,它还支持换行文本。\pkg{ulem}宏包还提供了一些实用命令:

		%!TEX root = ../dk_log.tex
\chapter{项目平台}
这一章主要记录一些小的项目搭建,这些小的项目需要安装操作系统和一些软件组合并经过一些配置搭建起一个工作平台,然后在这个平台下做一些事情。比如LAMP、SIS系统、Smart PLC培训。
\section{我的个人网站}
我的个人网站是我一直在学习中搭建的一个网站,其中使用了很多技术,在此记录下。
\subsection{php生成PDF文件}
\subsection{远程操作latex生成pdf文件}
\subsection{远程操作python生成doc文件}
\subsection{动态更新svg画面}
\subsection{Dygraph生成历史曲线}
\section{LAMP环境}
\href{https://blog.csdn.net/sj349781478/article/details/84224440?ops_request_misc=%257B%2522request%255Fid%2522%253A%252249718a9a6d88ebb92a001e305b31fd12%2522%252C%2522scm%2522%253A%252220140713.130102334..%2522%257D&request_id=49718a9a6d88ebb92a001e305b31fd12&biz_id=0&utm_medium=distribute.pc_search_result.none-task-blog-2~all~top_positive~default-1-84224440-null-null.142^v101^pc_search_result_base7&utm_term=lamp%E6%9E%B6%E6%9E%84&spm=1018.2226.3001.4187}{LAMP(Linux Apache Mysql Php)}是指一组通常一起使用来运行动态网站或者服务器的自由软件名称首字母缩写;Linux系统下Apache+MySQL+PHP这种网站服务器架构, LAMP环境主要是给WEB端应用程序(各种类型的网站项目),提供了一个部署安装和使用的平台。
\begin{description}
	\item[L:]Linux操作系统,提供了项目部署时所需要的操作系统环境
	\item[A:]Apache服务器:WEB应用程序的服务器,提供软件源文件的存放地,提供了程序访问时所需要的端口(接口)
	\item[M:]MySQL数据库,提供项目或者程序在使用时数据的存储与解析的工作
	\item[P:]PHP/Python开发语言,提供软件或者项目程序部署时所需要的开发环境的支持
\end{description}

只要把这四个软件安装完成,就形成了LAMP环境,环境有了之后,只需要把WEB应用程序对应的源文件,部署在apache服务器上即可,这样用户就可以直接访问该网站。
\subsection{LAMP环境搭建}
\href{https://blog.csdn.net/weixin_35886269/article/details/116039239?spm=1001.2014.3001.5506}{LAMP环境搭建}的指定就是在Linux系统上安装这三款软件然后配置相关文件后协同工作来建设一个WEB服务器。这几款软件的安装前边已经记录过了,这里主要记录软件之间相互配置调用和搭建过程中注意事项。

主要是PHP版本必须是PHP7,而且随着linux系统升级PHP7安装过程中会有各种问题,\href{https://blog.csdn.net/weinsheimer/article/details/131855546?spm=1001.2014.3001.5506}{这里记录在Debian12环境下的注意事项},安装完成后记得重启apache2服务,不然部分php模块无法正常加载。
\section{LNMP}
\href{https://blog.csdn.net/G_D0120/article/details/136338594?ops_request_misc=%257B%2522request%255Fid%2522%253A%2522845af08f81d51851cea0a81b281138f3%2522%252C%2522scm%2522%253A%252220140713.130102334..%2522%257D&request_id=845af08f81d51851cea0a81b281138f3&biz_id=0&utm_medium=distribute.pc_search_result.none-task-blog-2~all~sobaiduend~default-2-136338594-null-null.142^v101^pc_search_result_base7&utm_term=lNmp&spm=1018.2226.3001.4187}{LNMP(Linux Nginx Mysql Php)}是指一组通常一起使用来运行动态网站或者服务器的自由软件名称首字母缩写;Linux系统下Nginx+MySQL+PHP这种网站服务器架构, LNMP环境主要是给WEB端应用程序(各种类型的网站项目),提供了一个部署安装和使用的平台。
\subsection{LNMP环境搭建}
我最初接触的是LAMP环境,在搭建Pysacada环境时因为需要才开始接触LNMP环境,相比较而言LNMP就是把WEB服务由Apache更换为Nginx。
\section{PyScada}
	pyScada是一款基于Python的开源SCADA系统,可以在它的\href{https://gitcode.com/gh_mirrors/py/PyScada/?utm_source=artical_gitcode&index=top&type=card&webUrl&isLogin=1}{开源项目}和\href{https://pyscada.readthedocs.io/en/main/}{官方网站}上了解和下载该系统。
\subsection{PyScadaa安装部署}

	现将\href{https://blog.csdn.net/XKPP023/article/details/140079389?spm=1001.2014.3001.5506}{PyScada系统安装记录}记录在此,值得注意的是PyScada系统是以NGINX作为WEB服务器的,所以得先停运Apache2服务器,否则无法正常启动NGINX,因为它两使用同一端口,不过可以重新配置端口让两种服务器同时运行,甚至经过配置还可以让两种服务器配合运行各取所长,这也是我下一个目标。

官网安装指导部分有这么一句话This installation guide covers the installation of PyScada for Debian 10/11 , Raspberry Pi OS based Linux systems using MariaDB as Database, Gunicorn as WSGI HTTP Server and nginx as HTTP Server。
\begin{enumerate}
	\item 获取安装包,值得注意的是解压缩需要使用命令行工具uzip,否则安装最后后出现文件权限问题
	\item 停运Apache2服务,最好禁止Apache2开机启动否则下次开机,PyScada还是不能正常启动
	\item 安装MariaDB并初始化完成,安装过程中要使用数据库并创建专门的数据库
	\item 可以选择本机或docker安装,我们选着本机,因为docker还未安装成功...
	\item 安装Python是必须的,设置pip国内源否则安装过程中下载速度过慢甚至报错失败
	\item 不需要安装虚拟环境,因为安装第一步就是创建PyScada文件夹并创建.env虚拟环境
\end{enumerate}
\begin{shell}
#在Debian11下安装PyScada记录
sudo apt install wget
wget https://github.com/pyscada/PyScada/archive/refs/heads/main.zip -O PyScada-main.zip
sudo apt install unzip
unzip ./PyScada-main.zip
rm ./PyScada-main.zip
cd PyScada-main

sudo ./install.sh


\end{shell}
\subsection{PyScada安装插件}
刚安装完成的PyScada中Devices中驱动只有一个generic,我们需要安装我们使用的OPCUA驱动。官网的例子是安装Modbus驱动,我们照猫画虎安装OPCUA驱动,\href{https://github.com/pyscada/}{在这里可以看到和下载所需要的驱动}。
\begin{shell}
#安装PyScada-OPCUA驱动
sudo apt install git
cd /home/pyscada
sudo -u pyscada git clone https://github.com/pyscada/PyScada-OPCUA.git
cd PyScada-OPCUA

# 激活PyScada虚拟环境
source /home/pyscada/.venv/bin/activate
#安装驱动 
sudo -u pyscada -E env PATH=${PATH} pip3 install .#特别注意这个地方有个点表示当前目录
# run migrations
sudo -u pyscada -E env PATH=${PATH} python3 /var/www/pyscada/PyScadaServer/manage.py migrate
# copy static files
sudo -u pyscada -E env PATH=${PATH} python3 /var/www/pyscada/PyScadaServer/manage.py collectstatic --no-input
#重新启动gunicorn和PyScada服务
sudo systemctl restart gunicorn pyscada

pip3 list | grep cada#查看驱动是否安装成功
sudo -u pyscada -E env PATH=\${PATH} pip3 uninstall yourPlugin#卸载驱动
\end{shell}
\subsection{PyScadaa使用}
在本机浏览器输入127.0.0.1会出现登录界面,输入你安装过程中创建的账号密码就可以进入系统,在安装成功后也会提示这一步并显示登录账号密码。
点击右上角Admin后出现后台管理界面
\subsubsection{编写一个OPCUA服务器}
这里有个插入一个小插曲,因为后边需连接OPCUA服务器读取变量进行测试,如果安装软件或者是从其他主机联机读取的话比较费劲,这里使用Python编写一个简单的OPCUA服务器,很方便的,代码如下
\begin{shell}
import sys
sys.path.insert(0, "..")
import time
from opcua import ua, Server
if __name__ == "__main__":
    # setup our server
    server = Server()
    #server.set_endpoint("opc.tcp://127.0.0.1:4840/freeopcua/server/")
    server.set_endpoint("opc.tcp://192.168.1.5:4840/freeopcua/server/")
    # setup our own namespace, not really necessary but should as spec
    uri = "http://automan.freeopcua.github.io"
    idx = server.register_namespace(uri)
    # get Objects node, this is where we should put our nodes
    objects = server.get_objects_node()
    # populating our address space
    myobj = objects.add_object(idx, "MyObject")
    myvar = myobj.add_variable(idx, "MyVariable", 6.7)
    myvar.set_writable()    # Set MyVariable to be writable by clients
    # starting!
    server.start()
    try:
        count = 0
        while True:
            time.sleep(1)
            count += 0.1
            myvar.set_value(count)
    finally:
        #close connection, remove subcsriptions, etc
        server.stop()
\end{shell}
\subsubsection{添加第一个OPCUA变量}
首先添加Devie,然后添加变量,选择刚刚添加的deveic,重要的是变量最下方的s和n需要按照UA服务器变量的ID输入。
\section{SIS系统}
	搭建SIS系统大体分为几个阶段:
	第一阶段:通过OPC DA从DCS系统读取数据;
	第二阶段:将读取的数据添加至OPC UA服务器,供外界读取;
	第三阶段:SIS软件从OPC UA服务器读取数据并显示;
	第四阶段:SIS软件将从OPC UA服务器读取数据存储形成历史数据并提供查询功能;
	第一款软件无疑就是VIM了,那第二款肯定是Latex了,这玩意就是用他两鼓捣出来的。	编辑器的配置大概是需要讲解一下的,毕竟对于初学者来说是很头疼的事情。本手册就以\TeX\ studio为例进行配置。首先你应该安装一个\TeX{} Live,他是完全免费的,网址:\url{http://tug.org/texlive/}。
\subsection{KepServer采集DCS数据}
\subsubsection{通过KepServer采集DCS数据}
\subsubsection{通过KepServer采集DCS数据并存储至数据库}
KepServer软件采集数据
\subsection{Dcom编成采集DCS数据}
注册OPCDA.dll
\subsubsection{Python下的Dcom编成}
\subsubsection{单向网闸配置}
我接触单向网闸是在2024年9月份,记得当时是公司SIS系统按照等保测评要求在DCS系统和SIS系统中间使用了电力系统专用的单向隔离网闸,它的特点是数据只能从内网向外网正常传输,外网向内网传输数据只能是单Bit,所以就导致外网安装的软件KepWare无法正常发送连接请求到DCS系统OPC DA服务器,最后还是我使用Python语言DCOM编成从DCS OPC DA服务器读取出数据。
\subsubsection{平台搭建}
在Win7 SP1 64位操作系统安装NR软件后,配置主机IP地址为,用网线连接电脑和网闸内网管理口,ping测试正常,将2和1转换器插到内网console口,并在2和1转换器插入其中一个操作员秘匙。
\subsubsection{设备激活}
初次登陆需要创建系统管理员,账户:rekongroot密码:rekong1314root! pin码:Nari6702
初始化用户信息,ukey账户信息被删除,生成ukey证书请求(内容随便填写),使用证书签发系统签发后,将签发证书上传装置。出现登录界面用系统管理员账户登录后开始生成激活申请文件,其中最终用户名称必须是购买合同中的单位名称,按照规定格式发送邮件并将激活申请文件作为附件上传,等待邮件回复。


虽然它体积较大,但是却是最一劳永逸、最不需要花时间去配置的方法,同时它大概也是功能支持最强的\LaTeX\ 发行版。

打开\TeX\ Studio后,选择选项$\rightarrow$设置\TeX\ Studio $\rightarrow$ 构建 $\rightarrow$ 默认编译器,选择\xelatex{}。这主要是基于中文文档编译的考虑,同时\xelatex 也能很好的编译英文文档。我建议始终使用它作为默认编译器。


之后你可以在窗口输入一篇小文档,并保存为tex扩展名的文件进行测试:
\begin{latex}
\documentclass{ctexart}
\begin{document}
Hello,world!
你好,世界!
\end{document}
\end{latex}
点击编译按钮生成,F7查看。生成pdf在你的tex文件保存目录中。具体各行的含义我们后在后文介绍。

\section{Smart PLC培训}
这个主要是针对2024年厂里的一套氨水控制系统做培训时搭建的过程记录
Windows 7旗舰版Service Pack 1 64位操作系统
STEP7-MicroWIN-SMAT-V2.4
WinCC V7.3
PC\_ACCESS\_V2.3
\subsection{VIM插件管理Plum-vim}
\LaTeX\ 中的\co{命令}通常是由一个反斜杠加上命令名称,再加上花括号内的参数构成的(有的命令不带参数,例如:\latexline{TeX})。
\begin{latex}
\documentclass{ctexart}
\end{latex}
如果有一些选项是备选的,那么通常会在花括号前用方括号标出。比如:
\begin{latex}
\documentclass[a4paper]{ctexart}
\end{latex}
还有一种重要指令叫做\co{环境}。它被定义与控制命令\latexline{begin\{environment\}}\\和\latexline{end\{environment\}}间的内容。比如:%不加入强制换行会溢出?
\begin{latex}
\begin{document}
...内容...
\end{document}
\end{latex}
环境如果有备选参数,只需要写在\latexline{begin[...]\{name\}}这里就行。

注意:不带花括号的命令后面如果想打印空格,请加上{\color{cyan}{一对内部为空的花括号}}再键入空格。否则空格会被忽略。例如:\verb+\LaTex{} Studio+。
\subsection{VIM常用操作}
\LaTeX\ 中有许多字符有着特殊的含义,在你生成文档时不会直接打印。例如每个命令的第一个字符:反斜杠。单独输入一个反斜杠在你的行文中不会有任何帮助,甚至可能产生错误。\LaTeX\ 中的保留字符有:
\begin{center}
	\texttt{\# \$ \% \^ \& \_ \{ \} \char92}
\end{center}

它们的作用分别是:
%\begin{description}[font=\bfseries\ttfamily,itemsep=0pt,parsep=0pt,topsep=0pt,partopsep=0pt]
\begin{para}
\item[\#{}:]自定义命令时,用于标明参数序号。
\item[\S{}:]数学环境命令符。
%\end{description}
\end{para}

以上除了反斜杠外,均能用前加反斜杠的形式输出。即你只需要键入:
\begin{center}
\verb|\# \$ \% \^ \& \_ \{ \}|
\end{center}

唯独反斜杠的输出比较头痛,你可以尝试:
\begin{codeshow}
$\backslash$ \textbackslash
\texttt{\char92}
\end{codeshow}

其中命令\latexline{char[num]}是一个特殊的命令
\begin{verbatim}
\texttt{\char`~}%输出一个波浪线
\end{verbatim}

\subsection{导言区}
任何一份\LaTeX{}文档都应该包含以下结构:
\begin{latex}
\documentclass['\itshape options']{doc-class}%没有斜体option
\begin{document}
...
\end{document}
\end{latex}
其中,在语句\latexline{begin\{document\}}之前的内容成为\co{导言区}。导言区可以留空,也可以进行一些、文档的准备操作。你可以粗浅地理解为:\RED{导演区即模板定义}。\\

文档类的参数doc-class和可选选项{\textit{options}}有取值:%引用图表失败
%\begin{table}[!htb]原文中有htb,现在无法正确编译
	\begin{table}[!htb]
	\centering
	\caption{文档类和选项}
	\label{tab:documentclass}
	\begin{tabular}{p{5em} @{\ -\ } p{24em}}
		\hline
		\multicolumn{2}{l}{\bfseries doc-class文档类\footnotemark}\\
		\hline
		article & 科学期刊,演示文稿,短报告,邀请函。\\
		proc	& 基于article的会议论文集。\\
		report	& 多章节的长报告、博士论文、短篇书。\\
		book	& 书籍。\\
		slides	& 幻灯片,使用了大号Scans Serif字体。\\
		\hline
		\multicolumn{2}{l}{\bfseries\itshape options}\\
		\hline
		字体	&默认10pt,可选11pt和12pt。\\
		\hline
	\end{tabular}
\end{table}

在本文中,多数的文档类提及的均为report/book类。如果有article类将会特别指明。其余的文档类不与说明。本手册排版即使用了report类。

在导言区最常见的是\co{宏包}的加载工作,命令形式如:\latexline{usepackage\{package\}}。通俗地讲,宏包是指一系列已经制作好的功能``模块'',在你需要使用一些原生\LaTeX\ 不带有的功能时,只需要调用这些宏包就可以了。比如文本的代码就是利用\pkg{listings}宏包实现的。

宏包的具体使用将参在个部分内容说明中进行讲解。如果你想学习一个宏包的使用,按Win+R组合键呼出运行对话框,输入texdoc加上宏包名称即可打开宏包帮助pdf文档。例如:\verb|texdoc xeCJK|。
	\footnotetext{此外还有\pkg{beamer}宏包定义的beamer文档类,常用于创建幻灯片。}

\subsection{错误的排查}
	\label{subsec:debug}
	在编辑器界面上,下方的日志是显示编译过程的地方。在你编译通过后,会出现这样的字样:
\begin{feai}
	\item {\bfseries{Errors错误}}:严重的错误。一般地,编译若通过了,该项是零。
	\item {\bfseries{Warnings警告}}:一些不影响生成文档的瑕疵。
	\item {\bfseries{Bad Boxes坏箱}\footnote{Box是\LaTeX{}中的一个特殊概念,具体将在进行讲解。}}:指排版中出现的长度问题,比如长度超出(Overfull)等。后面的Badness表示错误的严重程度,程度越高数值越大。这类问题需要检查,排除Badness高的选项。\marginpar{\textcolor{red!70}{此处注解在后续章节,目前未链接}}
\end{feai}

	你可以向上翻越日志记录(即.log文件),来找到Warning开头的记录,或者Overfull/ Underfull开头的记录。这些记录会指出你的问题出在哪一行(比如line 1-2)或者在pdf的哪一页(比如active[12]。注意,这个12表示计数器技术页码,而不是文件打印出来的真实页数)。此外你还需要了解:
\begin{feai}
\item 值得指出的是,由于\LaTeX{}的编译原理(第一次生成aux文件,第二次再引用它),目录想要合理显示{\bfseries{需要连续编译两次}}。在连续编译两此后,你会发现一些Warnings会在第二次编译后消失。在\TeX\ Studio中,你可以只单击一次“构建并查看”,他会检测到文章的变化并自动决定是否需要编译两次。
\item 对于大型文档,寻找行号十分痛苦。你需要学会合理地拆分tex文件,参阅内容。%未引用3.13节内容
\end{feai}

	这里也推荐宏包\pkg{syntonly},在导言区加入它支持的\latexline{syntaxonly}命令,会只排查语法错误而不生成任何文档,这可以使你更快地编译。不过他似乎不太稳定,例如本文档可以正常编译,但是使用该命令时则会出错。
\subsection{文件输出}
	\LaTeX{}的输出一般推荐pdf格式,有\LaTeX\ 直接生成dvi的方法并不推荐。

你在tex文档的文件夹下可能看到的其他文件类型:
\begin{tabbing}
	.sty{\hspace{2em}}\=宏包文件\\%第一行必须此格式否则会编译出错
	.cls	\> 文档类文件。\\
	.aux	\> 用于存储交叉引用信息的文件。\\
	.out	\> 宏包生成的pdf书签记录。
\end{tabbing}
\section{latex}
	英文符号一般用于数学$|<>+=$一般用于数学环境中,如果在文本中使用,请在它们两侧加上“\$”。如果你在\LaTeX\ 中直接输入大于、小于号而不是把它们、放在数学环境中,它们并不会被正确打印。你应该使用\latexline{textgreater},\latexline{texless}命令。

	在部分科技文章中,中文的句号可能使用全角原点“.”\footnote{这个标点是 u+FF0E,称为 FULLWIDTH FULL STOP。},而不是平常的“。”,也不是正常的英文句点“.”。这个符号很难正常输入;你可以先输入正常句点,最后再替换。
	\subsection{latex安装与配置}
	英文单引号并不使用两个\verb|'|符号组合。左单引号是重音符\verb|`|(键盘上数字1左侧),而右单引号是常用的引号符。英文中,{\color{cyan}{左双引号就是连续两个重音符号}}。
	英文下的引号嵌套需要借助\latexline{thinspace}命令分隔,比如:
\begin{codeshow}[listing side text, listing options={escapeinside=++}]%方括号里的是啥意思?
``\thinspace`Max' is here.''
\end{codeshow}
中文下的单引号和双引号你可以用中文输入法直接输入。

\subsection{latex常用命令}
英文的短横分为三种:
\begin{feai}
\item 连字符:输入一个短横:\verb|-|,效果如daughter-in-law
\item 数字起止符:输入两个短横:\verb|--|,效果如:page 1--2
\item 破折号:输入三个短横:\verb|---|,效果如:Listen---I'm serious.
\end{feai}

中文的破折号你也许可以直接使用日常的输入方式。中文的省略号同样。但是注意,英文的省略号使用\latexline{ldots}这个命令而不是三个句点。
\subsection{Tizk}
\LaTeX\ 中专门有个叫做\latexline{emph\{text\}}的命令,可以强调文本。对于通常的西文文本,上述命令的作用就是斜体。如果你对一段已经这样转换为斜体的文本再使用这个命令,它就会取消斜体,而成为正体。

西文中一般采用上述的斜体强调方式而不是粗体,例如在说明书的时候可能就会使用以上命令。关于字体更多内容参考字体这一节。

\subsection{下划线与删除线}
\LaTeX\ 原生提供的\latexline{underline}命令简直烂的可以,建议你使用\pkg{ulem}宏包下的\texttt{uline}命令代替,它还支持换行文本。\pkg{ulem}宏包还提供了一些实用命令:


		\printindex%输出索引
		\bibliography{Bib}

\end{document}

