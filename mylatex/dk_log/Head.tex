\documentclass[a4paper,zihao=-4,linespread=1]{ctexrep}
\renewcommand{\CTEXthechapter}{\thechapter}%目录显示章节编号样式命令
% 最小行间间距设定
\setlength{\lineskiplimit}{3pt}
\setlength{\lineskip}{3pt}
% 下划线宏包
\usepackage[normalem]{ulem}
%LaTeX符号宏包
\usepackage{hologo}
\newcommand{\xelatex}{\Hologo{XeLaTeX}}%输出XeLaTeX标志
%颜色
\usepackage[table]{xcolor}
%奇怪的小定义
\newcommand{\RED}[1]{{\color{cyan}{#1}}}
\newcommand{\co}[1]{{\bfseries{#1}}}
% 编号列表宏包,并自定义了三个列表
\usepackage[inline]{enumitem}
        \setlist[enumerate]{font=\bfseries,itemsep=0pt}
        \setlist[itemize]{font=\bfseries,leftmargin=\parindent}
        \setlist[description]{font=\bfseries\uline,labelindent=\parindent,itemsep=0pt,parsep=0pt,topsep=0pt,partopsep=0pt}
\newenvironment{fead}{
    \begin{description}[font=\bfseries\uline,labelindent=\parindent,itemsep=0pt,parsep=0pt,topsep=0pt,partopsep=0pt]}
        {\end{description}}
%带宽度的
\newenvironment{para}{
        \begin{description}[font=\bfseries\ttfamily,itemsep=0pt,parsep=0pt,topsep=0pt,partopsep=0pt]}
        {\end{description}}
	%下边的定义猜测是缩小行间距
\newenvironment{feai}{/
        \begin{itemize}[font=\bfseries,itemsep=0pt,parsep=0pt,topsep=0pt,partopsep=0pt]}
        {\end{itemize}}
	\newenvironment{feae}{
        \begin{enumerate}[font=\bfseries,labelindent=0pt,itemsep=0pt,parsep=0pt,topsep=0pt,partopsep=0pt]}
	{\end{enumerate}}
%目录和章节样式
\usepackage{titlesec}
\usepackage{titletoc}
\titlecontents{chapter}[1.5em]{}{\contentslabel{1.5em}}{\hspace*{-2em}}{\hfill\contentspage}
\titlecontents{section}[3.3em]{}{\contentslabel{1.8em}}
        {\hspace*{-2.3em}}{\titlerule*[8pt]{$\cdot$}\contentspage}
\titlecontents*{subsection}[2.5em]{\small}{\thecontentslabel{} }{}{, \thecontentspage}[;\qquad][.]
% 章节样式
\setcounter{secnumdepth}{3} % 自动编号一直到subsubsection
\newcommand{\chaformat}[1]{%
        \parbox[b]{.5\textwidth}{\hfill\bfseries #1}%
        \quad\rule[-12pt]{2pt}{70pt}\quad
        {\fontsize{60}{60}\selectfont\thechapter}}
\titleformat{\chapter}[block]{\hfill\LARGE\sffamily}{}{0pt}{\chaformat}[\vspace{2.5pc}\normalsize
        \startcontents\setlength{\lineskiplimit}{0pt}\printcontents{}{1}{\setcounter{tocdepth}{2}\songti}]
\titleformat*{\section}{\centering\Large\bfseries}
%\titleformat{\subsubsection}[hang]{\bfseries\large}{\rule{1.5ex}{1.5ex}}{0.5em}{}%有它的时候subsubsection显示方块,所以删除了它
% 图表
\usepackage{array,multirow,makecell}
  \setlength\extrarowheight{2pt} % 行高增加
\usepackage{diagbox}
\usepackage{longtable}
\usepackage{graphicx,wrapfig}
  \graphicspath{{./tikz/}}
\usepackage{animate}
\usepackage{caption,subcaption}
  \captionsetup[sub]{labelformat=simple}
  \renewcommand{\thesubtable}{(\alph{subtable})}
% 三线表
\usepackage{booktabs}
%\renewcoemand{\contentsname}{目录}
% 代码环境
\usepackage{listings}%代码高亮显示
% 复制代码时不复制行号
\usepackage{accsupp}
  \newcommand{\emptyaccsupp}[1]{\BeginAccSupp{ActualText={}}#1\EndAccSupp{}}
\usepackage{tcolorbox}
  \tcbuselibrary{listings,skins,breakable,xparse}

% global style
\lstset{
  basicstyle=\small\ttfamily,
  % Word styles
  keywordstyle=\color{blue},
  commentstyle=\color{green!50!black},
  columns=fullflexible,  % Avoid too sparse word spaces
  keepspaces=true,
  % Numbering
  numbers=left,
  numberstyle=\tiny\color{red!75!black}\emptyaccsupp,
  % Lines and Skips
  aboveskip=0pt plus 6pt,
  belowskip=0pt plus 6pt,
  breaklines=true,
  breakatwhitespace=true,
  emptylines=1,  % Avoid >1 consecutive empty lines
  escapeinside=``
}

% 对于 tcolorbox 中 listings 库的 ''tcblatex'' style 的重现,
% 添加了新的关键词
\lstdefinestyle{latexcn}{
  language=[LaTeX]TeX,
  % More Keywords
  classoffset=0,
  texcsstyle=*\color{blue},
  moretexcs={
    % LaTeX extension
    chapter,section,subsection,setlength,
    thechapter,thesection,thesubsection,theequation,
    chaptermark,chaptername,appendix,
    bibname,refname,bibpreamble,bibfont,citenumfont,bibnumfmt,bibsep,
  },
  classoffset=1,
  texcsstyle=*\color{orange!75!black},
  moretexcs={
    % XeCJK & CTeX
    xeCJKsetup,setCJKmainfont,newCJKfontfamily,CJKfontspec,
    CTEXthechapter,songti,heiti,fangsong,kaishu,yahei,lishu,youyuan,
    % AMSmath / AMSsymb / AMSthm
    middle,text,tag,boldsymbol,mathbb,dddot,ddddot,iint,varoiint,
    dfrac,tfrac,cfrac,leftroot,uproot,underbracket,xleftarrow,xrightarrow,
    overset,underset,sideset,mathring,leqslant,geqslant,because,therefore,
    shortintertext,binom,dbinom,implies,thesubequation,
    impliedby,genfrac,theoremstyle,qedhere,
    % Other math packages
    wideparen,intertext,
    xlongequal,xLeftrightarrow,xleftrightarrow,xLongleftarrow,xLongrightarrow,
    % xcolor
    definecolor,color,textcolor,colorbox,fcolorbox,
    % hyperref
    hyperref,autoref,href,url,nolinkurl,
    % Graph & Table
    includegraphics,graphicspath,scalebox,rotatebox,animategraphics,
    newcolumntype,arraybackslash,multirow,captionsetup,
    thead,multirowcell,makecell,Xhline,Xcline,diagbox,
    toprule,midrule,bottomrule,DeclareFloatingEnvironment,
    % ulem
    uline,uuline,dashuline,dotuline,uwave,sout,xout,
    % fancyhdr
    lhead,chead,rhead,lfoot,cfoot,rfoot,
    fancyhf,fancyhead,fancyfoot,fancypagestyle,
    % fontspec
    newfontfamily,
    % titlesec & titletoc
    titlelabel,titleformat,titlespacing,titleline,titlerule,dottecontents,titlecontents,
    % enumitem
    setlist,
    % Listings & tcolorbox
    lstdefinelanguage,lstdefinestyle,lstset,lstnewenvironment,
    tcbuselibrary,newtcblisting,newtcbox,DeclareTCBListing
    % citation & index: natbib, imakeidx
    setcitestyle,printindex,
    % Other packages
    hologo,lettrine,endfirsthead,endhead,endlastfoot,columncolor,rowcolors,modulolinenumbers,MakeShortVerb,tikz,Hologo
  }
}
%定义shell命令格式
\lstdefinestyle{shell}{
  language=bash,
  % More Keywords
  classoffset=0,
  texcsstyle=*\color{blue},%shell命令颜色
  morekeywords={ssh,screen,scp}%增加shell命令关键词
}

% cmd & envi

\newtcbox{\latexline}[1][green]{on line,before upper=\ttfamily\char`\\,
  arc=0pt,outer arc=0pt,colback=#1!10!white,colframe=#1!50!black,
  boxsep=0pt,left=1pt,right=1pt,top=1pt,bottom=1pt,
  boxrule=0pt,bottomrule=1pt,toprule=1pt}
%pkg
\newtcbox{\pkg}[1][orange!70!red]{on line,before upper={\rule[-0.2ex]{0pt}{1ex}\ttfamily},
  arc=0.8ex,colback=#1!30!white,colframe=#1!50!black,
  boxsep=0pt,left=1.5pt,right=1.5pt,top=1pt,bottom=1pt,
  boxrule=1pt}

% tcblisting definitions
\newtcblisting{latex}{breakable,skin=bicolor,colback=gray!30!white,
  colbacklower=white,colframe=cyan!75!black,listing only, 
  left=6mm,top=2pt,bottom=2pt,fontupper=\small,
  listing options={style=latexcn}
}
%声明shell环境
\newtcblisting{shell}{breakable,skin=bicolor,colback=gray!30!white,
  colbacklower=white,colframe=cyan!75!black,listing only, 
  left=6mm,top=2pt,bottom=2pt,fontupper=\small,
  listing options={style=shell}
}
\NewTCBListing{codeshow}{ !O{listing side text} }{
  skin=bicolor,colback=gray!30!white,
  colbacklower=pink!50!yellow,colframe=cyan!75!black,
  valign lower=center,
  left=6mm,righthand width=0.4\linewidth,fontupper=\small,
  % listing style
  listing options={style=latexcn},#1,
}
%索引与参考文献
\usepackage{makeidx}%调用索引宏包
\bibliographystyle{plain}
\renewcommand{\bibname}{参考文献}
\usepackage[numindex,numbib]{tocbibind}
\usepackage[square,super,sort&compress]{natbib}
%引用
\usepackage{hyperref}
%目录,标签变成红色
\hypersetup{colorlinks, bookmarksopen = true, bookmarksnumbered = true, pdftitle=LaTeX-cn, pdfauthor=K.L Wu, pdfstartview=FitH}
