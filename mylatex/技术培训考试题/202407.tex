		\ifx \allfiles \undefined		%编译PPT时注释该行
\documentclass{book}
\usepackage{ifthen}
%\usepackage[a3paper,landscape,showframe,margin=1.25in]{geometry}
%\usepackage[a3paper,landscape,margin=1.1in]{geometry}
\usepackage[a3paper,landscape,top=1.25in,bottom=0.8in,left=1in,right=1in]{geometry}
\usepackage{tikz,units}
\usepackage{circuitikz}
\usepackage{subfig}
\usetikzlibrary{backgrounds,circuits.ee.IEC.relay}
\usetikzlibrary{positioning}
\usepackage{tikzsymbols}
\usepackage{pgffor}
\usetikzlibrary {math}
\usepackage{lastpage}
\usepackage{fancyhdr}
\pagestyle{fancy}
\fancyhf{}
\fancyhead[ER,OL]{\heiti \zihao{-5} 热工专业图纸}
\fancyhead[OR,EL]{\heiti \zihao{-5} \leftmark}
\fancyfoot[CE,CO]{热工专业图纸~第~\thepage~页(共 \pageref{LastPage} 页)}
\renewcommand{\headrulewidth}{0.4pt}
\renewcommand{\footrulewidth}{0.4pt}
\tikzset{
box/.style={rectangle,minimum height=17pt,minimum width=20pt,text=red}
}
\tikzset{
boxA/.style={rectangle,minimum height=17pt,minimum width=20pt,draw=black}
}
\tikzset{
boxB/.style={rectangle,minimum height=17pt,minimum width=30pt,draw=black}
}
\tikzset{
boxC/.style={rectangle,minimum height=17pt,minimum width=120pt,draw=black}
}
\tikzset{
boxD/.style={minimum width=140pt,above left}
}

\begin{document}
		\else						%编译PPT时注释该行
		%\newpage
		\fi						%编译PPT时注释该行
\watermark{50}{9}{热工班组}
\chapter[2024年07月份技术培训考试]{	\hspace*{-0.3em}\biaoti{2024}{07}{热工专业}}
姓名:\uline{ \ \  \  \ \ \ \ \ \ \ \ \ \ \ \ \ \ }\hfill 得分:\uline{ \ \  \  \ \ \ \ \ \  \ \ \ \ \ \ }
%\zysx
\section{\xzt{5}{2}{10}}
\begin{enumerate}
\item 我厂给煤机二次表供电来至\xz{B}
\xx{配电室UPS段220VAC电源}{电子间机柜内24VDC电源}{就地控制柜内电源模块输出24VDC}{配电室MCC段380VAC电源}
\item 我厂给煤机控制回路供电来至\xz{A}
\xx{配电室UPS段220VAC电源}{电子间机柜内24VDC电源}{就地控制柜内电源模块输出24VDC}{配电室MCC段380VAC电源}
\item 我厂给煤机进出口插板门控制回路供电来至\xz{D}
\xx{配电室UPS段220VAC电源}{电子间机柜内24VDC电源}{就地控制柜内电源模块输出24VDC}{配电室MCC段380VAC电源}
\item 我厂磨煤机出口插板门控制回路供电来至\xz{B}
\xx{配电室UPS段220VAC电源}{电子间机柜内24VDC电源}{就地控制柜内电源模块输出24VDC}{配电室MCC段380VAC电源}
\item 热电偶补偿导线的作用是\xz{C}。
\xx{补偿冷端温度变化}{便于信号传输}{延伸热电偶冷端}{提高测量的准确性}
\end{enumerate}
\section{\tkt{4}{2}{30}}
\begin{enumerate}
	\setcounter{enumi}{5}
	\item 我厂给煤机控制仪表控制模式分为三种分别为\tk{称重控制}、\tk{容积控制}和容积同步控制模式,正常测量过程中工作在\tk{称重控制}模式下。
\item 我厂给煤机有四个程序可实现基本标定功能分别为皮带整圈脉冲数设定、\tk[2]{去皮程序}、\tk[2]{零点设定}和\\ \tk[2.5]{量程标定程序},这三个程序都是在\tk{容积控制}模式\tk[4]{额定流量或20t/h}下进行的。
	\item 我厂给煤机控制仪表显示区域分为顶部显示区,\tk{故障显示区},上部显示区和下部显示区,其中顶部显示区主要用来显示秤的\tk[2]{运行状态}和\tk[2]{工作模式},正常运行过程中分别显示\tk[2]{转动的十字}和\tk[1]{否}(是否显示V)。
\item 我厂给煤机变频器故障就是使变频器停止的条件,有两种故障类型,分别是\tk{可自动复位}和\tk{不可自动复位}。
\end{enumerate}
\section{\stt{40}}
\begin{enumerate}
	\setcounter{enumi}{9}
	\item 去皮程和零点设定程序的执行,实际上是记录皮带运行一整圈或几个整圈的零点平均误差并在以后的操作中消除此误差,但两者又存在些许差别。
\\其中去皮程序功能是什么?\fenzhi{5}
\wdt[2]{答:记录基本皮重 (包括设备本体的影响,预加载等) ,通常在初次调试服务和设备维护期间进行去皮操作。}\\
\\其中零点设定程序功能是什么?\fenzhi{5}
\wdt[2]{答:记录当皮带表面被污染或其他情况下,零点较小的误差并修正。}\\
去皮程序和零点设定程序区别是什么?\fenzhi{5}
\wdt[2]{答:零点设定程序可以修正的最大误差是受参数设置限制的;而去皮程序的结果无此限制。在执行去皮程序得出基本皮重后,零点设定程序将根据新的基本皮重值和限制幅值修正零点误差。}\\
去皮程序和零点设定程序结束后,需要修改哪些参数,是否需要手动修改\fenzhi{10}
\wdt[3]{答:程序结束时在顶部显示区显示TW:等待确认,表明程序已经结束,按下确认键自动更改参数。零点设定程序自动修改P04.05(皮重修正值),去皮程序自动修改P04.04(基本皮重)同时将P04.05(皮重修正值)置零。}\\
\item 量程标定程序用于控制和补偿仪表测量信号的衰减。 通过向称重台面加载已知重量的砝码,由仪表自动计算此负荷率下,皮带运行一整圈或多个整圈的累积量。这个累积量理论值用于修正实际测量出的累积量显示值,同时修正以后的测量值。
\\量程标定程序执行的前提条件有哪些?\fenzhi{10}
\wdt[2]{答:1.执行过去皮程序或零点设定程序;2.将已知的砝码重量输入 P03.08 参数;3.砝码就位;4.仪表工作在容积控制模式。}\\
量程标定程序结束后,需要修改哪些参数,是否需要手动修改\fenzhi{5}
\wdt[3]{答:程序结束时在顶部显示区显示CW:等待确认,表明程序已经结束,按下确认键退出程序,需要手动修改P04.02(量程校正系数)。}\\
\end{enumerate}
\section{\jdt{20}}
\begin{enumerate}
	\setcounter{enumi}{11}
\item 给煤机作为重点设备,重点巡检内容有哪些?常见故障有哪些?简述给煤机风扇故障应急处置方法?\fenzhi{20}
\wdt[6]{答:\\
给煤机变频器是否有报警,风扇工作是否正常(3分)
给煤机二次表是否有报警,流量/转速是否有明显跳变现象(3分)
变频器风扇不转导致变频器超温故障(3分)
给煤机二次表未切至运行模式,断电后未手动启动导致给煤机启动不转(3分)
给煤机风散应急处置方法:备用风散在工程师站第三个玻璃文件柜内,处置的时候拿上手持风机,风机吹变频器冷却。首先到达现场后,断电复位变频器消除故障,风机冷却,让运行启动给煤机,启动后在线更换风散,减少给煤机失备时间。(8分)
}

\end{enumerate}
		\ifx \allfiles \undefined
\end{document}
\fi
