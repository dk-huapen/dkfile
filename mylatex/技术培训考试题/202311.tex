		\ifx \allfiles \undefined		%编译PPT时注释该行
\documentclass{book}
%%%%%%%%%%%%%%%%%%%%%%%%%%%%%%%%%%%%%%%%%
% 模板資訊:
% 模板名稱:Beamer
% 版本:1.0 (2023.07.09)
% 修改者:Ernie
% 編譯器:XeLaTeX
%
% 原始模板的資訊:
% 模板名稱:Beamer Presentation
% 作者:Vel (vel@latextemplates.com)
% 編譯器:XeLaTeX
% 授權:CC BY-NC-SA 4.0 (https://creativecommons.org/licenses/by-nc-sa/4.0/)
% 下載連結:https://www.LaTeXTemplates.com
%
% 製作本模板之目的:
% 為了讓 LaTeX 初學者能夠輕鬆地完成專業的學術簡報,因此我針對 Vel 製作的模板做了大幅度的修改及附上清楚明瞭的註解。
%
% 如果您有任何問題,可以透過 Email 聯繫我們:stateconlab@gmail.com
% 
% p.s. 也別忘了關注我們的 YouTube、IG 和 Medium 喔!
% 1. YouTube:https://www.youtube.com/@StatEconLab
% 2. IG:https://www.instagram.com/stateconlab
% 3. Mediun:https://medium.com/@stateconlab
%%%%%%%%%%%%%%%%%%%%%%%%%%%%%%%%%%%%%%%%%

%----------------------------------------------------------------------------------------
%	封包與文檔配置
%----------------------------------------------------------------------------------------

\usepackage[space,noindent]{ctex}

% 自訂字體顏色的封包
\usepackage{xcolor} 

%% 自訂顏色

\definecolor{pbblue}{HTML}{0A75A8}% color for the progress bar and the circle
% 數學工具及符號
%\usepackage{mathtools, amsmath, amsfonts, amsthm, latexsym} 

% 分別將數學符號間的間隔加大及加粗
%\usepackage{newtxtext,newtxmath}

% 圖表自動編號的封包
%\usepackage{caption} 

%% 設定自動編號
%\setbeamertemplate{caption}[numbered]

%% 設定圖表編號及標籤的字體大小及字形
%\captionsetup[figure]{font=small, labelfont=md}
%\captionsetup[table]{font=small, labelfont=md}

% 導入圖形與表格的封包
%\usepackage{graphicx}  % \scalebox{} 可用於將過大的表格縮小
%\usepackage{booktabs}

% 排列多個子圖形的封包
%\usepackage{subfigure} 

% 允許表格的一格能多列呈現的封包
%\usepackage{multirow} 

% 可指定表格排版的封包
%\usepackage{array}

% 翻轉表格的封包
%\usepackage{lscape} 

% 序列標號
%\usepackage{enumerate} 

% 繪圖封包 (用於添加浮水印)
\usepackage{tikz}

% 引注參考資料
%\usepackage{natbib}

% 註釋掉大部分的封包
%\usepackage{comment}

\usetikzlibrary{shapes,fit,calc,positioning}

% 設定中文的標籤
%\renewcommand{\figurename}{圖} 
\renewcommand{\tablename}{表} 

%----------------------------------------------------------------------------------------
%	排版形式 (擇一,不選等同選擇默認的排版形式)
%----------------------------------------------------------------------------------------

%\mode<presentation>{
%\usetheme{default}
%\usetheme{AnnArbor}
%\usetheme{Antibes}
%\usetheme{Bergen}
%\usetheme{Berkeley}%演示主题为侧边导航条
%\usetheme{Berlin}
\usetheme{Boadilla}%蓝色主题
%\usetheme{CambridgeUS}
%\usetheme{Copenhagen}
%\usetheme{Darmstadt}
%\usetheme{Dresden}
%\usetheme{Frankfurt}
%\usetheme{Goettingen}
%\usetheme{Hannover}
%\usetheme{Ilmenau}
%\usetheme{JuanLesPins}
%\usetheme{Luebeck}
%\usetheme{Madrid}
%\usetheme{Malmoe}
%\usetheme{Marburg}
%\usetheme{Montpellier}
%\usetheme{PaloAlto}
%\usetheme{Pittsburgh}
%\usetheme{Rochester}
%\usetheme{Singapore}
%\usetheme{Szeged}
%\usetheme{Warsaw}

%----------------------------------------------------------------------------------------
%	外框形式 (擇一,不選等同選擇默認的外框形式)
%----------------------------------------------------------------------------------------

%\useoutertheme{default}
%\useoutertheme{infolines}
%\useoutertheme{miniframes}
%\useoutertheme{smoothbars}
%\useoutertheme{sidebar}
%\useoutertheme{split}
%\useoutertheme{shadow}
%\useoutertheme{tree}
%\useoutertheme{smoothtree}

%----------------------------------------------------------------------------------------
%	外框的自訂義調整 
%----------------------------------------------------------------------------------------

% 外框上緣的字 (fg) 為黑色,背景 (bg) 為白色。
%\setbeamercolor{section in head/foot}{fg=white, bg=black} 

% 外框上緣顯示的章節(section)頁數標籤是否關閉
%\setbeamertemplate{mini frames}{}   

% 調整外框形式的字體大小
%\setbeamerfont{headline}{size=\scriptsize}
%\setbeamerfont{footline}{size=\scriptsize}

% 取消右下方的跳轉工具列
\setbeamertemplate{navigation symbols}{} 

%% 自定義1:外框下緣僅出現名字及頁碼
%\setbeamertemplate{footline}
%{\leavevmode%
%\hbox{%
%\begin{beamercolorbox}[wd=0.5\paperwidth,ht=3ex,dp=1ex,leftskip=2ex]%
%{author in head/foot}%
%{\footnotesize\textbf{\insertshortauthor}}%
%\end{beamercolorbox}%
%\begin{beamercolorbox}[wd=0.5\paperwidth,ht=3ex,dp=1ex,right]%
%{author in head/foot}%
%\footnotesize \textbf{{\insertframenumber{} / \inserttotalframenumber\hspace*{2ex}}} %頁碼控制選項
%\end{beamercolorbox}%
%}}

%% 自定義2:清除外框下緣但僅出頁碼
%\setbeamertemplate{footline}[page number] 

%% 自定義3:清除外框下緣
%\setbeamertemplate{footline}[] 

%----------------------------------------------------------------------------------------
%	顏色主題 (擇一,不選等同選擇默認的顏色主題)
%----------------------------------------------------------------------------------------

%\usecolortheme{default}
%\usecolortheme{albatross}
%\usecolortheme{beaver}
%\usecolortheme{beetle}
%\usecolortheme{crane}
%\usecolortheme{dolphin}
%\usecolortheme{dove}
%\usecolortheme{fly}
%\usecolortheme{lily}
%\usecolortheme{orchid}
%\usecolortheme{rose}
%\usecolortheme{seagull}
%\usecolortheme{seahorse}
\usecolortheme{whale}%颜色主题为
%\usecolortheme{wolverine}

%----------------------------------------------------------------------------------------
%	顏色主題的自訂義調整 
%----------------------------------------------------------------------------------------

% 全文的主題色 (可以特別針對報告對象或機構的代表色調整!)
%\setbeamercolor{structure}{fg=Myblue} 

% 封面頁中標題區塊的底色及字體顏色
%\setbeamercolor{title}{bg=green, fg=black} 

% 各頁標題區塊的底色及字體顏色
%\setbeamercolor{frametitle}{bg=white,fg=black} 

% 全文的內文顏色
%\setbeamercolor{normal text}{fg=orange}

% 數學區塊的標題顏色 
%\setbeamercolor{block title}{bg=blue,fg=yellow} 

% 數學區塊的內文顏色 
%\setbeamercolor{block body}{bg=green,fg=red} 

% 警示文字的顏色
%\setbeamercolor{alerted text}{fg=red} 

%----------------------------------------------------------------------------------------
%	enumerate 及 item 的形狀
%----------------------------------------------------------------------------------------

%\useinnertheme{rounded} % 圓球 (3D)
%\useinnertheme{circles} % 圓形 (2D)
%\useinnertheme{rectangles} % 方形
%\useinnertheme{triangle} % 三角形
%\useinnertheme{inmargin} % 插入邊沿
%\setbeamertemplate{itemize items}[triangle]

%----------------------------------------------------------------------------------------
%	自訂 item 的顏色
%----------------------------------------------------------------------------------------

%\setbeamercolor{item projected}{bg=red}

%----------------------------------------------------------------------------------------
%	個人化的設置及細節調整
%----------------------------------------------------------------------------------------

% 設定頁面邊界
%\setbeamersize{text margin left=0.6cm, text margin right=0.6cm}
%\special{papersize=\the\paperwidth,\the\paperheight}
%\providecommand{\tabularnewline}{\\}
%}

%----------------------------------------------------------------------------------------
%	個人化的背景調控
%----------------------------------------------------------------------------------------

% 背景照片設置
%\setbeamertemplate{background}{\includegraphics[height=\paperheight]{Fig/Background.png}}

% 浮水印設定
%\usebackgroundtemplate{%
%	\tikz[overlay, remember picture] % 讓 logo 能每頁都顯示
%	\node[opacity=0.3, below=-1.25cm, at=(current page.center)] % 調整透明度 (opacity) 及浮水印的位置
%	{\includegraphics[scale = 0.14]{Fig/nthulogo.png}}; % 載入 logo 及調整大小
%	}

\begin{document}
		\else						%编译PPT时注释该行
		%\newpage
		\fi						%编译PPT时注释该行
\watermark{50}{9}{热工班组}
\chapter[2023年11月份技术培训考试]{	\hspace*{-0.3em}\biaoti{2023}{11}{热工专业}}
姓名:\uline{ \ \  \  \ \ \ \ \ \ \ \ \ \ \ \ \ \ }\hfill 得分:\uline{ \ \  \  \ \ \ \ \ \  \ \ \ \ \ \ }
%\zysx
\section{\tkt{4}{2}{24}}
\begin{enumerate}
	\setcounter{enumi}{0}
	\item 我厂800xA系统中主控制器的型号是\tk{PM864}。
	\item 我厂DCS服务器分为\tk{IM},\tk{AS},\tk{CS}三种。
	\item 采用按控制功能划分的设计原则时,分散控制系统可分为DAS、MCS、SCS、FSSS等子系统,其中MCS的中文含义是\tk[4]{模拟量控制系统},FSSS的中文含义是\tk[4]{炉膛安全监控系统}。
	\item 我厂800xA DCS系统除脱硫DCS系统共分为\tk[1]{5}个网段,其中一网段主要分布\tk{1、2号锅炉}设备,二网段主要分布\tk{3、4号锅炉}设备,三网段主要分布\tk{1、2、3号汽轮机}设备,四网段主要分布\tk{4、5号汽轮机}设备,五网段主要分布\tk{管网公用系统}设备。
\end{enumerate}
\section{\jdt{40}}
\begin{enumerate}
	\setcounter{enumi}{4}
	\item 简述PM864各指示灯作用及含义\fenzhi{15}
\wdt[3.5]{答:\\(1)F为故障指示灯,故障时为红色,正常运行不亮;(2)R为运行指示灯,正常运行的冗余设备,主设备该灯常亮,从设备该灯不亮;(3)P灯为电源指示灯,设备带电后该灯常亮;(4)B为电池指示灯,电池工作正常时该灯均常亮,异常时该灯闪烁;(5)PRIM灯为主站运行指示灯,冗余运行设备主设备该灯亮,从设备该灯不亮;(6)DUAL为冗余指示灯,正常运行的冗余设备该灯均常亮。}
	\item 画出我厂锅炉本体吹灰器二次回路控制原理图\fenzhi{25}
\wdt[9]{答:\\}
\end{enumerate}
\section{\stt{36}}
\begin{enumerate}
	\setcounter{enumi}{6}
	\item 该吹灰器启动时对远方脉冲或就地启动按钮有什么要求,,否则会有什么故障?\fenzhi{4}
\wdt{答:\\脉冲时间要求大于退到位限位开关脱开时间,否则吹灰器无法继续自保持前进。}
	\item 该吹灰器进到位后,吹灰器先执行后退动作还是先停留吹扫?简述后退过程中KT1与KT3的作用?\fenzhi{8}
\wdt{答:\\先停留吹扫,待KT1计时时间到后再进行后退动作并开始计时KT3,KT3计时时间到后再停下吹扫重新开始KT1计时,如此往复直到退到位限位开关动作。}
	\item 写出以下元器件在回路中的作用\fenzhi{24}
GK:\tk{检修开关},LSF:\tk{进到位限位},LSR:\tk{退到位限位},\\START:\tk{现场启动按钮},D-DJ1:\tk{远方启动脉冲},DJ1F:\tk{前进交流接触器},\\DJ1R:\tk{后退交流接触器},DJ1RJ:\tk{热继电器长闭点},SB:\tk{现场急退},\\DCS:\tk{远方急退},DJ1J:\tk{后退标记继电器},KA1J:\tk{后退动作继电器}。
\end{enumerate}
\begin{figure}[htbp]
\centering
\begin{tikzpicture}[circuit ee IEC relay,thick,x=8\tikzcircuitssizeunit,y=7\tikzcircuitssizeunit]
		
			\draw (0,0)node [contact,name=F1]{} -- ++(0,1) node [contact,name=G1]{}
				-- ++(0,1)node [contact,name=H1]{}
-- ++(0,1)node [contact,name=A1]{}
-- ++(0,1)node [contact,name=B1]{}
-- ++(0,1)node [contact,name=C1]{}
-- ++(0,1)node [contact,name=D1]{}
-- ++(0,1)node [contact,name=E1]{}
-- ++(0,1.2)node [contact,name=L]{};

	\draw (D1) -- ++(1,0)
		to [make contact={name=GK1,term=8-i}] ++(1,0)
		to [break contact={position switch={info=$LSR$}}] ++(1,0)
		node [contact,name=D2]{}
	to [make contact={info=$KA1J$}] ++(1,0) -- ++(1,0)
		to [break contact={info=$DJ1F$}] ++(1,0)
		to [relay coil={info=$DJ1R$,term=A2,term'=A1}] ++(1,0) 
		node [contact,name=D3]{};
	
				

		\draw (E1) node[contact]{}
		to [make contact={push button={info=$START$}}] ++(1,0)

		node [contact,name=E2]{}
		to [make contact={turn switch={info=$GK$},name=GK2}] ++(1,0)
		node [contact,name=E3]{}
		to [break contact={position switch={info=$LSF$}}] ++(1,0)
		to [break contact={info=OA1J}] ++(1,0)
		to [break contact={info=DJ1J}] ++(1,0)
		to [break contact={info=DJ1R}] ++(1,0)
		to [relay coil={info=DJ1F,term=A2,term'=A1}] ++(1,0)
		node [contact,name=E4]{};

\draw[dashed](GK1.mid) -- (GK2.mid);
\draw (E3) -- ++(0,-0.5) to [make contact={info={[right=0.3cm]:DJ1F}}] ++(1,0) -- (D2);
\draw (L) ++(1,0)node [contact,name=Di]{}--(E2);

		

		\draw (D2) -- ++(0,-1)
		to [break contact={info=DJ1F}] ++(1,0) -- ++(2,0)
		to [relay coil={info=DJ1J}] ++(1,0)
		node [contact,name=C3]{};
\draw (C3) -- (E4)
to [break contact={thermal switch={info=DJ1RJ}}] ++(0,1.2)node [contact,name=N]{};
\node[above,blue] at(N) {N};
\node[left,red] at(L) {A1};
\node[left,red] at(Di) {D-DJ1};

\draw (A1) -- ++(1,0)
to [break contact={info=OA1J}] ++(1,0)
to [make contact={info=DJ1J}] ++(1,0)
to [break contact={delayed deactivation={info=KT3}}] ++(1,0)
-- ++(2,0)
to [relay coil={slow operating={info=KT1}}]++(1,0)
node [contact,name=A2]{};

\draw (B1)
to [make contact={delayed deactivation={info=KT1}}] ++(1,0)
node [contact,name=B2]{}
to [relay coil={info=KA1J}] ++(1,0)
node [contact,name=B3]{}
-- ++(5,0)
node [contact,name=B4]{};

\draw (C1)
to [make contact={info=OA1J}] ++(1,0)
node [contact,name=C2]{}
to [relay coil={slow operating={info=KT3}}]++(1,0)
to (B3);

\draw (B2) -- (C2);

\draw (G1)
to [make contact={push button={info=$SB$,term''=现场急退}}] ++(2,0)
node [contact,name=G2]{};
\draw (H1)
to [make contact={info=$DCS$}] ++(2,0)
node [contact,name=H2]{}
-- ++(4,0)
to [relay coil={info=OA1J}] ++(1,0) -- (C3);
\draw (F1)
to [make contact={info=OA1J}] ++(1,0)
to [make contact={info=DJ1J}] ++(1,0) -- (H2);
%\draw (F2) -- (G2) -- (H2);
	\end{tikzpicture}
\caption{1-3号锅炉空预器多介质吹灰器二次控制原理图}
\end{figure}

		\ifx \allfiles \undefined
\end{document}
\fi
