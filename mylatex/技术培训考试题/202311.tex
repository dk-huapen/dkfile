		\ifx \allfiles \undefined		%编译PPT时注释该行
\documentclass{book}
\usepackage{ifthen}
%\usepackage[a3paper,landscape,showframe,margin=1.25in]{geometry}
%\usepackage[a3paper,landscape,margin=1.1in]{geometry}
\usepackage[a3paper,landscape,top=1.25in,bottom=0.8in,left=1in,right=1in]{geometry}
\usepackage{tikz,units}
\usepackage{circuitikz}
\usepackage{subfig}
\usetikzlibrary{backgrounds,circuits.ee.IEC.relay}
\usetikzlibrary{positioning}
\usepackage{tikzsymbols}
\usepackage{pgffor}
\usetikzlibrary {math}
\usepackage{lastpage}
\usepackage{fancyhdr}
\pagestyle{fancy}
\fancyhf{}
\fancyhead[ER,OL]{\heiti \zihao{-5} 热工专业图纸}
\fancyhead[OR,EL]{\heiti \zihao{-5} \leftmark}
\fancyfoot[CE,CO]{热工专业图纸~第~\thepage~页(共 \pageref{LastPage} 页)}
\renewcommand{\headrulewidth}{0.4pt}
\renewcommand{\footrulewidth}{0.4pt}
\tikzset{
box/.style={rectangle,minimum height=17pt,minimum width=20pt,text=red}
}
\tikzset{
boxA/.style={rectangle,minimum height=17pt,minimum width=20pt,draw=black}
}
\tikzset{
boxB/.style={rectangle,minimum height=17pt,minimum width=30pt,draw=black}
}
\tikzset{
boxC/.style={rectangle,minimum height=17pt,minimum width=120pt,draw=black}
}
\tikzset{
boxD/.style={minimum width=140pt,above left}
}

\begin{document}
		\else						%编译PPT时注释该行
		%\newpage
		\fi						%编译PPT时注释该行
\watermark{50}{9}{热工班组}
\chapter[2023年11月份技术培训考试]{	\hspace*{-0.3em}\biaoti{2023}{11}{热工专业}}
姓名:\uline{ \ \  \  \ \ \ \ \ \ \ \ \ \ \ \ \ \ }\hfill 得分:\uline{ \ \  \  \ \ \ \ \ \  \ \ \ \ \ \ }
%\zysx
\section{\tkt{4}{2}{24}}
\begin{enumerate}
	\setcounter{enumi}{0}
	\item 我厂800xA系统中主控制器的型号是\tk{PM864}。
	\item 我厂DCS服务器分为\tk{IM},\tk{AS},\tk{CS}三种。
	\item 采用按控制功能划分的设计原则时,分散控制系统可分为DAS、MCS、SCS、FSSS等子系统,其中MCS的中文含义是\tk[4]{模拟量控制系统},FSSS的中文含义是\tk[4]{炉膛安全监控系统}。
	\item 我厂800xA DCS系统除脱硫DCS系统共分为\tk[1]{5}个网段,其中一网段主要分布\tk{1、2号锅炉}设备,二网段主要分布\tk{3、4号锅炉}设备,三网段主要分布\tk{1、2、3号汽轮机}设备,四网段主要分布\tk{4、5号汽轮机}设备,五网段主要分布\tk{管网公用系统}设备。
\end{enumerate}
\section{\jdt{40}}
\begin{enumerate}
	\setcounter{enumi}{4}
	\item 简述PM864各指示灯作用及含义\fenzhi{15}
\wdt[3.5]{答:\\(1)F为故障指示灯,故障时为红色,正常运行不亮;(2)R为运行指示灯,正常运行的冗余设备,主设备该灯常亮,从设备该灯不亮;(3)P灯为电源指示灯,设备带电后该灯常亮;(4)B为电池指示灯,电池工作正常时该灯均常亮,异常时该灯闪烁;(5)PRIM灯为主站运行指示灯,冗余运行设备主设备该灯亮,从设备该灯不亮;(6)DUAL为冗余指示灯,正常运行的冗余设备该灯均常亮。}
	\item 画出我厂锅炉本体吹灰器二次回路控制原理图\fenzhi{25}
\wdt[9]{答:\\}
\end{enumerate}
\section{\stt{36}}
\begin{enumerate}
	\setcounter{enumi}{6}
	\item 该吹灰器启动时对远方脉冲或就地启动按钮有什么要求,,否则会有什么故障?\fenzhi{4}
\wdt{答:\\脉冲时间要求大于退到位限位开关脱开时间,否则吹灰器无法继续自保持前进。}
	\item 该吹灰器进到位后,吹灰器先执行后退动作还是先停留吹扫?简述后退过程中KT1与KT3的作用?\fenzhi{8}
\wdt{答:\\先停留吹扫,待KT1计时时间到后再进行后退动作并开始计时KT3,KT3计时时间到后再停下吹扫重新开始KT1计时,如此往复直到退到位限位开关动作。}
	\item 写出以下元器件在回路中的作用\fenzhi{24}
GK:\tk{检修开关},LSF:\tk{进到位限位},LSR:\tk{退到位限位},\\START:\tk{现场启动按钮},D-DJ1:\tk{远方启动脉冲},DJ1F:\tk{前进交流接触器},\\DJ1R:\tk{后退交流接触器},DJ1RJ:\tk{热继电器长闭点},SB:\tk{现场急退},\\DCS:\tk{远方急退},DJ1J:\tk{后退标记继电器},KA1J:\tk{后退动作继电器}。
\end{enumerate}
\begin{figure}[htbp]
\centering
\begin{tikzpicture}[circuit ee IEC relay,thick,x=8\tikzcircuitssizeunit,y=7\tikzcircuitssizeunit]
		
			\draw (0,0)node [contact,name=F1]{} -- ++(0,1) node [contact,name=G1]{}
				-- ++(0,1)node [contact,name=H1]{}
-- ++(0,1)node [contact,name=A1]{}
-- ++(0,1)node [contact,name=B1]{}
-- ++(0,1)node [contact,name=C1]{}
-- ++(0,1)node [contact,name=D1]{}
-- ++(0,1)node [contact,name=E1]{}
-- ++(0,1.2)node [contact,name=L]{};

	\draw (D1) -- ++(1,0)
		to [make contact={name=GK1,term=8-i}] ++(1,0)
		to [break contact={position switch={info=$LSR$}}] ++(1,0)
		node [contact,name=D2]{}
	to [make contact={info=$KA1J$}] ++(1,0) -- ++(1,0)
		to [break contact={info=$DJ1F$}] ++(1,0)
		to [relay coil={info=$DJ1R$,term=A2,term'=A1}] ++(1,0) 
		node [contact,name=D3]{};
	
				

		\draw (E1) node[contact]{}
		to [make contact={push button={info=$START$}}] ++(1,0)

		node [contact,name=E2]{}
		to [make contact={turn switch={info=$GK$},name=GK2}] ++(1,0)
		node [contact,name=E3]{}
		to [break contact={position switch={info=$LSF$}}] ++(1,0)
		to [break contact={info=OA1J}] ++(1,0)
		to [break contact={info=DJ1J}] ++(1,0)
		to [break contact={info=DJ1R}] ++(1,0)
		to [relay coil={info=DJ1F,term=A2,term'=A1}] ++(1,0)
		node [contact,name=E4]{};

\draw[dashed](GK1.mid) -- (GK2.mid);
\draw (E3) -- ++(0,-0.5) to [make contact={info={[right=0.3cm]:DJ1F}}] ++(1,0) -- (D2);
\draw (L) ++(1,0)node [contact,name=Di]{}--(E2);

		

		\draw (D2) -- ++(0,-1)
		to [break contact={info=DJ1F}] ++(1,0) -- ++(2,0)
		to [relay coil={info=DJ1J}] ++(1,0)
		node [contact,name=C3]{};
\draw (C3) -- (E4)
to [break contact={thermal switch={info=DJ1RJ}}] ++(0,1.2)node [contact,name=N]{};
\node[above,blue] at(N) {N};
\node[left,red] at(L) {A1};
\node[left,red] at(Di) {D-DJ1};

\draw (A1) -- ++(1,0)
to [break contact={info=OA1J}] ++(1,0)
to [make contact={info=DJ1J}] ++(1,0)
to [break contact={delayed deactivation={info=KT3}}] ++(1,0)
-- ++(2,0)
to [relay coil={slow operating={info=KT1}}]++(1,0)
node [contact,name=A2]{};

\draw (B1)
to [make contact={delayed deactivation={info=KT1}}] ++(1,0)
node [contact,name=B2]{}
to [relay coil={info=KA1J}] ++(1,0)
node [contact,name=B3]{}
-- ++(5,0)
node [contact,name=B4]{};

\draw (C1)
to [make contact={info=OA1J}] ++(1,0)
node [contact,name=C2]{}
to [relay coil={slow operating={info=KT3}}]++(1,0)
to (B3);

\draw (B2) -- (C2);

\draw (G1)
to [make contact={push button={info=$SB$,term''=现场急退}}] ++(2,0)
node [contact,name=G2]{};
\draw (H1)
to [make contact={info=$DCS$}] ++(2,0)
node [contact,name=H2]{}
-- ++(4,0)
to [relay coil={info=OA1J}] ++(1,0) -- (C3);
\draw (F1)
to [make contact={info=OA1J}] ++(1,0)
to [make contact={info=DJ1J}] ++(1,0) -- (H2);
%\draw (F2) -- (G2) -- (H2);
	\end{tikzpicture}
\caption{1-3号锅炉空预器多介质吹灰器二次控制原理图}
\end{figure}

		\ifx \allfiles \undefined
\end{document}
\fi
