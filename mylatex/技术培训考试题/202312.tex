		\ifx \allfiles \undefined		%编译PPT时注释该行
\documentclass{book}
\usepackage{ifthen}
%\usepackage[a3paper,landscape,showframe,margin=1.25in]{geometry}
%\usepackage[a3paper,landscape,margin=1.1in]{geometry}
\usepackage[a3paper,landscape,top=1.25in,bottom=0.8in,left=1in,right=1in]{geometry}
\usepackage{tikz,units}
\usepackage{circuitikz}
\usepackage{subfig}
\usetikzlibrary{backgrounds,circuits.ee.IEC.relay}
\usetikzlibrary{positioning}
\usepackage{tikzsymbols}
\usepackage{pgffor}
\usetikzlibrary {math}
\usepackage{lastpage}
\usepackage{fancyhdr}
\pagestyle{fancy}
\fancyhf{}
\fancyhead[ER,OL]{\heiti \zihao{-5} 热工专业图纸}
\fancyhead[OR,EL]{\heiti \zihao{-5} \leftmark}
\fancyfoot[CE,CO]{热工专业图纸~第~\thepage~页(共 \pageref{LastPage} 页)}
\renewcommand{\headrulewidth}{0.4pt}
\renewcommand{\footrulewidth}{0.4pt}
\tikzset{
box/.style={rectangle,minimum height=17pt,minimum width=20pt,text=red}
}
\tikzset{
boxA/.style={rectangle,minimum height=17pt,minimum width=20pt,draw=black}
}
\tikzset{
boxB/.style={rectangle,minimum height=17pt,minimum width=30pt,draw=black}
}
\tikzset{
boxC/.style={rectangle,minimum height=17pt,minimum width=120pt,draw=black}
}
\tikzset{
boxD/.style={minimum width=140pt,above left}
}

\begin{document}
		\else						%编译PPT时注释该行
		%\newpage
		\fi						%编译PPT时注释该行
\watermark{50}{9}{热工班组}
\chapter[2023年12月份技术培训考试]{	\hspace*{-0.3em}\biaoti{2023}{12}{热工专业}}
姓名:\uline{ \ \  \  \ \ \ \ \ \ \ \ \ \ \ \ \ \ }\hfill 得分:\uline{ \ \  \  \ \ \ \ \ \  \ \ \ \ \ \ }
%\zysx
\section{\xzt{5}{2}{10}}
\begin{enumerate}
\item DCS装置本身只是一个软件、硬件的组合体,只有经过\xz{A}以后才能成为真正适用于生产过程的应用控制系统。
\xx{软、硬件组态}{程序下载}{程序编写}{程序编译}
\item 对于DCS软件闭环控制的气动调节执行机构,下列哪些方法不改变其行程特性\xz{B}
\xx{更换位置变送器反馈凸轮}{更换远传位置反馈电路板}{更换主控制板}{调整位置反馈连杆}

\item 我厂哪种设备接线方式为四线制\xz{C}
\xx{ABB温度变送器}{ABB气动执行器反馈装置}{SIPOS电动执行机构反馈装置}{PDS压力变送器}

\item 我厂哪种设备接线方式为两线制\xz{C}
\xx{MTL温度变送器}{给煤机二次表瞬时流量反馈}{ABB气动执行器反馈装置}{PH计二次表反馈}

\item 我厂给煤机二次表供电来至\xz{B}
\xx{配电室UPS段220VAC电源}{电子间机柜内24VDC电源}{就地控制柜内电源模块输出24VDC}{配电室MCC段220VAC电源}
\end{enumerate}
\section{\tkt{4}{2}{30}}
\begin{enumerate}
	\setcounter{enumi}{5}
	\item 我厂Tricon控制系统为\tk{故障安全型}控制系统,在电机正常运行过程中电机故障信号为\tk[1]{1}。
\item 我厂给煤机控制仪表控制模式分为三种分别为\tk{称重控制模式}、\tk{容积控制模式}和容积同步控制模式,正常测量过程中工作在\tk{称重控制}模式下。
\item 我厂给煤机有四个程序可实现基本标定功能分别为皮带整圈脉冲数设定、\tk[2]{去皮程序}、\tk[2]{零点设定}和\\ \tk[2.5]{量程标定程序},这三个程序都是在\tk{容积控制}模式\tk[3]{额定流量或20t/h}下进行的。
	\item 我厂给煤机控制仪表显示区域分为顶部显示区,\tk{故障显示区},上部显示区和下部显示区,其中顶部显示区主要用来显示秤的\tk[2]{运行状态}和\tk[2]{工作模式},正常运行过程中分别显示\tk[2]{转动的十字}和\tk[1]{否}(是否显示V)。

\end{enumerate}
\section{\jdt{20}}
\begin{enumerate}
	\setcounter{enumi}{9}
	\item 我厂1、2、3号汽轮机正常运行过程中ETS跳闸信号输出为0还是1?ETS直接动作哪些设备?简述或画出操作台停机按钮接线方式\fenzhi{10}
\wdt[2]{答:\\正常运行过程中ETS跳闸信号输出为1\\ETS直接动作设备为4台AST电磁阀\\操作台停机按钮接线方式为:两个急停按钮常闭点并联}
	\item 我厂4、5号汽轮机正常运行过程中ETS跳闸信号输出为0还是1?ETS直接动作哪些设备?简述或画出操作台停机按钮接线方式\fenzhi{10}
\wdt[2]{答:\\正常运行过程中ETS跳闸信号输出为0\\ETS直接动作设备为主汽门、抽汽逆止门\\操作台停机按钮接线方式为:两个急停按钮常开点串联}
\end{enumerate}
\section{\stt{40}}
\begin{enumerate}
	\setcounter{enumi}{11}
	\item 锅炉正常运行过程中DCS系统MFT动作信号为0还是1?MFT继电器状态?DCS系统失灵,操作台MFT急停按钮能否正常触发锅炉MFT?\fenzhi{5}
\wdt{答:\\锅炉正常运行过程中DCS系统MFT动作信号为0,MFT继电器得电吸合状态,可以。}
\item 简述或画出DCS系统触发MFT动作和操作台MFT急停按钮接线方式?\fenzhi{10}
\wdt{答:\\DCS系统触发MFT动作接线为:三个MFT DO中间继电器分别用两对常闭点组成三取二回路控制MFT继电器\\操作台MFT急停按钮接线方式为:两个急停按钮常闭点并联}

	\item 画出DCS停止给煤机和MFT急停给煤机的接线方式\fenzhi{5}
\wdt{答:\\DCS指令中间继电器常开点与MFT继电器常闭点并联。}
	\item 画出MFT急停磨煤机出口门的接线方式\fenzhi{5}
\wdt{答:\\MFT继电器常闭点串入出口插板门控制回路。}

\item 画出DCS关闭磨煤机热风插板门和MFT关闭磨煤机热风插板门的接线方式\fenzhi{5}
\wdt{答:\\DCS指令中间继电器常开点与MFT继电器常闭点并联,分别接24VDC供电。}
\item 画出DCS关闭油角阀和MFT关闭油角阀的接线方式\fenzhi{5}
\wdt{答:\\DCS指令中间继电器常开点与MFT继电器常开点串联。}
\item 锅炉启动前试验过程中发现部分设备(一次风机、磨煤机、给煤机、油角阀等MFT硬接线联动设备)远方无法正常动作但检查对应设备联锁条均满足,可能原因是什么,应该怎么做?试验过程中发现只有4台磨煤机远方无法启动,可能是原因是什么,该怎么做?\fenzhi{5}
\wdt{答:\\MFT保护首出未复位,MFT旁路按钮复位首出。MFT控制回路电源没电,检查MFT控制回路220VDC电源(两者回答一个即可)\\
控制磨煤机MFT动作的继电器未吸合,检查该MFT继电器未得电原因并恢复。}
\end{enumerate}
		\ifx \allfiles \undefined
\end{document}
\fi
