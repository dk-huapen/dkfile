
\usepackage{geometry}
\geometry{a4paper,left=2cm,right=2cm,top=2.5cm,bottom=2.5cm}   %landscape页面横制,twocolumn双栏
\usepackage{ctex}
\usepackage[bodytextleadingratio=1.5,restoremathleading=true]{zhlineskip}
\lineskiplimit=4pt
\lineskip=5pt

\xeCJKsetup{CJKmath}
\usepackage{amsmath}
\usepackage{amssymb}
\usepackage{amsfonts} 
\usepackage{amsxtra} 
\usepackage[dvipdfm]{graphicx}
\usepackage{tikz,units}
\usetikzlibrary{backgrounds,circuits.ee.IEC.relay}
\usepackage{titlesec,titletoc}%定义章节格式
\titleformat{\chapter}{\zihao {5}}{}{0em}{}%%%
\titleformat{\section}{\zihao {5}\bfseries\linespread{1}\selectfont}{\chinese{section}、}{0em}{}
\titleformat{\subsection}{\linespread{1}\selectfont}{({\bf\chinese{subsection}})}{0em}{\zihao{5}\bfseries\bfseries\linespread{1}\selectfont}{}
\titlespacing*{\chapter}{0pt}{-8mm}{-10mm}
\titlespacing*{\section}{0pt}{0pt}{-0cm}
\titlespacing*{\subsection}{0pt}{0pt}{0pt}




\usepackage{graphbox}
\usepackage{xcolor}
\usepackage{pifont}	%各种小符号

\usepackage[shortlabels]{enumitem}
\setlist{itemsep=0pt,topsep=0pt,partopsep=0pt,parsep=0pt}
\setenumerate[1]{label=\arabic*.,leftmargin=2em,itemindent=0em,labelsep=0.5em,}
\setenumerate[2]{label=(\arabic*),leftmargin=2em,itemindent=0em,labelsep=0em,}


\usepackage{lastpage}%页码

\usepackage{fancyhdr}%设置页眉页脚
\usepackage{eso-pic}
%增加水印
\newcommand{\watermark}[3]{\AddToShipoutPictureBG{
\parbox[b][\paperheight]{\paperwidth}{
\vfill%
\centering%
\tikz[remember picture, overlay]%
  \node [rotate = #1, scale = #2] at (current page.center)%
    {\textcolor{gray!80!cyan!30}{#3}};
\vfill}}}
%%%%%
\fancypagestyle{plain}{\pagestyle{fancy}}
\pagestyle{fancy}
\fancyhf{}
\renewcommand{\chaptermark}[1]{%
	\markboth{#1}{}}
\fancyhead[ER,OL]{\heiti \zihao{-5} 热工专业}
\fancyhead[OR,EL]{\heiti \zihao{-5} \leftmark}
\fancyfoot[CE,CO]{热工技术培训试卷~第~\thepage~页(共 \pageref{LastPage} 页)}
\makeatletter %双线页眉
\def\headrule{{\if@fancyplain\let\headrulewidth\plainheadrulewidth\fi%
		{\color{red}\hrule\@height 1.0pt \@width\headwidth\vskip1pt%上面线为1pt粗
			\hrule\@height 0.5pt\@width\headwidth  }%下面0.5pt粗
		\vskip-2\headrulewidth\vskip-1pt}      %两条线的距离1pt
	\vspace{0.5cm}}     %双线与下面正文之间的垂直间距
\makeatother

\newcommand{\biaoti}[3]{%{\noindent\bf {绝密$\bigstar$启用前 }\vspace*{-0.5em}}
	\begin{center}
		\zihao{3}#1年#2月份#3技术培训考试\\
		%\makebox[5em][s]{\zihao{-2}{\bf #2数学}}{\zihao{4}#3}
\end{center}}
\newcommand{\zysx}{\linespread{1.5}\selectfont{{\noindent\bf 注意事项:}\par 
		1.答卷前,考生务必将自己的姓名、准考证号等填写在答题卡和试卷指定位置上。\par 
		2.回答选择题时,选出每个小题答案后,用铅笔把答题卡上对应题目的答案标号涂黑。如需改动,用橡皮擦干净后,再选涂其他答案标号,回答非选择题时,将答案写在答题卡上。写在本试卷上无效。\par 
		3. 考试结束后,将本试卷和答题卡一并交回。\par }
	
}

\newcommand{\xzt}[3]{选择题:本题共$#1$小题,每小题$#2$分,共$#3$分。在每小题给出的四个选项中,只有一项是符合题目要求的。}
\newcommand{\tkt}[3]{填空题:本题共$#1$小题,每空$#2$分,共$#3$分。}
\newcommand{\jdt}[1]{问答题:共$#1$分。}
\newcommand{\aqjdt}[1]{安全简答题:共$#1$分。}
\newcommand{\stt}[1]{识图题:共$#1$分。}
\newcommand{\htt}[1]{画图题:共$#1$分。}
\newcommand{\xkt}{选考题:共$10$分。请考生在第$22$、$23$题中任选一题作答。如果多做,则按所做的第一题计分。}

\newcommand{\fenzhi}[1]{(#1分)   

}

\usepackage{hlist}
\usepackage{ifthen}
%选择题的4个选项,使用一个命令根据选项内容长度自动排版
\newlength{\lab}
\newlength{\lb}
\newlength{\lc}
\newlength{\ld}
\newlength{\lhalf}
\newlength{\lquarter}
\newlength{\lmax}
\newcommand{\xx}[4]{%%%%%%%%%
	
	\settowidth{\lab}{A.#1}
	\settowidth{\lb}{B.#2}
	\settowidth{\lc}{C.#3}
	\settowidth{\ld}{D.#4}
	\ifthenelse{\lengthtest{\lab > \lb}}  {\setlength{\lmax}{\lab}}  {\setlength{\lmax}{\lb}}
	\ifthenelse{\lengthtest{\lmax < \lc}}  {\setlength{\lmax}{\lc}}  {}
	\ifthenelse{\lengthtest{\lmax < \ld}}  {\setlength{\lmax}{\ld}}  {}
	\setlength{\lhalf}{0.45\linewidth}
	\setlength{\lquarter}{0.2\linewidth}
	\ifthenelse{\lengthtest{\lmax > \lhalf}}
	{%
		\begin{hlist}[pre skip=0pt,item skip=0pt,,item offset={1.5em}, label=\Alpha {hlisti}.,pre label={}]1
			\hitem #1
			\hitem #2
			\hitem #3
			\hitem #4
		\end{hlist}
	}  %%%
	{%%
		\ifthenelse{\lengthtest{\lmax > \lquarter}} % 
		{%
			\begin{hlist}[pre skip=0pt,item skip=0pt,item offset={1.5em}, label=\Alpha {hlisti}.,pre label={}]2
				\hitem #1
				\hitem #2
				\hitem #3
				\hitem #4
			\end{hlist}
		}
		{%
			\begin{hlist}[\parskip=0pt,pre skip=0pt,item skip=0pt,item offset={1.5em}, label=\Alpha {hlisti}.,pre label={}]4
				\hitem #1
				\hitem #2
				\hitem #3
				\hitem #4
			\end{hlist}
}}}

\usepackage{ulem}
\newcommand{\tk}[2][2]{\uline{\makebox[#1cm][c]{%
			\ifanswer
			\textcolor{red}{#2}%
			\else
			\phantom{#2}%
			\fi}}}
		
\newcommand{\xz}[1]{(\  {{%
			\ifanswer
			{\makebox[0.4cm][c]{\textcolor{red}{#1}}}%
			\else
			\phantom{\makebox[0.4cm][c]{#1} }%
			\fi}\  )}}
%\newcommand{\wdt}[1]{\parbox[t][4cm]{15cm}{
\newcommand{\wdt}[2][2]{\parbox[t][#1cm]{16cm}{
			\ifanswer
			\textcolor{red}{#2}%
			\else
			\phantom{#2}%
			\fi}}

\newcommand{\tupian}[3][2]{\parbox[t][#1cm]{16cm}{
			\ifanswer
%\includegraphics[angle=0,width=450pt,trim=0 0 0 0,clip]{#2}\\
			{#2}
			\else
%\includegraphics[angle=0,width=450pt,trim=0 0 0 0,clip]{#3}\\
			{#3}
			\fi}}

\newif\ifanswer
\newcommand{\answer}[1]{\ifthenelse{\isodd{#1}}{\answertrue}{}}

\answer{1}%%为0时正常嗯打印试卷,为1打印带答案试卷
