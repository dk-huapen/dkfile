		\ifx \allfiles \undefined		%编译PPT时注释该行
\documentclass{book}
\usepackage{ifthen}
%\usepackage[a3paper,landscape,showframe,margin=1.25in]{geometry}
%\usepackage[a3paper,landscape,margin=1.1in]{geometry}
\usepackage[a3paper,landscape,top=1.25in,bottom=0.8in,left=1in,right=1in]{geometry}
\usepackage{tikz,units}
\usepackage{circuitikz}
\usepackage{subfig}
\usetikzlibrary{backgrounds,circuits.ee.IEC.relay}
\usetikzlibrary{positioning}
\usepackage{tikzsymbols}
\usepackage{pgffor}
\usetikzlibrary {math}
\usepackage{lastpage}
\usepackage{fancyhdr}
\pagestyle{fancy}
\fancyhf{}
\fancyhead[ER,OL]{\heiti \zihao{-5} 热工专业图纸}
\fancyhead[OR,EL]{\heiti \zihao{-5} \leftmark}
\fancyfoot[CE,CO]{热工专业图纸~第~\thepage~页(共 \pageref{LastPage} 页)}
\renewcommand{\headrulewidth}{0.4pt}
\renewcommand{\footrulewidth}{0.4pt}
\tikzset{
box/.style={rectangle,minimum height=17pt,minimum width=20pt,text=red}
}
\tikzset{
boxA/.style={rectangle,minimum height=17pt,minimum width=20pt,draw=black}
}
\tikzset{
boxB/.style={rectangle,minimum height=17pt,minimum width=30pt,draw=black}
}
\tikzset{
boxC/.style={rectangle,minimum height=17pt,minimum width=120pt,draw=black}
}
\tikzset{
boxD/.style={minimum width=140pt,above left}
}

\begin{document}
		\else						%编译PPT时注释该行
		%\newpage
		\fi						%编译PPT时注释该行
\watermark{50}{9}{热工班组}
\chapter[2025年07月份技术培训考试]{	\hspace*{-0.3em}\biaoti{2025}{07}{热工专业}}
姓名:\uline{ \ \  \  \ \ \ \ \ \ \ \ \ \ \ \ \ \ }\hfill 得分:\uline{ \ \  \  \ \ \ \ \ \  \ \ \ \ \ \ }
%\zysx
\section{\xzt{5}{2}{10}}
\begin{enumerate}
\item 我厂给煤机二次表供电来至\xz{B}
\xx{配电室UPS段220VAC电源}{电子间机柜内24VDC电源}{就地控制柜内电源模块输出24VDC}{配电室MCC段380VAC电源}
\item 我厂磨煤机出口插板门控制回路供电来至\xz{B}
\xx{配电室UPS段220VAC电源}{电子间机柜内24VDC电源}{就地控制柜内电源模块输出24VDC}{配电室MCC段380VAC电源}
\item 我厂3汽号轮机的AST电磁阀供电电压是\xz{D}
\xx{220ACV}{220DCV}{380AVC}{24DCV}
\item 我厂4汽号轮机的AST电磁阀供电电压是\xz{B}
\xx{220ACV}{220DCV}{380AVC}{24DCV}
\item 我厂脱硫净烟气CEMS二氧化硫和氮氧化物分析仪采用\xz{A}测量方法。
\xx{稀释法}{抽取式}{激光后散射法}{等速采样}
\end{enumerate}
\section{\tkt{6}{2}{30}}
\begin{enumerate}
	\setcounter{enumi}{5}
\item 我厂给煤机有四个程序可实现基本标定功能分别为皮带整圈脉冲数设定、\tk[2]{去皮程序}、\tk[2]{零点设定}和\\ \tk[2.5]{量程标定程序},这三个程序都是在\tk{容积控制}模式\tk[4]{额定流量或20t/h}下进行的。
\item 我厂给煤机变频器故障就是使变频器停止的条件,有两种故障类型,分别是\tk{可自动复位}和\tk{不可自动复位}。
	\item 我厂3号汽轮机高调门伺服阀为\tk{两级}、\tk{先导}电液伺服阀。	
	\item 我厂4号汽轮机高调门伺服阀为\tk{单级}、\tk{直驱}电液伺服阀。
	\item 2号汽轮机高调门伺服阀输入电流信号范围\tk{20-160mA},3号汽轮机高调门伺服卡输入电流信号范围\tk{4-20mA}。
\item 我厂净烟气粉尘,SO2,NOX折算时,如果氧量值大于6\%,则折算值\tk{大于}干基值,干基值是指烟气经预处理,露点温度小于等于\tk{4}℃时,烟气中各污染物的浓度。
\end{enumerate}
\section{\aqjdt{15}}
\begin{enumerate}
	\setcounter{enumi}{11}
	\item 员工安全行为十不准?\fenzhi{7}
\wdt[2]{答:1.不准违章操作、违章指挥;2.不准班前、班中饮酒;3.不准脱岗、睡岗;4.不准开超速车;5.不准随意进入要害部位;6.不准擅自开动各种开关、阀门和设备;7.不准穿戴不规范防护用品上岗;8.不准在起吊物下行走或逗留;9.不准在厂房内奔跑;10.不准在厂内燃放烟花爆竹。}\\
	\item 防触电伤害中迅速脱离电源的方法有哪些?\fenzhi{8}
\wdt[4]{答:1.发现有人触电时,应立即拉闸停电。距电闸较远时,可使用绝缘钳或干燥木柄斧子切断电源; 2.在电容器或电缆线路中解救时,应切断电源进行放电后,再去救护触电人员; 3.救护人员不得用手拉或用金属棒、潮湿物品救护,应使用绝缘器具使触电人员脱离电源; 4.高压触电,应在保救护人员的安全情况下,因地制宜采用相应救护措施。解救触电人员时,要做好防护,以免触电人员再受二次伤害。 }\\
\end{enumerate}
\section{\jdt{45}}
\begin{enumerate}
	\setcounter{enumi}{13}
\item 给煤机作为重点设备,重点巡检内容有哪些?常见故障有哪些?简述给煤机风扇故障应急处置方法?\fenzhi{20}
\wdt[6]{答:\\
给煤机变频器是否有报警,风扇工作是否正常(3分)
给煤机二次表是否有报警,流量/转速是否有明显跳变现象(3分)
变频器风扇不转导致变频器超温故障(3分)
给煤机二次表未切至运行模式,断电后未手动启动导致给煤机启动不转(3分)
给煤机风散应急处置方法:备用风散在工程师站第三个玻璃文件柜内,处置的时候拿上手持风机,风机吹变频器冷却。首先到达现场后,断电复位变频器消除故障,风机冷却,让运行启动给煤机,启动后在线更换风散,减少给煤机失备时间。(8分)
}
	\item 画出我厂锅炉本体吹灰器二次回路控制原理图\fenzhi{25}

\end{enumerate}
		\ifx \allfiles \undefined
\end{document}
\fi
