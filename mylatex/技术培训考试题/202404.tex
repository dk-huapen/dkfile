		\ifx \allfiles \undefined		%编译PPT时注释该行
\documentclass{book}
\usepackage{ifthen}
%\usepackage[a3paper,landscape,showframe,margin=1.25in]{geometry}
%\usepackage[a3paper,landscape,margin=1.1in]{geometry}
\usepackage[a3paper,landscape,top=1.25in,bottom=0.8in,left=1in,right=1in]{geometry}
\usepackage{tikz,units}
\usepackage{circuitikz}
\usepackage{subfig}
\usetikzlibrary{backgrounds,circuits.ee.IEC.relay}
\usetikzlibrary{positioning}
\usepackage{tikzsymbols}
\usepackage{pgffor}
\usetikzlibrary {math}
\usepackage{lastpage}
\usepackage{fancyhdr}
\pagestyle{fancy}
\fancyhf{}
\fancyhead[ER,OL]{\heiti \zihao{-5} 热工专业图纸}
\fancyhead[OR,EL]{\heiti \zihao{-5} \leftmark}
\fancyfoot[CE,CO]{热工专业图纸~第~\thepage~页(共 \pageref{LastPage} 页)}
\renewcommand{\headrulewidth}{0.4pt}
\renewcommand{\footrulewidth}{0.4pt}
\tikzset{
box/.style={rectangle,minimum height=17pt,minimum width=20pt,text=red}
}
\tikzset{
boxA/.style={rectangle,minimum height=17pt,minimum width=20pt,draw=black}
}
\tikzset{
boxB/.style={rectangle,minimum height=17pt,minimum width=30pt,draw=black}
}
\tikzset{
boxC/.style={rectangle,minimum height=17pt,minimum width=120pt,draw=black}
}
\tikzset{
boxD/.style={minimum width=140pt,above left}
}

\begin{document}
		\else						%编译PPT时注释该行
		%\newpage
		\fi						%编译PPT时注释该行
\watermark{50}{9}{热工班组}
\chapter[2024年04月份技术培训考试]{	\hspace*{-0.3em}\biaoti{2024}{04}{热工专业}}
姓名:\uline{ \ \  \  \ \ \ \ \ \ \ \ \ \ \ \ \ \ }\hfill 得分:\uline{ \ \  \  \ \ \ \ \ \  \ \ \ \ \ \ }
%\zysx
\section{\xzt{5}{2}{10}}
\begin{enumerate}
\item 我厂3汽号轮机的AST电磁阀供电电压是\xz{D}。
\xx{220ACV}{220DCV}{380AVC}{24DCV}
\item 我厂3汽号轮机的AST电磁阀供电端子保险是\xz{C}。
\xx{0.2A}{1A}{3A}{0.1A}
\item 我厂4汽号轮机的AST电磁阀供电电压是\xz{B}。
\xx{220ACV}{220DCV}{380AVC}{24DCV}
\item 我厂4汽号轮机的主汽门电磁阀供电电压是\xz{A}。
\xx{220ACV}{220DCV}{380AVC}{24DCV}
\item 我厂次高压汽轮机ETS系统S7 200PLC与DEH系统之间数据传递采用哪种方式\xz{D}。
\xx{OPC通讯}{RS 232/485通讯}{4-20mA}{PROFIBUS通讯}
\end{enumerate}
\section{\tkt{4}{2}{30}}
\begin{enumerate}
	\setcounter{enumi}{5}
	\item 我厂次高压汽轮机ETS系统控制器使用的PLC型号为CPU 224 CN AC/DC/RLY,其中AC/DC/RLY分别对应着\tk{电源电压}/\tk{输入电压}/\tk{输出电压},AC表示交流\tk{220V},DC表示\tk{直流24V},RLY表示\tk{继电器输出},通过\tk{并联输入、输出}实现控制系统冗余。
\item 汽轮机常规的超速保护有\tk{OPC}、\tk{电超速}、\tk{机械超速}三种。
\item 我厂次高压汽轮机正常运行时,汽轮机的润滑油压由\tk{主油泵}供给,当润滑油压降低时应自动联动\tk{交流润滑油泵}。
\item 电涡流传感器系统的三个组成部分是\tk{探头}、\tk{延长电缆}、\tk{前置器}。
\end{enumerate}
\section{\stt{20}}
\begin{enumerate}
	\setcounter{enumi}{9}
	\item 下图为我厂次高压汽轮机AST模块油回路图,根据下图回答一下问题
\\图中压力油和安全油作用分别是什么?\fenzhi{5}
\wdt[3]{答:}\\
\\图中1YV/2YV/3YV/4YV代表什么设备且电源分布情况?\fenzhi{5}
\wdt[3]{答:}\\
电磁阀什么状态下AST动作?\fenzhi{5}
\wdt[3]{答:}\\
图中压力表代表什么油压?作用是什么?\fenzhi{5}
\wdt[3]{答:}\\
%\includegraphics[angle=0,width=500pt,trim=0 0 0 0,clip]{picture/200plc.png}%答案需要替换下图片
\includegraphics[angle=0,width=500pt,trim=0 0 0 0,clip]{picture/AST.png}%带答案图片

\end{enumerate}
\section{\jdt{40}}
\begin{enumerate}
	\setcounter{enumi}{10}
\item 我厂次高压汽轮机ETS保护有哪些?其中DCS请求停机保护包含哪些?DEH请求停机保护包含哪些?\fenzhi{20}
\wdt[6]{答:\\
汽轮机轴承振动大:现场8支电涡流探头通过前置器输出电压信号至TSI系统,TSI系统对应卡件分别输出4-20mA信号至DCS系统后进行逻辑判断后触发DCS停机信号
主油箱油位低:现场油位测量装置通过TSI机柜内变送器输出信号至TSI系统,TSI系统对应卡件分别输出4-20mA信号至DCS系统后进行逻辑判断后触发DCS停机信号
轴承金属温度高与回油温度高:现场热电阻元件输出信号至TSI系统对应卡件后输出4-20mA信号至DCS系统后进行逻辑判断后触发DCS停机信号
}
\item 我厂次高压汽轮机ETS保护动作信号分别送至哪些设备?在ETS控制器故障情况下操作台急停按钮是否能正常触发ETS保护?DCS请求停机首出怎么复位?\fenzhi{20}
\wdt[6]{答:\\
汽轮机轴承振动大:现场8支电涡流探头通过前置器输出电压信号至TSI系统,TSI系统对应卡件分别输出4-20mA信号至DCS系统后进行逻辑判断后触发DCS停机信号
主油箱油位低:现场油位测量装置通过TSI机柜内变送器输出信号至TSI系统,TSI系统对应卡件分别输出4-20mA信号至DCS系统后进行逻辑判断后触发DCS停机信号
轴承金属温度高与回油温度高:现场热电阻元件输出信号至TSI系统对应卡件后输出4-20mA信号至DCS系统后进行逻辑判断后触发DCS停机信号
}
\end{enumerate}
		\ifx \allfiles \undefined
\end{document}
\fi
